\chapter{Maximal Elastic Slow Diversity}
%%%%%%%%%%%%%%%%%%%%%%%%%%%%%%%%%%%%%%%%%%%%%%%%%%%%%%%%%%%%%%%%%%%%%%%%%%%%%%%%%%%%%%%%%%%%%%%%%%%%%%%%%%%%%%%%%%%%
In this chapter I will consider the kinematics of a 2-to-2 scattering process that has maximal elastic slow diversity. This means that all external quanta are slow, but you have two distinct elastic pairs:
\begin{equation}
	m_{1}^{2} = -\abs{p_{1}}^{2} = -\abs{p_{3}}^{2}, \qquad m_{2}^{2} = -\abs{p_{2}}^{2} = -\abs{p_{4}}^{2}.
\end{equation}
Thus, this is an elastic process. For example,
\begin{equation}
	A(p_{1}) + B(p_{2}) \longrightarrow A(p_{3}) + B(p_{4}).
\end{equation}
For this process, (\ref{eq:stu_slow}) becomes
\begin{equation}
	s + t + u = 2m_{1}^{2} + 2m_{2}^{2}.
\end{equation}
%%%%%%%%%%%%%%%%%%%%%%%%%%%%%%%%%%%%%%%%%%%%%%%%%%%%%%%%%%%%%%%%%%%%%%%%%%%%%%%%%%%%%%%%%%%%%%%%%%%%%%%%%%%%%%%%%%%%
\section{Simplicial Invariants}
%%%%%%%%%%%%%%%%%%%%%%%%%%%%%%%%%%%%%%%%%%%%%%%%%%%%%%%%%%%%%%%%%%%%%%%%%%%%%%%%%%%%%%%%%%%%%%%%%%%%%%%%%%%%%%%%%%%%
Since this scattering process is elastic, there are fewer distinct values among the simplicial invariants.
%%%%%%%%%%%%%%%%%%%%%%%%%%%%%%%%%%%%%%%%%%%%%%%%%%%%%%%%%%%%%%%%%%%%%%%%%%%%%%%%%%%%%%%%%%%%%%%%%%%%%%%%%%%%%%%%%%%%
\subsection{1-Simplex Invariants}
%%%%%%%%%%%%%%%%%%%%%%%%%%%%%%%%%%%%%%%%%%%%%%%%%%%%%%%%%%%%%%%%%%%%%%%%%%%%%%%%%%%%%%%%%%%%%%%%%%%%%%%%%%%%%%%%%%%%
For the \red{red} tetrahedron you have
\begin{equation}
\begin{split}
	C_{\red{12}} = m_{1}^{2}, \qquad C_{\red{13}} &= s, \qquad C_{\red{14}} = m_{1}^{2}, \\
	C_{\red{23}} = m_{2}^{2}, \qquad C_{\red{24}} &= t, \qquad C_{\red{34}} = m_{2}^{2}. \\
\end{split}
\end{equation}
Similarly, for the \blue{blue} tetrahedron you have
\begin{equation}
\begin{split}
	C_{\blue{12}} = m_{1}^{2}, \qquad C_{\blue{13}} &= t, \qquad C_{\blue{14}} = m_{2}^{2}, \\
	C_{\blue{23}} = m_{1}^{2}, \qquad C_{\blue{24}} &= u, \qquad C_{\blue{34}} = m_{2}^{2}. \\
\end{split}
\end{equation}
Finally, for the \green{green} tetrahedron you have
\begin{equation}
\begin{split}
	C_{\green{12}} = m_{2}^{2}, \qquad C_{\green{13}} &= u, \qquad C_{\green{14}} = m_{1}^{2}, \\
	C_{\green{23}} = m_{1}^{2}, \qquad C_{\green{24}} &= s, \qquad C_{\green{34}} = m_{2}^{2}. \\
\end{split}
\end{equation}
%%%%%%%%%%%%%%%%%%%%%%%%%%%%%%%%%%%%%%%%%%%%%%%%%%%%%%%%%%%%%%%%%%%%%%%%%%%%%%%%%%%%%%%%%%%%%%%%%%%%%%%%%%%%%%%%%%%%
\subsection{2-Simplex Invariants}
%%%%%%%%%%%%%%%%%%%%%%%%%%%%%%%%%%%%%%%%%%%%%%%%%%%%%%%%%%%%%%%%%%%%%%%%%%%%%%%%%%%%%%%%%%%%%%%%%%%%%%%%%%%%%%%%%%%%
For the \red{red} tetrahedron you have
\begin{equation}
\begin{split}
	C_{\red{123}} &= \kallen{s}{m_{1}}{m_{2}}, \\
	C_{\red{124}} &= t \left[t - 4 m_{1}^{2} \right], \\
	C_{\red{134}} &= \kallen{s}{m_{1}}{m_{2}}, \\
	C_{\red{234}} &= t \left[t - 4 m_{2}^{2} \right].
\end{split}
\end{equation}
Similarly, for the \blue{blue} tetrahedron you have
\begin{equation}
\begin{split}
	C_{\blue{123}} &= t \left[t - 4 m_{1}^{2} \right], \\
	C_{\blue{124}} &= \kallen{u}{m_{1}}{m_{2}}, \\
	C_{\blue{134}} &= t \left[t - 4 m_{2}^{2} \right], \\
	C_{\blue{234}} &= \kallen{u}{m_{1}}{m_{2}}.
\end{split}
\end{equation}
Finally, for the \green{green} tetrahedron you have
\begin{equation}
\begin{split}
	C_{\green{123}} &= \kallen{u}{m_{1}}{m_{2}}, \\
	C_{\green{124}} &= \kallen{s}{m_{1}}{m_{2}}, \\
	C_{\green{134}} &= \kallen{u}{m_{1}}{m_{2}}, \\
	C_{\green{234}} &= \kallen{s}{m_{1}}{m_{2}}.
\end{split}
\end{equation}
%%%%%%%%%%%%%%%%%%%%%%%%%%%%%%%%%%%%%%%%%%%%%%%%%%%%%%%%%%%%%%%%%%%%%%%%%%%%%%%%%%%%%%%%%%%%%%%%%%%%%%%%%%%%%%%%%%%%
\subsection{3-Simplex Invariants}
%%%%%%%%%%%%%%%%%%%%%%%%%%%%%%%%%%%%%%%%%%%%%%%%%%%%%%%%%%%%%%%%%%%%%%%%%%%%%%%%%%%%%%%%%%%%%%%%%%%%%%%%%%%%%%%%%%%%
Each tetrahedron has only one possible 3-simplex invariant. Indeed,
\begin{equation}
	C_{\red{1234}} = C_{\blue{1234}} = C_{\green{1234}} = t \left[s u - \left(m_{1} - m_{2}\right)^{2} \left(m_{1} + m_{2}\right)^{2} \right].
\end{equation}
%%%%%%%%%%%%%%%%%%%%%%%%%%%%%%%%%%%%%%%%%%%%%%%%%%%%%%%%%%%%%%%%%%%%%%%%%%%%%%%%%%%%%%%%%%%%%%%%%%%%%%%%%%%%%%%%%%%%
\section{Dual Conformal Invariants}
%%%%%%%%%%%%%%%%%%%%%%%%%%%%%%%%%%%%%%%%%%%%%%%%%%%%%%%%%%%%%%%%%%%%%%%%%%%%%%%%%%%%%%%%%%%%%%%%%%%%%%%%%%%%%%%%%%%%
For the \red{red} tetrahedron, you have
\begin{equation}
\begin{split}
	\crat{\red{1}}{\red{2}}{\red{3}}{\red{4}} = \frac{s t}{m_{1}^{2} m_{2}^{2}}, \qquad
	\crat{\red{1}}{\red{3}}{\red{4}}{\red{2}} &= 1, \qquad
	\crat{\red{1}}{\red{4}}{\red{2}}{\red{3}} = \frac{m_{1}^{2} m_{2}^{2}}{s t}, \\
	\crat{\red{1}}{\red{2}}{\red{4}}{\red{3}} = \frac{m_{1}^{2} m_{2}^{2}}{s t}, \qquad
	\crat{\red{1}}{\red{3}}{\red{2}}{\red{4}} &= 1, \qquad
	\crat{\red{1}}{\red{4}}{\red{3}}{\red{2}} = \frac{s t}{m_{1}^{2} m_{2}^{2}}.
\end{split}
\end{equation}
Similarly, for the \blue{blue} tetrahedron, you have
\begin{equation}
\begin{split}
	\crat{\blue{1}}{\blue{2}}{\blue{3}}{\blue{4}} = \frac{t u}{m_{1}^{2} m_{2}^{2}}, \qquad
	\crat{\blue{1}}{\blue{3}}{\blue{4}}{\blue{2}} &= 1, \qquad
	\crat{\blue{1}}{\blue{4}}{\blue{2}}{\blue{3}} = \frac{m_{1}^{2} m_{2}^{2}}{t u}, \\
	\crat{\blue{1}}{\blue{2}}{\blue{4}}{\blue{3}} = \frac{m_{1}^{2} m_{2}^{2}}{t u}, \qquad
	\crat{\blue{1}}{\blue{3}}{\blue{2}}{\blue{4}} &= 1, \qquad
	\crat{\blue{1}}{\blue{4}}{\blue{3}}{\blue{2}} = \frac{t u}{m_{1}^{2} m_{2}^{2}}.
\end{split}
\end{equation}
Finally, for the \green{green} tetrahedron, you have
\begin{equation}
\begin{split}
	\crat{\green{1}}{\green{2}}{\green{3}}{\green{4}} = \frac{s u}{m_{1}^{4}}, \qquad
	\crat{\green{1}}{\green{3}}{\green{4}}{\green{2}} &= \frac{m_{1}^{4}}{m_{2}^{4}}, \qquad
	\crat{\green{1}}{\green{4}}{\green{2}}{\green{3}} = \frac{m_{2}^{4}}{s u}, \\
	\crat{\green{1}}{\green{2}}{\green{4}}{\green{3}} = \frac{m_{1}^{4}}{s u}, \qquad
	\crat{\green{1}}{\green{3}}{\green{2}}{\green{4}} &= \frac{m_{2}^{4}}{m_{1}^{4}}, \qquad
	\crat{\green{1}}{\green{4}}{\green{3}}{\green{2}} = \frac{s u}{m_{2}^{4}}.
\end{split}
\end{equation}
Note that
\begin{equation}
	\crat{1}{2}{3}{4} \crat{1}{3}{4}{2} \crat{1}{4}{2}{3} = 1,
\end{equation}
which is analogous to the constraint satisfied by the three Mandelstam invariants.
%%%%%%%%%%%%%%%%%%%%%%%%%%%%%%%%%%%%%%%%%%%%%%%%%%%%%%%%%%%%%%%%%%%%%%%%%%%%%%%%%%%%%%%%%%%%%%%%%%%%%%%%%%%%%%%%%%%%
\section{Center-of-Momentum Frame}
%%%%%%%%%%%%%%%%%%%%%%%%%%%%%%%%%%%%%%%%%%%%%%%%%%%%%%%%%%%%%%%%%%%%%%%%%%%%%%%%%%%%%%%%%%%%%%%%%%%%%%%%%%%%%%%%%%%%
In the center-of-momentum frame you write the energy-momentum vectors as
\begin{equation}
	p_{1} = \begin{pmatrix} E_{1} & \mathbf{p}_{1} \end{pmatrix}, \qquad p_{2} = \begin{pmatrix} E_{2} & -\mathbf{p}_{1} \end{pmatrix}, \qquad p_{3} = \begin{pmatrix} E_{3} & \mathbf{p}_{3} \end{pmatrix}, \qquad p_{4} = \begin{pmatrix} E_{4} & -\mathbf{p}_{3} \end{pmatrix}.
\end{equation}
One of the first things to notice is that
\begin{equation}
	s = (E_{1} + E_{2})^{2} = (E_{3} + E_{4})^{2}.
\end{equation}
Thus, $s$ can be interpreted as the total energy in the center-of-momentum frame. It also follows that $s$ must be positive.

From the on-shell constraints, it follows that
\begin{equation}
\begin{split}
	E_{1} = \sqrt{m_{1}^{2} + \abs{\mathbf{p}_{1}}^{2}}, &\qquad E_{2} = \sqrt{m_{2}^{2} + \abs{\mathbf{p}_{1}}^{2}}, \\
	E_{3} = \sqrt{m_{1}^{2} + \abs{\mathbf{p}_{3}}^{2}}, &\qquad E_{4} = \sqrt{m_{2}^{2} + \abs{\mathbf{p}_{3}}^{2}}.
\end{split}
\end{equation}
Using the relations
\begin{equation}
	E_{2} = \sqrt{s} - E_{1}, \qquad E_{4} = \sqrt{s} - E_{3},
\end{equation}
you find that
\begin{equation}
	\abs{\mathbf{p}_{1}} = \abs{\mathbf{p}_{3}} = \frac{\sqrt{\Lambda_{12}(s)}}{2 \sqrt{s}}.
\end{equation}
Thus,
\begin{equation}
	E_{1} = E_{3} = \frac{s + m_{1}^{2} - m_{2}^{2}}{2 \sqrt{s}}, \qquad E_{2} = E_{4} = \frac{s - m_{1}^{2} + m_{2}^{2}}{2 \sqrt{s}}.
\end{equation}
Note that
\begin{equation}
	E_{1} - E_{2} = \frac{m_{1}^{2} - m_{2}^{2}}{\sqrt{s}}.
\end{equation}
%%%%%%%%%%%%%%%%%%%%%%%%%%%%%%%%%%%%%%%%%%%%%%%%%%%%%%%%%%%%%%%%%%%%%%%%%%%%%%%%%%%%%%%%%%%%%%%%%%%%%%%%%%%%%%%%%%%%
\subsection{Speed and Rapidity}
%%%%%%%%%%%%%%%%%%%%%%%%%%%%%%%%%%%%%%%%%%%%%%%%%%%%%%%%%%%%%%%%%%%%%%%%%%%%%%%%%%%%%%%%%%%%%%%%%%%%%%%%%%%%%%%%%%%%
The speed of each external quantum is
\begin{align}
	\abs{\mathbf{v}_{1}} &= \abs{\mathbf{v}_{3}} = \frac{\sqrt{\Lambda_{12}(s)}}{s + m_{1}^{2} - m_{2}^{2}}, \\
	\abs{\mathbf{v}_{2}} &= \abs{\mathbf{v}_{4}} = \frac{\sqrt{\Lambda_{12}(s)}}{s - m_{1}^{2} + m_{2}^{2}}.
\end{align}
The rapidity of each external quantum is
\begin{align}
	\eta_{1} &= \eta_{3} = \frac{1}{2} \log{\left[\frac{s + m_{1}^{2} - m_{2}^{2} + \sqrt{\Lambda_{12}(s)}}{s + m_{1}^{2} - m_{2}^{2} - \sqrt{\Lambda_{12}(s)}}\right]}, \\
	\eta_{2} &= \eta_{4} = \frac{1}{2} \log{\left[\frac{s - m_{1}^{2} + m_{2}^{2} + \sqrt{\Lambda_{12}(s)}}{s - m_{1}^{2} + m_{2}^{2} - \sqrt{\Lambda_{12}(s)}}\right]}.
\end{align}
The sum of the incoming rapidites is equal to the sum of the outgoing rapidities:
\begin{equation}
	\eta_{1} + \eta_{2} = \eta_{3} + \eta_{4} = \frac{1}{2} \log{\left[ \frac{s - m_{1}^{2} - m_{2}^{2} + \sqrt{\Lambda_{12}(s)}}{s - m_{1}^{2} - m_{2}^{2} - \sqrt{\Lambda_{12}(s)}} \right]}.
\end{equation}
Using
\begin{equation}
	\frac{s - (m_{1} + m_{2})^{2} + \sqrt{\Lambda_{12}(s)}}{s - (m_{1} + m_{2})^{2} - \sqrt{\Lambda_{12}(s)}} = \frac{2 m_{1} m_{2}}{m_{1}^{2} + m_{2}^{2} - s + \sqrt{\Lambda_{12}(s)}} = \frac{m_{1}^{2} + m_{2}^{2} - s - \sqrt{\Lambda_{12}(s)}}{2 m_{1} m_{2}},
\end{equation}
allows you to write
\begin{equation}
	\eta_{1} + \eta_{2} = \log{\left[ \frac{s - (m_{1} + m_{2})^{2} + \sqrt{\Lambda_{12}(s)}}{s - (m_{1} + m_{2})^{2} - \sqrt{\Lambda_{12}(s)}} \right]}.
\end{equation}
%%%%%%%%%%%%%%%%%%%%%%%%%%%%%%%%%%%%%%%%%%%%%%%%%%%%%%%%%%%%%%%%%%%%%%%%%%%%%%%%%%%%%%%%%%%%%%%%%%%%%%%%%%%%%%%%%%%%
\subsection{Physical Scattering Region}
%%%%%%%%%%%%%%%%%%%%%%%%%%%%%%%%%%%%%%%%%%%%%%%%%%%%%%%%%%%%%%%%%%%%%%%%%%%%%%%%%%%%%%%%%%%%%%%%%%%%%%%%%%%%%%%%%%%%
...
%%%%%%%%%%%%%%%%%%%%%%%%%%%%%%%%%%%%%%%%%%%%%%%%%%%%%%%%%%%%%%%%%%%%%%%%%%%%%%%%%%%%%%%%%%%%%%%%%%%%%%%%%%%%%%%%%%%%
\subsection{Scattering Regimes}
%%%%%%%%%%%%%%%%%%%%%%%%%%%%%%%%%%%%%%%%%%%%%%%%%%%%%%%%%%%%%%%%%%%%%%%%%%%%%%%%%%%%%%%%%%%%%%%%%%%%%%%%%%%%%%%%%%%%
...
%%%%%%%%%%%%%%%%%%%%%%%%%%%%%%%%%%%%%%%%%%%%%%%%%%%%%%%%%%%%%%%%%%%%%%%%%%%%%%%%%%%%%%%%%%%%%%%%%%%%%%%%%%%%%%%%%%%%
\subsubsection{Slow Small-Speed Scattering}
%%%%%%%%%%%%%%%%%%%%%%%%%%%%%%%%%%%%%%%%%%%%%%%%%%%%%%%%%%%%%%%%%%%%%%%%%%%%%%%%%%%%%%%%%%%%%%%%%%%%%%%%%%%%%%%%%%%%
Small slow speeds occur when $s$ is close to the threshold value $(m_{1} + m_{2})^{2}$.
%%%%%%%%%%%%%%%%%%%%%%%%%%%%%%%%%%%%%%%%%%%%%%%%%%%%%%%%%%%%%%%%%%%%%%%%%%%%%%%%%%%%%%%%%%%%%%%%%%%%%%%%%%%%%%%%%%%%
\subsubsection{Slow Fixed-Speed Scattering}
%%%%%%%%%%%%%%%%%%%%%%%%%%%%%%%%%%%%%%%%%%%%%%%%%%%%%%%%%%%%%%%%%%%%%%%%%%%%%%%%%%%%%%%%%%%%%%%%%%%%%%%%%%%%%%%%%%%%
The speed of each external quantum can be written as a function of two dimensionless ratios
\begin{equation}
	\frac{s}{m_{1} m_{2}}, \qquad \frac{m_{1}}{m_{2}}.
\end{equation}
Slow fixed-speed scattering corresponds to the kinematic regime where we keep these ratios fixed:
\begin{equation}
	\frac{s}{m_{1} m_{2}} \text{ fixed}, \qquad \frac{m_{1}}{m_{2}} \text{ fixed}.
\end{equation}
This regime is appropriate for either large $s$ and large masses, or small $s$ and small masses. Note that fixed-speed is equivalent to fixed-rapidity. By itself, this regime is not very helpful, but when combined with other limits it leads to important approximations.
%%%%%%%%%%%%%%%%%%%%%%%%%%%%%%%%%%%%%%%%%%%%%%%%%%%%%%%%%%%%%%%%%%%%%%%%%%%%%%%%%%%%%%%%%%%%%%%%%%%%%%%%%%%%%%%%%%%%
\subsubsection{Slow Large-Speed Scattering}
%%%%%%%%%%%%%%%%%%%%%%%%%%%%%%%%%%%%%%%%%%%%%%%%%%%%%%%%%%%%%%%%%%%%%%%%%%%%%%%%%%%%%%%%%%%%%%%%%%%%%%%%%%%%%%%%%%%%
Large-speed scattering involves the regime
\begin{equation}
	\frac{s}{m_{1} m_{2}} \rightarrow \infty, \qquad \frac{m_{1}}{m_{2}} \text{ fixed}.
\end{equation}
That is, $s$ is very large compared to the masses.
%%%%%%%%%%%%%%%%%%%%%%%%%%%%%%%%%%%%%%%%%%%%%%%%%%%%%%%%%%%%%%%%%%%%%%%%%%%%%%%%%%%%%%%%%%%%%%%%%%%%%%%%%%%%%%%%%%%%
\subsubsection{Regge Scattering}
%%%%%%%%%%%%%%%%%%%%%%%%%%%%%%%%%%%%%%%%%%%%%%%%%%%%%%%%%%%%%%%%%%%%%%%%%%%%%%%%%%%%%%%%%%%%%%%%%%%%%%%%%%%%%%%%%%%%
Regge scattering is the regime of fixed-speed and large (unphysical) $z_{13}$. This corresponds to
\begin{equation}
	\frac{t}{s} \rightarrow \infty, \qquad \frac{s}{m_{1} m_{2}} \text{ fixed}, \qquad \frac{m_{1}}{m_{2}} \text{ fixed}.
\end{equation}
As a corollary, you have
\begin{equation}
	u = 2m_{1}^{2} + 2m_{2}^{2} - s - t \quad \Longrightarrow \quad \frac{u}{s} \rightarrow -\infty.
\end{equation}
In this regime all dual conformal invariants become trivial:
\begin{equation}
\begin{split}
	\crat{\red{1}}{\red{2}}{\red{3}}{\red{4}} \rightarrow \infty, \qquad
	\crat{\red{1}}{\red{3}}{\red{4}}{\red{2}} &= 1, \qquad
	\crat{\red{1}}{\red{4}}{\red{2}}{\red{3}} \rightarrow 0, \\
	\crat{\red{1}}{\red{2}}{\red{4}}{\red{3}} \rightarrow 0, \qquad
	\crat{\red{1}}{\red{3}}{\red{2}}{\red{4}} &= 1, \qquad
	\crat{\red{1}}{\red{4}}{\red{3}}{\red{2}} \rightarrow \infty.
\end{split}
\end{equation}
\begin{equation}
\begin{split}
	\crat{\blue{1}}{\blue{2}}{\blue{3}}{\blue{4}} \rightarrow -\infty, \qquad
	\crat{\blue{1}}{\blue{3}}{\blue{4}}{\blue{2}} &= 1, \qquad
	\crat{\blue{1}}{\blue{4}}{\blue{2}}{\blue{3}} \rightarrow 0, \\
	\crat{\blue{1}}{\blue{2}}{\blue{4}}{\blue{3}} \rightarrow 0, \qquad
	\crat{\blue{1}}{\blue{3}}{\blue{2}}{\blue{4}} &= 1, \qquad
	\crat{\blue{1}}{\blue{4}}{\blue{3}}{\blue{2}} \rightarrow -\infty.
\end{split}
\end{equation}
\begin{equation}
\begin{split}
	\crat{\green{1}}{\green{2}}{\green{3}}{\green{4}} \rightarrow -\infty, \qquad
	\crat{\green{1}}{\green{3}}{\green{4}}{\green{2}} &= \frac{m_{1}^{4}}{m_{2}^{4}}, \qquad
	\crat{\green{1}}{\green{4}}{\green{2}}{\green{3}} \rightarrow 0, \\
	\crat{\green{1}}{\green{2}}{\green{4}}{\green{3}} \rightarrow 0, \qquad
	\crat{\green{1}}{\green{3}}{\green{2}}{\green{4}} &= \frac{m_{2}^{4}}{m_{1}^{4}}, \qquad
	\crat{\green{1}}{\green{4}}{\green{3}}{\green{2}} \rightarrow -\infty.
\end{split}
\end{equation}
%%%%%%%%%%%%%%%%%%%%%%%%%%%%%%%%%%%%%%%%%%%%%%%%%%%%%%%%%%%%%%%%%%%%%%%%%%%%%%%%%%%%%%%%%%%%%%%%%%%%%%%%%%%%%%%%%%%%
\subsubsection{Forward Scattering}
%%%%%%%%%%%%%%%%%%%%%%%%%%%%%%%%%%%%%%%%%%%%%%%%%%%%%%%%%%%%%%%%%%%%%%%%%%%%%%%%%%%%%%%%%%%%%%%%%%%%%%%%%%%%%%%%%%%%
Forward scattering is the regime of fixed-speed and small (physical) scattering angles (i.e. $z_{13} \rightarrow 1$). This can be stated as
\begin{equation}
	\frac{t}{s} \rightarrow 0, \qquad \frac{s}{m_{1} m_{2}} \text{ fixed}, \qquad \frac{m_{1}}{m_{2}} \text{ fixed}.
\end{equation}
As a corollary, you have
\begin{equation}
	u = 2m_{1}^{2} + 2m_{2}^{2} - s - t \quad \Longrightarrow \quad \frac{u}{s} \text{ fixed}.
\end{equation}
Unlike the case of Regge scattering, in the forward regime only some of the dual conformal invariants are trivial:
\begin{equation}
\begin{split}
	\crat{\red{1}}{\red{2}}{\red{3}}{\red{4}} \rightarrow 0, \qquad
	\crat{\red{1}}{\red{3}}{\red{4}}{\red{2}} &= 1, \qquad
	\crat{\red{1}}{\red{4}}{\red{2}}{\red{3}} \rightarrow \infty, \\
	\crat{\red{1}}{\red{2}}{\red{4}}{\red{3}} \rightarrow \infty, \qquad
	\crat{\red{1}}{\red{3}}{\red{2}}{\red{4}} &= 1, \qquad
	\crat{\red{1}}{\red{4}}{\red{3}}{\red{2}} = \rightarrow 0.
\end{split}
\end{equation}
\begin{equation}
\begin{split}
	\crat{\blue{1}}{\blue{2}}{\blue{3}}{\blue{4}} \rightarrow 0, \qquad
	\crat{\blue{1}}{\blue{3}}{\blue{4}}{\blue{2}} &= 1, \qquad
	\crat{\blue{1}}{\blue{4}}{\blue{2}}{\blue{3}} = \rightarrow \infty, \\
	\crat{\blue{1}}{\blue{2}}{\blue{4}}{\blue{3}} = \rightarrow \infty, \qquad
	\crat{\blue{1}}{\blue{3}}{\blue{2}}{\blue{4}} &= 1, \qquad
	\crat{\blue{1}}{\blue{4}}{\blue{3}}{\blue{2}} \rightarrow 0.
\end{split}
\end{equation}
The rest of the dual conformal invariants are fixed and nontrivial:
\begin{equation}
\begin{split}
	\crat{\green{1}}{\green{2}}{\green{3}}{\green{4}} = \frac{s u}{m_{1}^{4}}, \qquad
	\crat{\green{1}}{\green{3}}{\green{4}}{\green{2}} &= \frac{m_{1}^{4}}{m_{2}^{4}}, \qquad
	\crat{\green{1}}{\green{4}}{\green{2}}{\green{3}} = \frac{m_{2}^{4}}{s u}, \\
	\crat{\green{1}}{\green{2}}{\green{4}}{\green{3}} = \frac{m_{1}^{4}}{s u}, \qquad
	\crat{\green{1}}{\green{3}}{\green{2}}{\green{4}} &= \frac{m_{2}^{4}}{m_{1}^{4}}, \qquad
	\crat{\green{1}}{\green{4}}{\green{3}}{\green{2}} = \frac{s u}{m_{2}^{4}}.
\end{split}
\end{equation}
%%%%%%%%%%%%%%%%%%%%%%%%%%%%%%%%%%%%%%%%%%%%%%%%%%%%%%%%%%%%%%%%%%%%%%%%%%%%%%%%%%%%%%%%%%%%%%%%%%%%%%%%%%%%%%%%%%%%
\subsubsection{Backward Scattering}
%%%%%%%%%%%%%%%%%%%%%%%%%%%%%%%%%%%%%%%%%%%%%%%%%%%%%%%%%%%%%%%%%%%%%%%%%%%%%%%%%%%%%%%%%%%%%%%%%%%%%%%%%%%%%%%%%%%%
Backward scattering is the regime of fixed-speed and large (physical) scattering angles (i.e. $z_{13} \rightarrow -1$).
%%%%%%%%%%%%%%%%%%%%%%%%%%%%%%%%%%%%%%%%%%%%%%%%%%%%%%%%%%%%%%%%%%%%%%%%%%%%%%%%%%%%%%%%%%%%%%%%%%%%%%%%%%%%%%%%%%%%
\subsubsection{Fixed-Angle Scattering}
%%%%%%%%%%%%%%%%%%%%%%%%%%%%%%%%%%%%%%%%%%%%%%%%%%%%%%%%%%%%%%%%%%%%%%%%%%%%%%%%%%%%%%%%%%%%%%%%%%%%%%%%%%%%%%%%%%%%
Fixed-angle scattering is the regime of large-speed and (physical) fixed-angle. This can be stated as
\begin{equation}
	\frac{t}{s} \text{ fixed}, \qquad \frac{s}{m_{1} m_{2}} \rightarrow \infty, \qquad \frac{m_{1}}{m_{2}} \text{ fixed}.
\end{equation}
As a corollary, you have
\begin{equation}
	u = 2m_{1}^{2} + 2m_{2}^{2} - s - t \quad \Longrightarrow \quad \frac{u}{s} \text{ fixed}.
\end{equation}
In this regime all dual conformal invariants become trivial:
\begin{equation}
\begin{split}
	\crat{\red{1}}{\red{2}}{\red{3}}{\red{4}} \rightarrow -\infty, \qquad
	\crat{\red{1}}{\red{3}}{\red{4}}{\red{2}} &= 1, \qquad
	\crat{\red{1}}{\red{4}}{\red{2}}{\red{3}} \rightarrow 0, \\
	\crat{\red{1}}{\red{2}}{\red{4}}{\red{3}} \rightarrow 0, \qquad
	\crat{\red{1}}{\red{3}}{\red{2}}{\red{4}} &= 1, \qquad
	\crat{\red{1}}{\red{4}}{\red{3}}{\red{2}} \rightarrow -\infty.
\end{split}
\end{equation}
\begin{equation}
\begin{split}
	\crat{\blue{1}}{\blue{2}}{\blue{3}}{\blue{4}} \rightarrow \infty, \qquad
	\crat{\blue{1}}{\blue{3}}{\blue{4}}{\blue{2}} &= 1, \qquad
	\crat{\blue{1}}{\blue{4}}{\blue{2}}{\blue{3}} \rightarrow 0, \\
	\crat{\blue{1}}{\blue{2}}{\blue{4}}{\blue{3}} \rightarrow 0, \qquad
	\crat{\blue{1}}{\blue{3}}{\blue{2}}{\blue{4}} &= 1, \qquad
	\crat{\blue{1}}{\blue{4}}{\blue{3}}{\blue{2}} \rightarrow \infty.
\end{split}
\end{equation}
\begin{equation}
\begin{split}
	\crat{\green{1}}{\green{2}}{\green{3}}{\green{4}} \rightarrow -\infty, \qquad
	\crat{\green{1}}{\green{3}}{\green{4}}{\green{2}} &= \frac{m_{1}^{4}}{m_{2}^{4}}, \qquad
	\crat{\green{1}}{\green{4}}{\green{2}}{\green{3}} \rightarrow 0, \\
	\crat{\green{1}}{\green{2}}{\green{4}}{\green{3}} \rightarrow 0, \qquad
	\crat{\green{1}}{\green{3}}{\green{2}}{\green{4}} &= \frac{m_{2}^{4}}{m_{1}^{4}}, \qquad
	\crat{\green{1}}{\green{4}}{\green{3}}{\green{2}} \rightarrow -\infty.
\end{split}
\end{equation}
From the point of view of dual conformal invariants, Regge scattering and fixed-angle scattering are very similar.