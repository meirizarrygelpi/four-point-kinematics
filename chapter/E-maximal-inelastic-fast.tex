\chapter{Maximal Inelastic Fast Diversity}
%%%%%%%%%%%%%%%%%%%%%%%%%%%%%%%%%%%%%%%%%%%%%%%%%%%%%%%%%%%%%%%%%%%%%%%%%%%%%%%%%%%%%%%%%%%%%%%%%%%%%%%%%%%%%%%%%%%%
In this chapter I will consider the kinematics of a 2-to-2 scattering process that has maximal inelastic fast diversity. This means that all external quanta are fast, and all of the corresponding masses are distinct:
\begin{equation}
	w_{1}^{2} = \abs{p_{1}}^{2}, \qquad w_{2}^{2} = \abs{p_{2}}^{2}, \qquad w_{3}^{2} = \abs{p_{3}}^{2}, \qquad w_{4}^{2} = \abs{p_{4}}^{2}.
\end{equation}
Thus, this is an inelastic process. For example,
\begin{equation}
	a(p_{1}) + b(p_{2}) \longrightarrow y(p_{3}) + z(p_{4}).
\end{equation}
For this process, (\ref{eq:stu_slow}) becomes
\begin{equation}
	s + t + u = -w_{1}^{2} - w_{2}^{2} - w_{3}^{2} - w_{4}^{2}.
\end{equation}
%%%%%%%%%%%%%%%%%%%%%%%%%%%%%%%%%%%%%%%%%%%%%%%%%%%%%%%%%%%%%%%%%%%%%%%%%%%%%%%%%%%%%%%%%%%%%%%%%%%%%%%%%%%%%%%%%%%%
\section{Simplicial Invariants}
%%%%%%%%%%%%%%%%%%%%%%%%%%%%%%%%%%%%%%%%%%%%%%%%%%%%%%%%%%%%%%%%%%%%%%%%%%%%%%%%%%%%%%%%%%%%%%%%%%%%%%%%%%%%%%%%%%%%
The simplicial invariants for fast quanta are slightly different from the slow quanta invariants. Due to diversity, there are many distinct values.
%%%%%%%%%%%%%%%%%%%%%%%%%%%%%%%%%%%%%%%%%%%%%%%%%%%%%%%%%%%%%%%%%%%%%%%%%%%%%%%%%%%%%%%%%%%%%%%%%%%%%%%%%%%%%%%%%%%%
\subsection{1-Simplex Invariants}
%%%%%%%%%%%%%%%%%%%%%%%%%%%%%%%%%%%%%%%%%%%%%%%%%%%%%%%%%%%%%%%%%%%%%%%%%%%%%%%%%%%%%%%%%%%%%%%%%%%%%%%%%%%%%%%%%%%%
For the \red{red} tetrahedron you have
\begin{equation}
	C_{\red{12}} = -w_{1}^{2}, \quad C_{\red{13}} = s, \quad C_{\red{14}} = -w_{3}^{2}, \quad
	C_{\red{23}} = -w_{2}^{2}, \quad C_{\red{24}} = t, \quad C_{\red{34}} = -w_{4}^{2}.
\end{equation}
Similarly, for the \blue{blue} tetrahedron you have
\begin{equation}
	C_{\blue{12}} = -w_{3}^{2}, \quad C_{\blue{13}} = t, \quad C_{\blue{14}} = -w_{2}^{2}, \quad
	C_{\blue{23}} = -w_{1}^{2}, \quad C_{\blue{24}} = u, \quad C_{\blue{34}} = -w_{4}^{2}.
\end{equation}
Finally, for the \green{green} tetrahedron you have
\begin{equation}
	C_{\green{12}} = -w_{2}^{2}, \quad C_{\green{13}} = u, \quad C_{\green{14}} = -w_{1}^{2}, \quad
	C_{\green{23}} = -w_{3}^{2}, \quad C_{\green{24}} = s, \quad C_{\green{34}} = -w_{4}^{2}.
\end{equation}
%%%%%%%%%%%%%%%%%%%%%%%%%%%%%%%%%%%%%%%%%%%%%%%%%%%%%%%%%%%%%%%%%%%%%%%%%%%%%%%%%%%%%%%%%%%%%%%%%%%%%%%%%%%%%%%%%%%%
\subsection{2-Simplex Invariants}
%%%%%%%%%%%%%%%%%%%%%%%%%%%%%%%%%%%%%%%%%%%%%%%%%%%%%%%%%%%%%%%%%%%%%%%%%%%%%%%%%%%%%%%%%%%%%%%%%%%%%%%%%%%%%%%%%%%%
For the \red{red} tetrahedron you have
\begin{equation}
\begin{split}
	C_{\red{123}} &= \kallenf{s}{w_{1}}{w_{2}} = \Upsilon_{12}(s), \\
	C_{\red{124}} &= \kallenf{t}{w_{1}}{w_{3}} = \Upsilon_{13}(t), \\
	C_{\red{134}} &= \kallenf{s}{w_{3}}{w_{4}} = \Upsilon_{34}(s), \\
	C_{\red{234}} &= \kallenf{t}{w_{2}}{w_{4}} = \Upsilon_{24}(t).
\end{split}
\end{equation}
Similarly, for the \blue{blue} tetrahedron you have
\begin{equation}
\begin{split}
	C_{\blue{123}} &= \kallenf{t}{w_{1}}{w_{3}} = \Upsilon_{13}(t), \\
	C_{\blue{124}} &= \kallenf{u}{w_{2}}{w_{3}} = \Upsilon_{23}(u), \\
	C_{\blue{134}} &= \kallenf{t}{w_{2}}{w_{4}} = \Upsilon_{24}(t), \\
	C_{\blue{234}} &= \kallenf{u}{w_{1}}{w_{4}} = \Upsilon_{14}(u).
\end{split}
\end{equation}
Finally, for the \green{green} tetrahedron you have
\begin{equation}
\begin{split}
	C_{\green{123}} &= \kallenf{u}{w_{2}}{w_{3}} = \Upsilon_{23}(u), \\
	C_{\green{124}} &= \kallenf{s}{w_{1}}{w_{2}} = \Upsilon_{12}(s), \\
	C_{\green{134}} &= \kallenf{u}{w_{1}}{w_{4}} = \Upsilon_{14}(u), \\
	C_{\green{234}} &= \kallenf{s}{w_{3}}{w_{4}} = \Upsilon_{34}(s).
\end{split}
\end{equation}
%%%%%%%%%%%%%%%%%%%%%%%%%%%%%%%%%%%%%%%%%%%%%%%%%%%%%%%%%%%%%%%%%%%%%%%%%%%%%%%%%%%%%%%%%%%%%%%%%%%%%%%%%%%%%%%%%%%%
\subsection{3-Simplex Invariants}
%%%%%%%%%%%%%%%%%%%%%%%%%%%%%%%%%%%%%%%%%%%%%%%%%%%%%%%%%%%%%%%%%%%%%%%%%%%%%%%%%%%%%%%%%%%%%%%%%%%%%%%%%%%%%%%%%%%%
A tetrahedron has only one possible 3-simplex invariant. Indeed,
\begin{equation}
	C_{\red{1234}} = C_{\blue{1234}} = C_{\green{1234}}.
\end{equation}
%%%%%%%%%%%%%%%%%%%%%%%%%%%%%%%%%%%%%%%%%%%%%%%%%%%%%%%%%%%%%%%%%%%%%%%%%%%%%%%%%%%%%%%%%%%%%%%%%%%%%%%%%%%%%%%%%%%%
\section{Dual Conformal Invariants}
%%%%%%%%%%%%%%%%%%%%%%%%%%%%%%%%%%%%%%%%%%%%%%%%%%%%%%%%%%%%%%%%%%%%%%%%%%%%%%%%%%%%%%%%%%%%%%%%%%%%%%%%%%%%%%%%%%%%
For the \red{red} tetrahedron, you have
\begin{equation}
\begin{split}
	\crat{\red{1}}{\red{2}}{\red{3}}{\red{4}} = \frac{s t}{w_{2}^{2} w_{3}^{2}}, \qquad
	\crat{\red{1}}{\red{3}}{\red{4}}{\red{2}} &= \left( \frac{w_{2} w_{3}}{w_{1} w_{4}} \right)^{2}, \qquad
	\crat{\red{1}}{\red{4}}{\red{2}}{\red{3}} = \frac{w_{1}^{2} w_{4}^{2}}{s t}, \\
	\crat{\red{1}}{\red{2}}{\red{4}}{\red{3}} = \frac{w_{2}^{2} w_{3}^{2}}{s t}, \qquad
	\crat{\red{1}}{\red{3}}{\red{2}}{\red{4}} &= \left( \frac{w_{1} w_{4}}{w_{2} w_{3}} \right)^{2}, \qquad
	\crat{\red{1}}{\red{4}}{\red{3}}{\red{2}} = \frac{s t}{w_{1}^{2} w_{4}^{2}}.
\end{split}
\end{equation}
Similarly, for the \blue{blue} tetrahedron, you have
\begin{equation}
\begin{split}
	\crat{\blue{1}}{\blue{2}}{\blue{3}}{\blue{4}} = \frac{t u}{w_{1}^{2} w_{2}^{2}}, \qquad
	\crat{\blue{1}}{\blue{3}}{\blue{4}}{\blue{2}} &= \left( \frac{w_{1} w_{2}}{w_{3} w_{4}} \right)^{2}, \qquad
	\crat{\blue{1}}{\blue{4}}{\blue{2}}{\blue{3}} = \frac{w_{3}^{2} w_{4}^{2}}{t u}, \\
	\crat{\blue{1}}{\blue{2}}{\blue{4}}{\blue{3}} = \frac{w_{1}^{2} w_{2}^{2}}{t u}, \qquad
	\crat{\blue{1}}{\blue{3}}{\blue{2}}{\blue{4}} &= \left( \frac{w_{3} w_{4}}{w_{1} w_{2}} \right)^{2}, \qquad
	\crat{\blue{1}}{\blue{4}}{\blue{3}}{\blue{2}} = \frac{t u}{w_{3}^{2} w_{4}^{2}}.
\end{split}
\end{equation}
Finally, for the \green{green} tetrahedron, you have
\begin{equation}
\begin{split}
	\crat{\green{1}}{\green{2}}{\green{3}}{\green{4}} = \frac{s u}{w_{1}^{2} w_{3}^{2}}, \qquad
	\crat{\green{1}}{\green{3}}{\green{4}}{\green{2}} &= \left( \frac{w_{1} w_{3}}{w_{2} w_{4}} \right)^{2}, \qquad
	\crat{\green{1}}{\green{4}}{\green{2}}{\green{3}} = \frac{w_{2}^{2} w_{4}^{2}}{s u}, \\
	\crat{\green{1}}{\green{2}}{\green{4}}{\green{3}} = \frac{w_{1}^{2} w_{3}^{2}}{s u}, \qquad
	\crat{\green{1}}{\green{3}}{\green{2}}{\green{4}} &= \left( \frac{w_{2} w_{4}}{w_{1} w_{3}} \right)^{2}, \qquad
	\crat{\green{1}}{\green{4}}{\green{3}}{\green{2}} = \frac{s u}{w_{2}^{2} w_{4}^{2}}.
\end{split}
\end{equation}
%%%%%%%%%%%%%%%%%%%%%%%%%%%%%%%%%%%%%%%%%%%%%%%%%%%%%%%%%%%%%%%%%%%%%%%%%%%%%%%%%%%%%%%%%%%%%%%%%%%%%%%%%%%%%%%%%%%%
\section{Center-of-Momentum Frame}
%%%%%%%%%%%%%%%%%%%%%%%%%%%%%%%%%%%%%%%%%%%%%%%%%%%%%%%%%%%%%%%%%%%%%%%%%%%%%%%%%%%%%%%%%%%%%%%%%%%%%%%%%%%%%%%%%%%%
In the center-of-momentum frame you write the energy-momentum vectors as
\begin{equation}
	p_{1} = \begin{pmatrix} E_{1} & \mathbf{p}_{1} \end{pmatrix}, \qquad p_{2} = \begin{pmatrix} E_{2} & -\mathbf{p}_{1} \end{pmatrix}, \qquad p_{3} = \begin{pmatrix} E_{3} & \mathbf{p}_{3} \end{pmatrix}, \qquad p_{4} = \begin{pmatrix} E_{4} & -\mathbf{p}_{3} \end{pmatrix}.
\end{equation}
Using the on-shell constraints leads to:
\begin{equation}
	w^{2} = -E^{2} + \abs{\mathbf{p}}^{2}, \quad \Longrightarrow \quad \abs{\mathbf{p}} = \sqrt{w^{2} + E^{2}}.
\end{equation}
Thus,
\begin{equation}
	\abs{\mathbf{p}_{1}} = \sqrt{w_{1}^{2} + E_{1}^{2}} = \sqrt{w_{2}^{2} + E_{2}^{2}}, \qquad \abs{\mathbf{p}_{3}} = \sqrt{w_{3}^{2} + E_{3}^{2}} = \sqrt{w_{4}^{2} + E_{4}^{2}}.
\end{equation}
One of the first things to notice is that
\begin{equation}
	s = (E_{1} + E_{2})^{2} = (E_{3} + E_{4})^{2}.
\end{equation}
Thus, $s$ can be interpreted as the total energy in the center-of-momentum frame. It also follows that $s$ must be positive. Using
\begin{equation}
	E_{2} = \sqrt{s} - E_{1}, \qquad E_{4} = \sqrt{s} - E_{3},
\end{equation}
leads to
\begin{equation}
	E_{1} = \frac{s - w_{1}^{2} + w_{2}^{2}}{2 \sqrt{s}} , \qquad E_{3} = \frac{s - w_{3}^{2} + w_{4}^{2}}{2 \sqrt{s}}.
\end{equation}
Hence
\begin{equation}
	E_{2} = \frac{s + w_{1}^{2} - w_{2}^{2}}{2 \sqrt{s}} , \qquad E_{4} = \frac{s + w_{3}^{2} - w_{4}^{2}}{2 \sqrt{s}}.
\end{equation}
Going back to the spatial momentum, you find
\begin{equation}
	\abs{\mathbf{p}_{1}} = \frac{\sqrt{\Upsilon_{12}(s)}}{2 \sqrt{s}}, \qquad
	\abs{\mathbf{p}_{3}} = \frac{\sqrt{\Upsilon_{34}(s)}}{2 \sqrt{s}}.
\end{equation}
%%%%%%%%%%%%%%%%%%%%%%%%%%%%%%%%%%%%%%%%%%%%%%%%%%%%%%%%%%%%%%%%%%%%%%%%%%%%%%%%%%%%%%%%%%%%%%%%%%%%%%%%%%%%%%%%%%%%
\subsection{Speed and Rapidity}
%%%%%%%%%%%%%%%%%%%%%%%%%%%%%%%%%%%%%%%%%%%%%%%%%%%%%%%%%%%%%%%%%%%%%%%%%%%%%%%%%%%%%%%%%%%%%%%%%%%%%%%%%%%%%%%%%%%%
The speed of each external quantum is
\begin{align}
	\abs{\mathbf{v}_{1}} &= \frac{\sqrt{\Upsilon_{12}(s)}}{s - w_{1}^{2} + w_{2}^{2}}, \\
	\abs{\mathbf{v}_{2}} &= \frac{\sqrt{\Upsilon_{12}(s)}}{s + w_{1}^{2} - w_{2}^{2}}, \\
	\abs{\mathbf{v}_{3}} &= \frac{\sqrt{\Upsilon_{34}(s)}}{s - w_{3}^{2} + w_{4}^{2}}, \\
	\abs{\mathbf{v}_{4}} &= \frac{\sqrt{\Upsilon_{34}(s)}}{s + w_{3}^{2} - w_{4}^{2}}.
\end{align}
The (fast) rapidity of each external quantum is
\begin{align}
	\xi_{1} &= \frac{1}{2} \log{\left[ \frac{\sqrt{\Upsilon_{12}(s)} + s - w_{1}^{2} + w_{2}^{2}}{\sqrt{\Upsilon_{12}(s)} - s + w_{1}^{2} - w_{2}^{2}} \right]}, \\
	\xi_{2} &= \frac{1}{2} \log{\left[ \frac{\sqrt{\Upsilon_{12}(s)} + s + w_{1}^{2} - w_{2}^{2}}{\sqrt{\Upsilon_{12}(s)} - s - w_{1}^{2} + w_{2}^{2}} \right]}, \\
	\xi_{3} &= \frac{1}{2} \log{\left[ \frac{\sqrt{\Upsilon_{34}(s)} + s - w_{3}^{2} + w_{4}^{2}}{\sqrt{\Upsilon_{34}(s)} - s + w_{3}^{2} - w_{4}^{2}} \right]}, \\
	\xi_{4} &= \frac{1}{2} \log{\left[ \frac{\sqrt{\Upsilon_{34}(s)} + s + w_{3}^{2} - w_{4}^{2}}{\sqrt{\Upsilon_{34}(s)} - s - w_{3}^{2} + w_{4}^{2}} \right]}.
\end{align}
The sum of the incoming (fast) rapidities is
\begin{equation}
	\xi_{1} + \xi_{2} = \frac{1}{2} \log{\left[ \frac{s + w_{1}^{2} + w_{2}^{2} + \sqrt{\Upsilon_{12}(s)}}{s + w_{1}^{2} + w_{2}^{2} - \sqrt{\Upsilon_{12}(s)}} \right]}.
\end{equation}
Similarly, the sum of the outgoing (fast) rapidities is
\begin{equation}
	\xi_{3} + \xi_{4} = \frac{1}{2} \log{\left[ \frac{s + w_{3}^{2} + w_{4}^{2} + \sqrt{\Upsilon_{34}(s)}}{s + w_{3}^{2} + w_{4}^{2} - \sqrt{\Upsilon_{34}(s)}} \right]}.
\end{equation}
%%%%%%%%%%%%%%%%%%%%%%%%%%%%%%%%%%%%%%%%%%%%%%%%%%%%%%%%%%%%%%%%%%%%%%%%%%%%%%%%%%%%%%%%%%%%%%%%%%%%%%%%%%%%%%%%%%%%
\subsection{Physical Scattering Region}
%%%%%%%%%%%%%%%%%%%%%%%%%%%%%%%%%%%%%%%%%%%%%%%%%%%%%%%%%%%%%%%%%%%%%%%%%%%%%%%%%%%%%%%%%%%%%%%%%%%%%%%%%%%%%%%%%%%%
The on-shell relation for a fast quantum is
\begin{equation}
	w^{2} = -E^{2} + \abs{\mathbf{p}}^{2} \quad \Longrightarrow \quad \abs{\mathbf{p}}^{2} = w^{2} + E^{2},
\end{equation}
so requiring $E > 0$ leads to $\abs{\mathbf{p}} > w$. The energies are positive as long as
\begin{equation}
	s > \abs{w_{1} - w_{2}} \left( w_{1} + w_{2} \right), \qquad s > \abs{w_{3} - w_{4}} \left( w_{3} + w_{4} \right).
\end{equation}
%%%%%%%%%%%%%%%%%%%%%%%%%%%%%%%%%%%%%%%%%%%%%%%%%%%%%%%%%%%%%%%%%%%%%%%%%%%%%%%%%%%%%%%%%%%%%%%%%%%%%%%%%%%%%%%%%%%%
\subsection{Scattering Regimes}
%%%%%%%%%%%%%%%%%%%%%%%%%%%%%%%%%%%%%%%%%%%%%%%%%%%%%%%%%%%%%%%%%%%%%%%%%%%%%%%%%%%%%%%%%%%%%%%%%%%%%%%%%%%%%%%%%%%%
...
%%%%%%%%%%%%%%%%%%%%%%%%%%%%%%%%%%%%%%%%%%%%%%%%%%%%%%%%%%%%%%%%%%%%%%%%%%%%%%%%%%%%%%%%%%%%%%%%%%%%%%%%%%%%%%%%%%%%
\subsubsection{Fast Small-Speed Scattering}
%%%%%%%%%%%%%%%%%%%%%%%%%%%%%%%%%%%%%%%%%%%%%%%%%%%%%%%%%%%%%%%%%%%%%%%%%%%%%%%%%%%%%%%%%%%%%%%%%%%%%%%%%%%%%%%%%%%%
Fast small-speed scattering is the regime when
\begin{equation}
	\frac{s}{w_{1} w_{2}} \rightarrow \infty, \qquad \frac{w_{1}}{w_{2}} \text{ fixed}, \qquad \frac{w_{1}}{w_{3}} \text{ fixed}, \qquad \frac{w_{2}}{w_{4}} \text{ fixed}.
\end{equation}
In this regime, all speeds approach unity.
%%%%%%%%%%%%%%%%%%%%%%%%%%%%%%%%%%%%%%%%%%%%%%%%%%%%%%%%%%%%%%%%%%%%%%%%%%%%%%%%%%%%%%%%%%%%%%%%%%%%%%%%%%%%%%%%%%%%
\subsubsection{Fast Fixed-Speed Scattering}
%%%%%%%%%%%%%%%%%%%%%%%%%%%%%%%%%%%%%%%%%%%%%%%%%%%%%%%%%%%%%%%%%%%%%%%%%%%%%%%%%%%%%%%%%%%%%%%%%%%%%%%%%%%%%%%%%%%%
The speed of each external quantum can be written as a function of four dimensionless ratios
\begin{equation}
	\frac{s}{w_{1} w_{2}}, \qquad \frac{w_{1}}{w_{2}}, \qquad \frac{w_{1}}{w_{3}}, \qquad \frac{w_{2}}{w_{4}}.
\end{equation}
Fixed-speed scattering corresponds to the kinematic regime where we keep these ratios fixed:
\begin{equation}
	\frac{s}{w_{1} w_{2}} \text{ fixed}, \qquad \frac{w_{1}}{w_{2}} \text{ fixed}, \qquad \frac{w_{1}}{w_{3}} \text{ fixed}, \qquad \frac{w_{2}}{w_{4}} \text{ fixed}.
\end{equation}
This regime is appropriate for either large $s$ and large masses, or small $s$ and small masses. Note that fixed-speed is equivalent to fixed-rapidity. By itself, this regime is not very helpful, but when combined with other limits it leads to important approximations.
%%%%%%%%%%%%%%%%%%%%%%%%%%%%%%%%%%%%%%%%%%%%%%%%%%%%%%%%%%%%%%%%%%%%%%%%%%%%%%%%%%%%%%%%%%%%%%%%%%%%%%%%%%%%%%%%%%%%
\subsubsection{Fast Large-Speed Scattering}
%%%%%%%%%%%%%%%%%%%%%%%%%%%%%%%%%%%%%%%%%%%%%%%%%%%%%%%%%%%%%%%%%%%%%%%%%%%%%%%%%%%%%%%%%%%%%%%%%%%%%%%%%%%%%%%%%%%%
Fast large speeds occur when
\begin{equation}
	s \rightarrow \pm \left(w_{1} - w_{2}\right)\left(w_{1} + w_{2}\right), \qquad s \rightarrow \pm \left(w_{3} - w_{4}\right)\left(w_{3} + w_{4}\right).
\end{equation}
In each of these four limits one of the speeds becomes infinite.
%%%%%%%%%%%%%%%%%%%%%%%%%%%%%%%%%%%%%%%%%%%%%%%%%%%%%%%%%%%%%%%%%%%%%%%%%%%%%%%%%%%%%%%%%%%%%%%%%%%%%%%%%%%%%%%%%%%%
\subsubsection{Regge Scattering}
%%%%%%%%%%%%%%%%%%%%%%%%%%%%%%%%%%%%%%%%%%%%%%%%%%%%%%%%%%%%%%%%%%%%%%%%%%%%%%%%%%%%%%%%%%%%%%%%%%%%%%%%%%%%%%%%%%%%
Regge scattering is the regime of fixed-speed and large (unphysical) $z_{13}$. That is,
\begin{equation}
	\frac{t}{s} \rightarrow \infty, \qquad \frac{s}{w_{1} w_{2}} \text{ fixed}, \qquad \frac{w_{1}}{w_{2}} \text{ fixed}, \qquad \frac{w_{1}}{w_{3}} \text{ fixed}, \qquad \frac{w_{2}}{w_{4}} \text{ fixed}.
\end{equation}
As a corollary, you have
\begin{equation}
	u = -w_{1}^{2} - w_{2}^{2} - w_{3}^{2} - w_{4}^{2} - s - t \quad \Longrightarrow \quad \frac{u}{s} \rightarrow -\infty.
\end{equation}
In this regime almost all values of the dual conformal invariant become trivial:
\begin{equation}
\begin{split}
	\crat{\red{1}}{\red{2}}{\red{3}}{\red{4}} \rightarrow +\infty, \qquad
	\crat{\red{1}}{\red{3}}{\red{4}}{\red{2}} &= \left( \frac{w_{2} w_{3}}{w_{1} w_{4}} \right)^{2}, \qquad
	\crat{\red{1}}{\red{4}}{\red{2}}{\red{3}} \rightarrow 0, \\
	\crat{\red{1}}{\red{2}}{\red{4}}{\red{3}} \rightarrow 0, \qquad
	\crat{\red{1}}{\red{3}}{\red{2}}{\red{4}} &= \left( \frac{w_{1} w_{4}}{w_{2} w_{3}} \right)^{2}, \qquad
	\crat{\red{1}}{\red{4}}{\red{3}}{\red{2}} \rightarrow +\infty.
\end{split}
\end{equation}
\begin{equation}
\begin{split}
	\crat{\blue{1}}{\blue{2}}{\blue{3}}{\blue{4}} \rightarrow -\infty, \qquad
	\crat{\blue{1}}{\blue{3}}{\blue{4}}{\blue{2}} &= \left( \frac{w_{1} w_{2}}{w_{3} w_{4}} \right)^{2}, \qquad
	\crat{\blue{1}}{\blue{4}}{\blue{2}}{\blue{3}} \rightarrow 0, \\
	\crat{\blue{1}}{\blue{2}}{\blue{4}}{\blue{3}} \rightarrow 0, \qquad
	\crat{\blue{1}}{\blue{3}}{\blue{2}}{\blue{4}} &= \left( \frac{w_{3} w_{4}}{w_{1} w_{2}} \right)^{2}, \qquad
	\crat{\blue{1}}{\blue{4}}{\blue{3}}{\blue{2}} \rightarrow -\infty.
\end{split}
\end{equation}
\begin{equation}
\begin{split}
	\crat{\green{1}}{\green{2}}{\green{3}}{\green{4}} \rightarrow -\infty, \qquad
	\crat{\green{1}}{\green{3}}{\green{4}}{\green{2}} &= \left( \frac{w_{1} w_{3}}{w_{2} w_{4}} \right)^{2}, \qquad
	\crat{\green{1}}{\green{4}}{\green{2}}{\green{3}} \rightarrow 0, \\
	\crat{\green{1}}{\green{2}}{\green{4}}{\green{3}} \rightarrow 0, \qquad
	\crat{\green{1}}{\green{3}}{\green{2}}{\green{4}} &= \left( \frac{w_{2} w_{4}}{w_{1} w_{3}} \right)^{2}, \qquad
	\crat{\green{1}}{\green{4}}{\green{3}}{\green{2}} \rightarrow -\infty.
\end{split}
\end{equation}
%%%%%%%%%%%%%%%%%%%%%%%%%%%%%%%%%%%%%%%%%%%%%%%%%%%%%%%%%%%%%%%%%%%%%%%%%%%%%%%%%%%%%%%%%%%%%%%%%%%%%%%%%%%%%%%%%%%%
\subsubsection{Forward Scattering}
%%%%%%%%%%%%%%%%%%%%%%%%%%%%%%%%%%%%%%%%%%%%%%%%%%%%%%%%%%%%%%%%%%%%%%%%%%%%%%%%%%%%%%%%%%%%%%%%%%%%%%%%%%%%%%%%%%%%
Forward scattering is the regime of fixed-speed and small (physical) scattering angles (i.e. $z_{13} \rightarrow +1$).
%%%%%%%%%%%%%%%%%%%%%%%%%%%%%%%%%%%%%%%%%%%%%%%%%%%%%%%%%%%%%%%%%%%%%%%%%%%%%%%%%%%%%%%%%%%%%%%%%%%%%%%%%%%%%%%%%%%%
\subsubsection{Backward Scattering}
%%%%%%%%%%%%%%%%%%%%%%%%%%%%%%%%%%%%%%%%%%%%%%%%%%%%%%%%%%%%%%%%%%%%%%%%%%%%%%%%%%%%%%%%%%%%%%%%%%%%%%%%%%%%%%%%%%%%
Backward scattering is the regime of fixed-speed and large (physical) scattering angles (i.e. $z_{13} \rightarrow -1$).
%%%%%%%%%%%%%%%%%%%%%%%%%%%%%%%%%%%%%%%%%%%%%%%%%%%%%%%%%%%%%%%%%%%%%%%%%%%%%%%%%%%%%%%%%%%%%%%%%%%%%%%%%%%%%%%%%%%%
\subsubsection{Fixed-Angle Scattering}
%%%%%%%%%%%%%%%%%%%%%%%%%%%%%%%%%%%%%%%%%%%%%%%%%%%%%%%%%%%%%%%%%%%%%%%%%%%%%%%%%%%%%%%%%%%%%%%%%%%%%%%%%%%%%%%%%%%%
Fixed-angle scattering is the regime of (fast) small-speeds and (physical) fixed-angle. This can be stated as
\begin{equation}
	\frac{t}{s} \text{ fixed}, \qquad \frac{s}{w_{1} w_{2}} \rightarrow \infty, \qquad \frac{w_{1}}{w_{2}} \text{ fixed}, \qquad \frac{w_{1}}{w_{3}} \text{ fixed}, \qquad \frac{w_{2}}{w_{4}} \text{ fixed}.
\end{equation}
As a corollary, you have
\begin{equation}
	u = -w_{1}^{2} - w_{2}^{2} - w_{3}^{2} - w_{4}^{2} - s - t \quad \Longrightarrow \quad \frac{u}{s} \text{ fixed}.
\end{equation}
In this regime almost all values of the dual conformal invariant become trivial:
\begin{equation}
\begin{split}
	\crat{\red{1}}{\red{2}}{\red{3}}{\red{4}} \rightarrow \infty, \qquad
	\crat{\red{1}}{\red{3}}{\red{4}}{\red{2}} &= \left( \frac{w_{2} w_{3}}{w_{1} w_{4}} \right)^{2}, \qquad
	\crat{\red{1}}{\red{4}}{\red{2}}{\red{3}} \rightarrow 0, \\
	\crat{\red{1}}{\red{2}}{\red{4}}{\red{3}} \rightarrow 0, \qquad
	\crat{\red{1}}{\red{3}}{\red{2}}{\red{4}} &= \left( \frac{w_{1} w_{4}}{w_{2} w_{3}} \right)^{2}, \qquad
	\crat{\red{1}}{\red{4}}{\red{3}}{\red{2}} \rightarrow \infty;
\end{split}
\end{equation}
\begin{equation}
\begin{split}
	\crat{\blue{1}}{\blue{2}}{\blue{3}}{\blue{4}} \rightarrow \infty, \qquad
	\crat{\blue{1}}{\blue{3}}{\blue{4}}{\blue{2}} &= \left( \frac{w_{1} w_{2}}{w_{3} w_{4}} \right)^{2}, \qquad
	\crat{\blue{1}}{\blue{4}}{\blue{2}}{\blue{3}} \rightarrow 0, \\
	\crat{\blue{1}}{\blue{2}}{\blue{4}}{\blue{3}} \rightarrow 0, \qquad
	\crat{\blue{1}}{\blue{3}}{\blue{2}}{\blue{4}} &= \left( \frac{w_{3} w_{4}}{w_{1} w_{2}} \right)^{2}, \qquad
	\crat{\blue{1}}{\blue{4}}{\blue{3}}{\blue{2}} \rightarrow \infty;
\end{split}
\end{equation}
\begin{equation}
\begin{split}
	\crat{\green{1}}{\green{2}}{\green{3}}{\green{4}} \rightarrow \infty, \qquad
	\crat{\green{1}}{\green{3}}{\green{4}}{\green{2}} &= \left( \frac{w_{1} w_{3}}{w_{2} w_{4}} \right)^{2}, \qquad
	\crat{\green{1}}{\green{4}}{\green{2}}{\green{3}} \rightarrow 0, \\
	\crat{\green{1}}{\green{2}}{\green{4}}{\green{3}} \rightarrow 0, \qquad
	\crat{\green{1}}{\green{3}}{\green{2}}{\green{4}} &= \left( \frac{w_{2} w_{4}}{w_{1} w_{3}} \right)^{2}, \qquad
	\crat{\green{1}}{\green{4}}{\green{3}}{\green{2}} \rightarrow \infty.
\end{split}
\end{equation}
From the point of view of dual conformal invariants, Regge scattering and fixed-angle scattering are very similar.