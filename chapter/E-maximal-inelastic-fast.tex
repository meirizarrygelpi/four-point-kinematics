\chapter{Maximal Inelastic Fast Diversity}
%%%%%%%%%%%%%%%%%%%%%%%%%%%%%%%%%%%%%%%%%%%%%%%%%%%%%%%%%%%%%%%%%%%%%%%%%%%%%%%%%%%%%%%%%%%%%%%%%%%%%%%%%%%%%%%%%%%%
In this chapter I will consider the kinematics of a 2-to-2 scattering process that has maximal inelastic fast diversity. This means that all external quanta are fast, and all of the corresponding masses are distinct:
\begin{equation}
	w_{1}^{2} = \abs{p_{1}}^{2}, \qquad w_{2}^{2} = \abs{p_{2}}^{2}, \qquad w_{3}^{2} = \abs{p_{3}}^{2}, \qquad w_{4}^{2} = \abs{p_{4}}^{2}.
\end{equation}
Thus, this is an inelastic process. For example,
\begin{equation}
	a(p_{1}) + b(p_{2}) \longrightarrow y(p_{3}) + z(p_{4}).
\end{equation}
For this process, (\ref{eq:stu_slow}) becomes
\begin{equation}
	s + t + u = -w_{1}^{2} - w_{2}^{2} - w_{3}^{2} - w_{4}^{2}.
\end{equation}
%%%%%%%%%%%%%%%%%%%%%%%%%%%%%%%%%%%%%%%%%%%%%%%%%%%%%%%%%%%%%%%%%%%%%%%%%%%%%%%%%%%%%%%%%%%%%%%%%%%%%%%%%%%%%%%%%%%%
\section{Simplicial Invariants}
%%%%%%%%%%%%%%%%%%%%%%%%%%%%%%%%%%%%%%%%%%%%%%%%%%%%%%%%%%%%%%%%%%%%%%%%%%%%%%%%%%%%%%%%%%%%%%%%%%%%%%%%%%%%%%%%%%%%
A tetrahedron is a 3-simplex. As such, it contains four vertices (0-simplex), six edges (1-simplex), four triangular faces (2-simplex), and one tetrahedron (3-simplex). The $n$-volume of an $n$-simplex is found by evaluating an $(n+1)$-point Cayley-Menger determinant. 
%%%%%%%%%%%%%%%%%%%%%%%%%%%%%%%%%%%%%%%%%%%%%%%%%%%%%%%%%%%%%%%%%%%%%%%%%%%%%%%%%%%%%%%%%%%%%%%%%%%%%%%%%%%%%%%%%%%%
\subsection{1-Simplex Invariants}
%%%%%%%%%%%%%%%%%%%%%%%%%%%%%%%%%%%%%%%%%%%%%%%%%%%%%%%%%%%%%%%%%%%%%%%%%%%%%%%%%%%%%%%%%%%%%%%%%%%%%%%%%%%%%%%%%%%%
Given two dual spacetime positions, the 2-point Cayley-Menger determinant is given by:
\begin{equation}
	C_{ij} = -\frac{1}{2} \det{
	\begin{pmatrix}
	0 & \abs{d_{ij}}^{2} & 1 \\
	\abs{d_{ij}}^{2} & 0 & 1 \\
	1 & 1 & 0
	\end{pmatrix}
	}, \qquad d_{ij} \equiv d_{i} - d_{j}.
\end{equation}
Since a tetrahedron has six edges, there are six possible 1-simplex invariants. These invariants will be either squared masses or Mandelstam invariants. For the \red{red} tetrahedron you have
\begin{equation}
\begin{split}
	C_{\red{12}} = -w_{1}^{2}, \qquad C_{\red{13}} &= s, \qquad C_{\red{14}} = -w_{3}^{2}, \\
	C_{\red{23}} = -w_{2}^{2}, \qquad C_{\red{24}} &= t, \qquad C_{\red{34}} = -w_{4}^{2}. \\
\end{split}
\end{equation}
Similarly, for the \blue{blue} tetrahedron you have
\begin{equation}
\begin{split}
	C_{\blue{12}} = -w_{3}^{2}, \qquad C_{\blue{13}} &= t, \qquad C_{\blue{14}} = -w_{2}^{2}, \\
	C_{\blue{23}} = -w_{1}^{2}, \qquad C_{\blue{24}} &= u, \qquad C_{\blue{34}} = -w_{4}^{2}. \\
\end{split}
\end{equation}
Finally, for the \green{green} tetrahedron you have
\begin{equation}
\begin{split}
	C_{\green{12}} = -w_{2}^{2}, \qquad C_{\green{13}} &= u, \qquad C_{\green{14}} = -w_{1}^{2}, \\
	C_{\green{23}} = -w_{3}^{2}, \qquad C_{\green{24}} &= s, \qquad C_{\green{34}} = -w_{4}^{2}. \\
\end{split}
\end{equation}
%%%%%%%%%%%%%%%%%%%%%%%%%%%%%%%%%%%%%%%%%%%%%%%%%%%%%%%%%%%%%%%%%%%%%%%%%%%%%%%%%%%%%%%%%%%%%%%%%%%%%%%%%%%%%%%%%%%%
\subsection{2-Simplex Invariants}
%%%%%%%%%%%%%%%%%%%%%%%%%%%%%%%%%%%%%%%%%%%%%%%%%%%%%%%%%%%%%%%%%%%%%%%%%%%%%%%%%%%%%%%%%%%%%%%%%%%%%%%%%%%%%%%%%%%%
Given three dual spacetime positions, the 3-point Cayley-Menger determinant is given by:
\begin{equation}
	S_{ijk} = \det{
	\begin{pmatrix}
	0 & \abs{d_{ij}}^{2} & \abs{d_{ik}}^{2} & 1 \\
	\abs{d_{ij}}^{2} & 0 & \abs{d_{jk}}^{2} & 1 \\
	\abs{d_{ik}}^{2} & \abs{d_{jk}}^{2} & 0 & 1 \\
	1 & 1 & 1 & 0
	\end{pmatrix}
	}, \qquad d_{ij} \equiv d_{i} - d_{j}.
\end{equation}
Since a tetrahedron has four triangular faces, there are four possible 2-simplex invariants. Each of these invariants has the form
\begin{equation}
	\Lambda_{IJ}(x) = \kallen{x}{m_{I}}{m_{J}},
\end{equation}
where $x$ is a Mandelstam invariant. This function is also known as the K\"{a}ll\'{e}n function. For the \red{red} tetrahedron you have
\begin{equation}
\begin{split}
	C_{\red{123}} &= \kallenf{s}{w_{1}}{w_{2}}, \\
	C_{\red{124}} &= \kallenf{t}{w_{1}}{w_{3}}, \\
	C_{\red{134}} &= \kallenf{s}{w_{3}}{w_{4}}, \\
	C_{\red{234}} &= \kallenf{t}{w_{2}}{w_{4}}.
\end{split}
\end{equation}
Similarly, for the \blue{blue} tetrahedron you have
\begin{equation}
\begin{split}
	C_{\blue{123}} &= \kallenf{t}{w_{1}}{w_{3}}, \\
	C_{\blue{124}} &= \kallenf{u}{w_{2}}{w_{3}}, \\
	C_{\blue{134}} &= \kallenf{t}{w_{2}}{w_{4}}, \\
	C_{\blue{234}} &= \kallenf{u}{w_{1}}{w_{4}}.
\end{split}
\end{equation}
Finally, for the \green{green} tetrahedron you have
\begin{equation}
\begin{split}
	C_{\green{123}} &= \kallenf{u}{w_{2}}{w_{3}}, \\
	C_{\green{124}} &= \kallenf{s}{w_{1}}{w_{2}}, \\
	C_{\green{134}} &= \kallenf{u}{w_{1}}{w_{4}}, \\
	C_{\green{234}} &= \kallenf{s}{w_{3}}{w_{4}}.
\end{split}
\end{equation}
%%%%%%%%%%%%%%%%%%%%%%%%%%%%%%%%%%%%%%%%%%%%%%%%%%%%%%%%%%%%%%%%%%%%%%%%%%%%%%%%%%%%%%%%%%%%%%%%%%%%%%%%%%%%%%%%%%%%
\subsection{3-Simplex Invariants}
%%%%%%%%%%%%%%%%%%%%%%%%%%%%%%%%%%%%%%%%%%%%%%%%%%%%%%%%%%%%%%%%%%%%%%%%%%%%%%%%%%%%%%%%%%%%%%%%%%%%%%%%%%%%%%%%%%%%
Given four dual spacetime positions, the 4-point Cayley-Menger determinant is given by:
\begin{equation}
	C_{ijkl} = -\frac{1}{2} \det{
	\begin{pmatrix}
	0 & \abs{d_{ij}}^{2} & \abs{d_{ik}}^{2} & \abs{d_{il}}^{2} & 1 \\
	\abs{d_{ij}}^{2} & 0 & \abs{d_{jk}}^{2} & \abs{d_{jl}}^{2} & 1 \\
	\abs{d_{ik}}^{2} & \abs{d_{jk}}^{2} & 0 & \abs{d_{kl}}^{2} & 1 \\
	\abs{d_{il}}^{2} & \abs{d_{jl}}^{2} & \abs{d_{kl}}^{2} & 0 & 1 \\
	1 & 1 & 1 & 1 & 0
	\end{pmatrix}
	}, \qquad d_{ij} \equiv d_{i} - d_{j} = .
\end{equation}
A tetrahedron has only one possible 3-simplex invariant. Indeed,
\begin{equation}
	C_{\red{1234}} = C_{\blue{1234}} = C_{\green{1234}} = ....
\end{equation}
%%%%%%%%%%%%%%%%%%%%%%%%%%%%%%%%%%%%%%%%%%%%%%%%%%%%%%%%%%%%%%%%%%%%%%%%%%%%%%%%%%%%%%%%%%%%%%%%%%%%%%%%%%%%%%%%%%%%
\section{Dual Conformal Invariants}
%%%%%%%%%%%%%%%%%%%%%%%%%%%%%%%%%%%%%%%%%%%%%%%%%%%%%%%%%%%%%%%%%%%%%%%%%%%%%%%%%%%%%%%%%%%%%%%%%%%%%%%%%%%%%%%%%%%%
For the \red{red} tetrahedron, you have
\begin{equation}
\begin{split}
	\crat{\red{1}}{\red{2}}{\red{3}}{\red{4}} = \frac{s t}{w_{2}^{2} w_{3}^{2}}, \qquad
	\crat{\red{1}}{\red{3}}{\red{4}}{\red{2}} &= \frac{w_{2}^{2} w_{3}^{2}}{w_{1}^{2} w_{4}^{2}}, \qquad
	\crat{\red{1}}{\red{4}}{\red{2}}{\red{3}} = \frac{w_{1}^{2} w_{4}^{2}}{s t}, \\
	\crat{\red{1}}{\red{2}}{\red{4}}{\red{3}} = \frac{w_{2}^{2} w_{3}^{2}}{s t}, \qquad
	\crat{\red{1}}{\red{3}}{\red{2}}{\red{4}} &= \frac{w_{1}^{2} w_{4}^{2}}{w_{2}^{2} w_{3}^{2}}, \qquad
	\crat{\red{1}}{\red{4}}{\red{3}}{\red{2}} = \frac{s t}{w_{1}^{2} w_{4}^{2}}.
\end{split}
\end{equation}
Similarly, for the \blue{blue} tetrahedron, you have
\begin{equation}
\begin{split}
	\crat{\blue{1}}{\blue{2}}{\blue{3}}{\blue{4}} = \frac{t u}{w_{1}^{2} w_{2}^{2}}, \qquad
	\crat{\blue{1}}{\blue{3}}{\blue{4}}{\blue{2}} &= \frac{w_{1}^{2} w_{2}^{2}}{w_{3}^{2} w_{4}^{2}}, \qquad
	\crat{\blue{1}}{\blue{4}}{\blue{2}}{\blue{3}} = \frac{w_{3}^{2} w_{4}^{2}}{t u}, \\
	\crat{\blue{1}}{\blue{2}}{\blue{4}}{\blue{3}} = \frac{w_{1}^{2} w_{2}^{2}}{t u}, \qquad
	\crat{\blue{1}}{\blue{3}}{\blue{2}}{\blue{4}} &= \frac{w_{3}^{2} w_{4}^{2}}{w_{1}^{2} w_{2}^{2}}, \qquad
	\crat{\blue{1}}{\blue{4}}{\blue{3}}{\blue{2}} = \frac{t u}{w_{3}^{2} w_{4}^{2}}.
\end{split}
\end{equation}
Finally, for the \green{green} tetrahedron, you have
\begin{equation}
\begin{split}
	\crat{\green{1}}{\green{2}}{\green{3}}{\green{4}} = \frac{s u}{w_{1}^{2} w_{3}^{2}}, \qquad
	\crat{\green{1}}{\green{3}}{\green{4}}{\green{2}} &= \frac{w_{1}^{2} w_{3}^{2}}{w_{2}^{2} w_{4}^{2}}, \qquad
	\crat{\green{1}}{\green{4}}{\green{2}}{\green{3}} = \frac{w_{2}^{2} w_{4}^{2}}{s u}, \\
	\crat{\green{1}}{\green{2}}{\green{4}}{\green{3}} = \frac{w_{1}^{2} w_{3}^{2}}{s u}, \qquad
	\crat{\green{1}}{\green{3}}{\green{2}}{\green{4}} &= \frac{w_{2}^{2} w_{4}^{2}}{w_{1}^{2} w_{3}^{2}}, \qquad
	\crat{\green{1}}{\green{4}}{\green{3}}{\green{2}} = \frac{s u}{w_{2}^{2} w_{4}^{2}}.
\end{split}
\end{equation}
%%%%%%%%%%%%%%%%%%%%%%%%%%%%%%%%%%%%%%%%%%%%%%%%%%%%%%%%%%%%%%%%%%%%%%%%%%%%%%%%%%%%%%%%%%%%%%%%%%%%%%%%%%%%%%%%%%%%
\section{Center-of-Momentum Frame}
%%%%%%%%%%%%%%%%%%%%%%%%%%%%%%%%%%%%%%%%%%%%%%%%%%%%%%%%%%%%%%%%%%%%%%%%%%%%%%%%%%%%%%%%%%%%%%%%%%%%%%%%%%%%%%%%%%%%
In the center-of-momentum frame you write the energy-momentum vectors as
\begin{equation}
	p_{1} = \begin{pmatrix} E_{1} & \mathbf{p}_{1} \end{pmatrix}, \qquad p_{2} = \begin{pmatrix} E_{2} & -\mathbf{p}_{1} \end{pmatrix}, \qquad p_{3} = \begin{pmatrix} E_{3} & \mathbf{p}_{3} \end{pmatrix}, \qquad p_{4} = \begin{pmatrix} E_{4} & -\mathbf{p}_{3} \end{pmatrix}.
\end{equation}
Using the on-shell constraints leads to:
\begin{equation}
	w^{2} = -E^{2} + \abs{\mathbf{p}}^{2}, \quad \Longrightarrow \quad \abs{\mathbf{p}} = \sqrt{w^{2} + E^{2}}.
\end{equation}
Thus,
\begin{equation}
	\abs{\mathbf{p}_{1}} = \sqrt{w_{1}^{2} + E_{1}^{2}} = \sqrt{w_{2}^{2} + E_{2}^{2}}, \qquad \abs{\mathbf{p}_{3}} = \sqrt{w_{3}^{2} + E_{3}^{2}} = \sqrt{w_{4}^{2} + E_{4}^{2}}.
\end{equation}
One of the first things to notice is that
\begin{equation}
	s = (E_{1} + E_{2})^{2} = (E_{3} + E_{4})^{2}.
\end{equation}
Thus, $s$ can be interpreted as the total energy in the center-of-momentum frame. It also follows that $s$ must be positive. Using
\begin{equation}
	E_{2} = \sqrt{s} - E_{1}, \qquad E_{4} = \sqrt{s} - E_{3},
\end{equation}
leads to
\begin{equation}
	E_{1} = \frac{s - w_{1}^{2} + w_{2}^{2}}{2 \sqrt{s}} , \qquad E_{3} = \frac{s - w_{3}^{2} + w_{4}^{2}}{2 \sqrt{s}}.
\end{equation}
Hence
\begin{equation}
	E_{2} = \frac{s + w_{1}^{2} - w_{2}^{2}}{2 \sqrt{s}} , \qquad E_{4} = \frac{s + w_{3}^{2} - w_{4}^{2}}{2 \sqrt{s}}.
\end{equation}
Going back the the spatial momentum, you find
\begin{align}
	\abs{\mathbf{p}_{1}} &= \frac{\sqrt{\kallenf{s}{w_{1}}{w_{2}}}}{2 \sqrt{s}}, \\
	\abs{\mathbf{p}_{3}} &= \frac{\sqrt{\kallenf{s}{w_{3}}{w_{4}}}}{2 \sqrt{s}}.
\end{align}
%%%%%%%%%%%%%%%%%%%%%%%%%%%%%%%%%%%%%%%%%%%%%%%%%%%%%%%%%%%%%%%%%%%%%%%%%%%%%%%%%%%%%%%%%%%%%%%%%%%%%%%%%%%%%%%%%%%%
\subsection{Speed and Rapidity}
%%%%%%%%%%%%%%%%%%%%%%%%%%%%%%%%%%%%%%%%%%%%%%%%%%%%%%%%%%%%%%%%%%%%%%%%%%%%%%%%%%%%%%%%%%%%%%%%%%%%%%%%%%%%%%%%%%%%
The speed of each external quantum is
\begin{align}
	\abs{\mathbf{v}_{1}} &= \frac{\sqrt{\kallenf{s}{w_{1}}{w_{2}}}}{s - w_{1}^{2} + w_{2}^{2}}, \\
	\abs{\mathbf{v}_{2}} &= \frac{\sqrt{\kallenf{s}{w_{1}}{w_{2}}}}{s + w_{1}^{2} - w_{2}^{2}}, \\
	\abs{\mathbf{v}_{3}} &= \frac{\sqrt{\kallenf{s}{w_{3}}{w_{4}}}}{s - w_{3}^{2} + w_{4}^{2}}, \\
	\abs{\mathbf{v}_{4}} &= \frac{\sqrt{\kallenf{s}{w_{3}}{w_{4}}}}{s + w_{3}^{2} - w_{4}^{2}}.
\end{align}
The rapidity of each external quantum is
%%%%%%%%%%%%%%%%%%%%%%%%%%%%%%%%%%%%%%%%%%%%%%%%%%%%%%%%%%%%%%%%%%%%%%%%%%%%%%%%%%%%%%%%%%%%%%%%%%%%%%%%%%%%%%%%%%%%
\subsection{Physical Scattering Region}
%%%%%%%%%%%%%%%%%%%%%%%%%%%%%%%%%%%%%%%%%%%%%%%%%%%%%%%%%%%%%%%%%%%%%%%%%%%%%%%%%%%%%%%%%%%%%%%%%%%%%%%%%%%%%%%%%%%%
...
%%%%%%%%%%%%%%%%%%%%%%%%%%%%%%%%%%%%%%%%%%%%%%%%%%%%%%%%%%%%%%%%%%%%%%%%%%%%%%%%%%%%%%%%%%%%%%%%%%%%%%%%%%%%%%%%%%%%
\subsection{Scattering Regimes}
%%%%%%%%%%%%%%%%%%%%%%%%%%%%%%%%%%%%%%%%%%%%%%%%%%%%%%%%%%%%%%%%%%%%%%%%%%%%%%%%%%%%%%%%%%%%%%%%%%%%%%%%%%%%%%%%%%%%
...
%%%%%%%%%%%%%%%%%%%%%%%%%%%%%%%%%%%%%%%%%%%%%%%%%%%%%%%%%%%%%%%%%%%%%%%%%%%%%%%%%%%%%%%%%%%%%%%%%%%%%%%%%%%%%%%%%%%%
\subsubsection{Fast Small-Speed Scattering}
%%%%%%%%%%%%%%%%%%%%%%%%%%%%%%%%%%%%%%%%%%%%%%%%%%%%%%%%%%%%%%%%%%%%%%%%%%%%%%%%%%%%%%%%%%%%%%%%%%%%%%%%%%%%%%%%%%%%
Fast small-speed scattering is the regime when
\begin{equation}
	\frac{s}{w_{1} w_{2}} \rightarrow \infty, \qquad \frac{w_{1}}{w_{2}} \text{ fixed}, \qquad \frac{w_{1}}{w_{3}} \text{ fixed}, \qquad \frac{w_{1}}{w_{4}} \text{ fixed}.
\end{equation}
In this regime, all speeds approach unity.
%%%%%%%%%%%%%%%%%%%%%%%%%%%%%%%%%%%%%%%%%%%%%%%%%%%%%%%%%%%%%%%%%%%%%%%%%%%%%%%%%%%%%%%%%%%%%%%%%%%%%%%%%%%%%%%%%%%%
\subsubsection{Fast Fixed-Speed Scattering}
%%%%%%%%%%%%%%%%%%%%%%%%%%%%%%%%%%%%%%%%%%%%%%%%%%%%%%%%%%%%%%%%%%%%%%%%%%%%%%%%%%%%%%%%%%%%%%%%%%%%%%%%%%%%%%%%%%%%
The speed of each external quantum can be written as a function of four dimensionless ratios
\begin{equation}
	\frac{s}{w_{1} w_{2}}, \qquad \frac{w_{1}}{w_{2}}, \qquad \frac{w_{1}}{w_{3}}, \qquad \frac{w_{1}}{w_{4}}.
\end{equation}
Fixed-speed scattering corresponds to the kinematic regime where we keep these ratios fixed:
\begin{equation}
	\frac{s}{w_{1} w_{2}} \text{ fixed}, \qquad \frac{w_{1}}{w_{2}} \text{ fixed}, \qquad \frac{w_{1}}{w_{3}} \text{ fixed}, \qquad \frac{w_{1}}{w_{4}} \text{ fixed}.
\end{equation}
This regime is appropriate for either large $s$ and large masses, or small $s$ and small masses. Note that fixed-speed is equivalent to fixed-rapidity. By itself, this regime is not very helpful, but when combined with other limits it leads to important approximations.
%%%%%%%%%%%%%%%%%%%%%%%%%%%%%%%%%%%%%%%%%%%%%%%%%%%%%%%%%%%%%%%%%%%%%%%%%%%%%%%%%%%%%%%%%%%%%%%%%%%%%%%%%%%%%%%%%%%%
\subsubsection{Fast Large-Speed Scattering}
%%%%%%%%%%%%%%%%%%%%%%%%%%%%%%%%%%%%%%%%%%%%%%%%%%%%%%%%%%%%%%%%%%%%%%%%%%%%%%%%%%%%%%%%%%%%%%%%%%%%%%%%%%%%%%%%%%%%
Fast large speeds occur when
\begin{equation}
	s \rightarrow \pm \left(w_{1} - w_{2}\right)\left(w_{1} + w_{2}\right), \qquad s \rightarrow \pm \left(w_{3} - w_{4}\right)\left(w_{3} + w_{4}\right).
\end{equation}
In each of these four limits one of the speeds becomes infinite.
%%%%%%%%%%%%%%%%%%%%%%%%%%%%%%%%%%%%%%%%%%%%%%%%%%%%%%%%%%%%%%%%%%%%%%%%%%%%%%%%%%%%%%%%%%%%%%%%%%%%%%%%%%%%%%%%%%%%
\subsubsection{Regge Scattering}
%%%%%%%%%%%%%%%%%%%%%%%%%%%%%%%%%%%%%%%%%%%%%%%%%%%%%%%%%%%%%%%%%%%%%%%%%%%%%%%%%%%%%%%%%%%%%%%%%%%%%%%%%%%%%%%%%%%%
Regge scattering is the regime of fixed-speed and large (unphysical) $z_{13}$. This corresponds to
\begin{equation}
	\frac{t}{s} \rightarrow \infty, \qquad \frac{s}{m_{1} m_{2}} \text{ fixed}, \qquad \frac{m_{1}}{m_{2}} \text{ fixed}.
\end{equation}
As a corollary, you have
\begin{equation}
	u = 2m_{1}^{2} + 2m_{2}^{2} - s - t \quad \Longrightarrow \quad \frac{u}{s} \rightarrow -\infty.
\end{equation}
In this regime all dual conformal invariants become trivial:
\begin{equation}
\begin{split}
	\crat{\red{1}}{\red{2}}{\red{3}}{\red{4}} \rightarrow \infty, \qquad
	\crat{\red{1}}{\red{3}}{\red{4}}{\red{2}} &= 1, \qquad
	\crat{\red{1}}{\red{4}}{\red{2}}{\red{3}} \rightarrow 0, \\
	\crat{\red{1}}{\red{2}}{\red{4}}{\red{3}} \rightarrow 0, \qquad
	\crat{\red{1}}{\red{3}}{\red{2}}{\red{4}} &= 1, \qquad
	\crat{\red{1}}{\red{4}}{\red{3}}{\red{2}} \rightarrow \infty.
\end{split}
\end{equation}
\begin{equation}
\begin{split}
	\crat{\blue{1}}{\blue{2}}{\blue{3}}{\blue{4}} \rightarrow -\infty, \qquad
	\crat{\blue{1}}{\blue{3}}{\blue{4}}{\blue{2}} &= 1, \qquad
	\crat{\blue{1}}{\blue{4}}{\blue{2}}{\blue{3}} \rightarrow 0, \\
	\crat{\blue{1}}{\blue{2}}{\blue{4}}{\blue{3}} \rightarrow 0, \qquad
	\crat{\blue{1}}{\blue{3}}{\blue{2}}{\blue{4}} &= 1, \qquad
	\crat{\blue{1}}{\blue{4}}{\blue{3}}{\blue{2}} \rightarrow -\infty.
\end{split}
\end{equation}
\begin{equation}
\begin{split}
	\crat{\green{1}}{\green{2}}{\green{3}}{\green{4}} \rightarrow -\infty, \qquad
	\crat{\green{1}}{\green{3}}{\green{4}}{\green{2}} &= \frac{m_{1}^{4}}{m_{2}^{4}}, \qquad
	\crat{\green{1}}{\green{4}}{\green{2}}{\green{3}} \rightarrow 0, \\
	\crat{\green{1}}{\green{2}}{\green{4}}{\green{3}} \rightarrow 0, \qquad
	\crat{\green{1}}{\green{3}}{\green{2}}{\green{4}} &= \frac{m_{2}^{4}}{m_{1}^{4}}, \qquad
	\crat{\green{1}}{\green{4}}{\green{3}}{\green{2}} \rightarrow -\infty.
\end{split}
\end{equation}
%%%%%%%%%%%%%%%%%%%%%%%%%%%%%%%%%%%%%%%%%%%%%%%%%%%%%%%%%%%%%%%%%%%%%%%%%%%%%%%%%%%%%%%%%%%%%%%%%%%%%%%%%%%%%%%%%%%%
\subsubsection{Forward Scattering}
%%%%%%%%%%%%%%%%%%%%%%%%%%%%%%%%%%%%%%%%%%%%%%%%%%%%%%%%%%%%%%%%%%%%%%%%%%%%%%%%%%%%%%%%%%%%%%%%%%%%%%%%%%%%%%%%%%%%
Forward scattering is the regime of fixed-speed and small (physical) scattering angles (i.e. $z_{13} \rightarrow 1$). This can be stated as
\begin{equation}
	\frac{t}{s} \rightarrow 0, \qquad \frac{s}{m_{1} m_{2}} \text{ fixed}, \qquad \frac{m_{1}}{m_{2}} \text{ fixed}.
\end{equation}
As a corollary, you have
\begin{equation}
	u = 2m_{1}^{2} + 2m_{2}^{2} - s - t \quad \Longrightarrow \quad \frac{u}{s} \text{ fixed}.
\end{equation}
Unlike the case of Regge scattering, in the forward regime only some of the dual conformal invariants are trivial:
\begin{equation}
\begin{split}
	\crat{\red{1}}{\red{2}}{\red{3}}{\red{4}} \rightarrow 0, \qquad
	\crat{\red{1}}{\red{3}}{\red{4}}{\red{2}} &= 1, \qquad
	\crat{\red{1}}{\red{4}}{\red{2}}{\red{3}} \rightarrow \infty, \\
	\crat{\red{1}}{\red{2}}{\red{4}}{\red{3}} \rightarrow \infty, \qquad
	\crat{\red{1}}{\red{3}}{\red{2}}{\red{4}} &= 1, \qquad
	\crat{\red{1}}{\red{4}}{\red{3}}{\red{2}} = \rightarrow 0.
\end{split}
\end{equation}
\begin{equation}
\begin{split}
	\crat{\blue{1}}{\blue{2}}{\blue{3}}{\blue{4}} \rightarrow 0, \qquad
	\crat{\blue{1}}{\blue{3}}{\blue{4}}{\blue{2}} &= 1, \qquad
	\crat{\blue{1}}{\blue{4}}{\blue{2}}{\blue{3}} = \rightarrow \infty, \\
	\crat{\blue{1}}{\blue{2}}{\blue{4}}{\blue{3}} = \rightarrow \infty, \qquad
	\crat{\blue{1}}{\blue{3}}{\blue{2}}{\blue{4}} &= 1, \qquad
	\crat{\blue{1}}{\blue{4}}{\blue{3}}{\blue{2}} \rightarrow 0.
\end{split}
\end{equation}
The rest of the dual conformal invariants are fixed and nontrivial:
\begin{equation}
\begin{split}
	\crat{\green{1}}{\green{2}}{\green{3}}{\green{4}} = \frac{s u}{m_{1}^{4}}, \qquad
	\crat{\green{1}}{\green{3}}{\green{4}}{\green{2}} &= \frac{m_{1}^{4}}{m_{2}^{4}}, \qquad
	\crat{\green{1}}{\green{4}}{\green{2}}{\green{3}} = \frac{m_{2}^{4}}{s u}, \\
	\crat{\green{1}}{\green{2}}{\green{4}}{\green{3}} = \frac{m_{1}^{4}}{s u}, \qquad
	\crat{\green{1}}{\green{3}}{\green{2}}{\green{4}} &= \frac{m_{2}^{4}}{m_{1}^{4}}, \qquad
	\crat{\green{1}}{\green{4}}{\green{3}}{\green{2}} = \frac{s u}{m_{2}^{4}}.
\end{split}
\end{equation}
%%%%%%%%%%%%%%%%%%%%%%%%%%%%%%%%%%%%%%%%%%%%%%%%%%%%%%%%%%%%%%%%%%%%%%%%%%%%%%%%%%%%%%%%%%%%%%%%%%%%%%%%%%%%%%%%%%%%
\subsubsection{Backward Scattering}
%%%%%%%%%%%%%%%%%%%%%%%%%%%%%%%%%%%%%%%%%%%%%%%%%%%%%%%%%%%%%%%%%%%%%%%%%%%%%%%%%%%%%%%%%%%%%%%%%%%%%%%%%%%%%%%%%%%%
Backward scattering is the regime of fixed-speed and large (physical) scattering angles (i.e. $z_{13} \rightarrow -1$).
%%%%%%%%%%%%%%%%%%%%%%%%%%%%%%%%%%%%%%%%%%%%%%%%%%%%%%%%%%%%%%%%%%%%%%%%%%%%%%%%%%%%%%%%%%%%%%%%%%%%%%%%%%%%%%%%%%%%
\subsubsection{Fixed-Angle Scattering}
%%%%%%%%%%%%%%%%%%%%%%%%%%%%%%%%%%%%%%%%%%%%%%%%%%%%%%%%%%%%%%%%%%%%%%%%%%%%%%%%%%%%%%%%%%%%%%%%%%%%%%%%%%%%%%%%%%%%
Fixed-angle scattering is the regime of large-speed and (physical) fixed-angle. This can be stated as
\begin{equation}
	\frac{t}{s} \text{ fixed}, \qquad \frac{s}{m_{1} m_{2}} \rightarrow \infty, \qquad \frac{m_{1}}{m_{2}} \text{ fixed}.
\end{equation}
As a corollary, you have
\begin{equation}
	u = 2m_{1}^{2} + 2m_{2}^{2} - s - t \quad \Longrightarrow \quad \frac{u}{s} \text{ fixed}.
\end{equation}
In this regime all dual conformal invariants become trivial:
\begin{equation}
\begin{split}
	\crat{\red{1}}{\red{2}}{\red{3}}{\red{4}} \rightarrow -\infty, \qquad
	\crat{\red{1}}{\red{3}}{\red{4}}{\red{2}} &= 1, \qquad
	\crat{\red{1}}{\red{4}}{\red{2}}{\red{3}} \rightarrow 0, \\
	\crat{\red{1}}{\red{2}}{\red{4}}{\red{3}} \rightarrow 0, \qquad
	\crat{\red{1}}{\red{3}}{\red{2}}{\red{4}} &= 1, \qquad
	\crat{\red{1}}{\red{4}}{\red{3}}{\red{2}} \rightarrow -\infty.
\end{split}
\end{equation}
\begin{equation}
\begin{split}
	\crat{\blue{1}}{\blue{2}}{\blue{3}}{\blue{4}} \rightarrow \infty, \qquad
	\crat{\blue{1}}{\blue{3}}{\blue{4}}{\blue{2}} &= 1, \qquad
	\crat{\blue{1}}{\blue{4}}{\blue{2}}{\blue{3}} \rightarrow 0, \\
	\crat{\blue{1}}{\blue{2}}{\blue{4}}{\blue{3}} \rightarrow 0, \qquad
	\crat{\blue{1}}{\blue{3}}{\blue{2}}{\blue{4}} &= 1, \qquad
	\crat{\blue{1}}{\blue{4}}{\blue{3}}{\blue{2}} \rightarrow \infty.
\end{split}
\end{equation}
\begin{equation}
\begin{split}
	\crat{\green{1}}{\green{2}}{\green{3}}{\green{4}} \rightarrow -\infty, \qquad
	\crat{\green{1}}{\green{3}}{\green{4}}{\green{2}} &= \frac{m_{1}^{4}}{m_{2}^{4}}, \qquad
	\crat{\green{1}}{\green{4}}{\green{2}}{\green{3}} \rightarrow 0, \\
	\crat{\green{1}}{\green{2}}{\green{4}}{\green{3}} \rightarrow 0, \qquad
	\crat{\green{1}}{\green{3}}{\green{2}}{\green{4}} &= \frac{m_{2}^{4}}{m_{1}^{4}}, \qquad
	\crat{\green{1}}{\green{4}}{\green{3}}{\green{2}} \rightarrow -\infty.
\end{split}
\end{equation}
From the point of view of dual conformal invariants, Regge scattering and fixed-angle scattering are very similar.