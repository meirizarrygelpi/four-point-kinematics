\chapter{Introduction}
%%%%%%%%%%%%%%%%%%%%%%%%%%%%%%%%%%%%%%%%%%%%%%%%%%%%%%%%%%%%%%%%%%%%%%%%%%%%%%%%%%%%%%%%%%%%%%%%%%%%%%%%%%%%%%%%%%%%
...
%%%%%%%%%%%%%%%%%%%%%%%%%%%%%%%%%%%%%%%%%%%%%%%%%%%%%%%%%%%%%%%%%%%%%%%%%%%%%%%%%%%%%%%%%%%%%%%%%%%%%%%%%%%%%%%%%%%%
\section{Mandelstam Invariants}
%%%%%%%%%%%%%%%%%%%%%%%%%%%%%%%%%%%%%%%%%%%%%%%%%%%%%%%%%%%%%%%%%%%%%%%%%%%%%%%%%%%%%%%%%%%%%%%%%%%%%%%%%%%%%%%%%%%%
For convenience, one defines the three Mandelstam invariants:
\begin{equation}
	s = -\abs{p_{1} + p_{2}}^{2}, \qquad t = -\abs{p_{1} - p_{3}}^{2}, \qquad u = -\abs{p_{1} - p_{4}}^{2}. 
\end{equation}
Energy-momentum conservation leads to a constraint,
\begin{equation}
	p_{1} + p_{2} = p_{3} + p_{4}.
\end{equation}
This constraint means that only three out of the four external energy-momentum vectors are linearly independent. Due to the conservation constraint, the three Mandelstam invariants satisfy a linear relation:
\begin{equation}
	s + t + u = -\abs{p_{1}}^{2} - \abs{p_{2}}^{2} - \abs{p_{3}}^{2} - \abs{p_{4}}^{2}.
	\label{eq:stu_slow}
\end{equation}
Thus, only two of the Mandelstam invariants are linearly independent.
%%%%%%%%%%%%%%%%%%%%%%%%%%%%%%%%%%%%%%%%%%%%%%%%%%%%%%%%%%%%%%%%%%%%%%%%%%%%%%%%%%%%%%%%%%%%%%%%%%%%%%%%%%%%%%%%%%%%
\section{Dual Spacetime}
%%%%%%%%%%%%%%%%%%%%%%%%%%%%%%%%%%%%%%%%%%%%%%%%%%%%%%%%%%%%%%%%%%%%%%%%%%%%%%%%%%%%%%%%%%%%%%%%%%%%%%%%%%%%%%%%%%%%
You can solve the conservation constraint by introducing dual spacetime coordinates:
\begin{equation}
	p_{1} = d_{\red{1}} - d_{\red{2}}, \quad p_{2} = d_{\red{2}} - d_{\red{3}}, \quad p_{4} = d_{\red{4}} - d_{\red{3}}, \quad p_{3} = d_{\red{1}} - d_{\red{4}}.
	\label{eq:d_red}
\end{equation}
Thus, each energy-momentum vector becomes a distance interval in a dual spacetime. However, the solution (\ref{eq:d_red}) is not the only one allowed. Indeed, two other solutions are allowed:
\begin{equation}
	{-p_{3}} = d_{\blue{1}} - d_{\blue{2}}, \quad p_{1} = d_{\blue{2}} - d_{\blue{3}}, \quad p_{4} = d_{\blue{4}} - d_{\blue{3}}, \quad -p_{2} = d_{\blue{1}} - d_{\blue{4}},
	\label{eq:d_blue}
\end{equation}
and
\begin{equation}
	p_{2} = d_{\green{1}} - d_{\green{2}}, \quad -p_{3} = d_{\green{2}} - d_{\green{3}}, \quad p_{4} = d_{\green{4}} - d_{\green{3}}, \quad -p_{1} = d_{\green{1}} - d_{\green{4}}.
	\label{eq:d_green}
\end{equation}
I will refer to (\ref{eq:d_red}) as the \red{red} planar class, (\ref{eq:d_blue}) as the \blue{blue} planar class, and (\ref{eq:d_green}) as the \green{green} planar class. In each planar class, the dual spacetime coordinates describe the positions of four points in the dual spacetime. These four points can be taken as the vertices of a (Minkowski) tetrahedron. Many kinematic quantities can be understood in terms of the geometry of these (Minkowski) tetrahedra.
%%%%%%%%%%%%%%%%%%%%%%%%%%%%%%%%%%%%%%%%%%%%%%%%%%%%%%%%%%%%%%%%%%%%%%%%%%%%%%%%%%%%%%%%%%%%%%%%%%%%%%%%%%%%%%%%%%%%
\section{Simplicial Invariants}
%%%%%%%%%%%%%%%%%%%%%%%%%%%%%%%%%%%%%%%%%%%%%%%%%%%%%%%%%%%%%%%%%%%%%%%%%%%%%%%%%%%%%%%%%%%%%%%%%%%%%%%%%%%%%%%%%%%%
A tetrahedron is a 3-simplex. As such, it contains four vertices (0-simplex), six edges (1-simplex), four triangular faces (2-simplex), and one tetrahedron (3-simplex). The $n$-volume of an $n$-simplex is found by evaluating an $(n+1)$-point Cayley-Menger determinant. These are useful kinematic invariants.
%%%%%%%%%%%%%%%%%%%%%%%%%%%%%%%%%%%%%%%%%%%%%%%%%%%%%%%%%%%%%%%%%%%%%%%%%%%%%%%%%%%%%%%%%%%%%%%%%%%%%%%%%%%%%%%%%%%%
\subsection{1-Simplex Invariants}
%%%%%%%%%%%%%%%%%%%%%%%%%%%%%%%%%%%%%%%%%%%%%%%%%%%%%%%%%%%%%%%%%%%%%%%%%%%%%%%%%%%%%%%%%%%%%%%%%%%%%%%%%%%%%%%%%%%%
Given two dual spacetime positions, the 2-point Cayley-Menger determinant is given by:
\begin{equation}
	C_{IJ} = -\frac{1}{2} \det{
	\begin{pmatrix}
	0 & \abs{d_{IJ}}^{2} & 1 \\
	\abs{d_{IJ}}^{2} & 0 & 1 \\
	1 & 1 & 0
	\end{pmatrix}
	}, \qquad d_{IJ} \equiv d_{I} - d_{J}.
\end{equation}
Since a tetrahedron has six edges, there are six possible 1-simplex invariants for each tetrahedron. These invariants will be either squared masses or Mandelstam invariants.
%%%%%%%%%%%%%%%%%%%%%%%%%%%%%%%%%%%%%%%%%%%%%%%%%%%%%%%%%%%%%%%%%%%%%%%%%%%%%%%%%%%%%%%%%%%%%%%%%%%%%%%%%%%%%%%%%%%%
\subsection{2-Simplex Invariants}
%%%%%%%%%%%%%%%%%%%%%%%%%%%%%%%%%%%%%%%%%%%%%%%%%%%%%%%%%%%%%%%%%%%%%%%%%%%%%%%%%%%%%%%%%%%%%%%%%%%%%%%%%%%%%%%%%%%%
Given three dual spacetime positions, the 3-point Cayley-Menger determinant is given by:
\begin{equation}
	C_{IJK} = \det{
	\begin{pmatrix}
	0 & \abs{d_{IJ}}^{2} & \abs{d_{IK}}^{2} & 1 \\
	\abs{d_{IJ}}^{2} & 0 & \abs{d_{JK}}^{2} & 1 \\
	\abs{d_{IK}}^{2} & \abs{d_{JK}}^{2} & 0 & 1 \\
	1 & 1 & 1 & 0
	\end{pmatrix}
	}, \qquad d_{IJ} \equiv d_{I} - d_{J}.
\end{equation}
Since a tetrahedron has four triangular faces, there are four possible 2-simplex invariants for each tetrahedron. These invariants take the form of K\"{a}ll\'{e}n functions. I will introduce three kinds of K\"{a}ll\'{e}n functions.
%%%%%%%%%%%%%%%%%%%%%%%%%%%%%%%%%%%%%%%%%%%%%%%%%%%%%%%%%%%%%%%%%%%%%%%%%%%%%%%%%%%%%%%%%%%%%%%%%%%%%%%%%%%%%%%%%%%%
\subsubsection{Slow K\"{a}ll\'{e}n Function}
%%%%%%%%%%%%%%%%%%%%%%%%%%%%%%%%%%%%%%%%%%%%%%%%%%%%%%%%%%%%%%%%%%%%%%%%%%%%%%%%%%%%%%%%%%%%%%%%%%%%%%%%%%%%%%%%%%%%
The slow K\"{a}ll\'{e}n function is
\begin{equation}
	\Lambda_{ij}(x) = \left[ \left(x - m_{i}^{2} - m_{j}^{2} \right)^{2} - 4 m_{i}^{2} m_{j}^{2} \right] = \kallen{x}{m_{i}}{m_{j}},
\end{equation}
where $x$ is a Mandelstam invariant. Note that $\Lambda_{ij}(x)$ can be either negative, zero, or positive for real values of $x$.

If the slow K\"{a}ll\'{e}n function is negative, then
\begin{equation}
	\Lambda_{ij}(x) < 0 \quad \Longrightarrow \quad \abs{x - m_{i}^{2} - m_{j}^{2}} < 2 m_{i} m_{j},
\end{equation}
which leads to
\begin{equation}
	\Lambda_{ij}(x) < 0 \quad \Longrightarrow \quad x > \left(m_{i} - m_{j}\right)^{2} \text{ and } x < \left(m_{i} + m_{j}\right)^{2}.
\end{equation}
Similarly, if the slow K\"{a}ll\'{e}n function is zero, then
\begin{equation}
	\Lambda_{ij}(x) = 0 \quad \Longrightarrow \quad \abs{x - m_{i}^{2} - m_{j}^{2}} = 2 m_{i} m_{j},
\end{equation}
which leads to
\begin{equation}
	\Lambda_{ij}(x) = 0 \quad \Longrightarrow \quad x = \left(m_{i} - m_{j}\right)^{2} \text{ or } x = \left(m_{i} + m_{j}\right)^{2}.
\end{equation}
Finally, if the slow K\"{a}ll\'{e}n function is positive, then
\begin{equation}
	\Lambda_{ij}(x) > 0 \quad \Longrightarrow \quad \abs{x - m_{i}^{2} - m_{j}^{2}} > 2 m_{i} m_{j},
\end{equation}
which leads to
\begin{equation}
	\Lambda_{ij}(x) > 0 \quad \Longrightarrow \quad x < \left(m_{i} - m_{j}\right)^{2} \text{ or } x > \left(m_{i} + m_{j}\right)^{2}.
\end{equation}
%%%%%%%%%%%%%%%%%%%%%%%%%%%%%%%%%%%%%%%%%%%%%%%%%%%%%%%%%%%%%%%%%%%%%%%%%%%%%%%%%%%%%%%%%%%%%%%%%%%%%%%%%%%%%%%%%%%%
\subsubsection{Fast K\"{a}ll\'{e}n Function}
%%%%%%%%%%%%%%%%%%%%%%%%%%%%%%%%%%%%%%%%%%%%%%%%%%%%%%%%%%%%%%%%%%%%%%%%%%%%%%%%%%%%%%%%%%%%%%%%%%%%%%%%%%%%%%%%%%%%
The fast K\"{a}ll\'{e}n function is
\begin{equation}
	\Upsilon_{ij}(x) = \left[ \left(x + w_{i}^{2} + w_{j}^{2} \right)^{2} - 4 w_{i}^{2} w_{j}^{2} \right] = \kallenf{x}{w_{i}}{w_{j}},
\end{equation}
where $x$ is a Mandelstam invariant. Just like $\Lambda_{ij}(x)$, the function $\Upsilon_{ij}$ can be either negative, zero, or positive.

If the fast K\"{a}ll\'{e}n function is negative, then
\begin{equation}
	\Upsilon_{ij}(x) < 0 \quad \Longrightarrow \quad \abs{x + w_{i}^{2} + w_{j}^{2}} < 2 w_{i} w_{j},
\end{equation}
which leads to
\begin{equation}
	\Upsilon_{ij}(x) < 0 \quad \Longrightarrow \quad x > -\left(w_{i} + w_{j}\right)^{2} \text{ and } x < -\left(w_{i} - w_{j}\right)^{2}.
\end{equation}
Similarly, if the fast K\"{a}ll\'{e}n function is zero, then
\begin{equation}
	\Upsilon_{ij}(x) = 0 \quad \Longrightarrow \quad \abs{x + w_{i}^{2} + w_{j}^{2}} = 2 w_{i} w_{j},
\end{equation}
which leads to
\begin{equation}
	\Upsilon_{ij}(x) = 0 \quad \Longrightarrow \quad x = -\left(w_{i} + w_{j}\right)^{2} \text{ or } x = -\left(w_{i} - w_{j}\right)^{2}.
\end{equation}
Finally, if the fast K\"{a}ll\'{e}n function is positive, then
\begin{equation}
	\Upsilon_{ij}(x) > 0 \quad \Longrightarrow \quad \abs{x + w_{i}^{2} + w_{j}^{2}} > 2 w_{i} w_{j},
\end{equation}
which leads to
\begin{equation}
	\Upsilon_{ij}(x) > 0 \quad \Longrightarrow \quad x < -\left(w_{i} + w_{j}\right)^{2} \text{ or } x > -\left(w_{i} - w_{j}\right)^{2}.
\end{equation}
%%%%%%%%%%%%%%%%%%%%%%%%%%%%%%%%%%%%%%%%%%%%%%%%%%%%%%%%%%%%%%%%%%%%%%%%%%%%%%%%%%%%%%%%%%%%%%%%%%%%%%%%%%%%%%%%%%%%
\subsubsection{Mixed K\"{a}ll\'{e}n Function}
%%%%%%%%%%%%%%%%%%%%%%%%%%%%%%%%%%%%%%%%%%%%%%%%%%%%%%%%%%%%%%%%%%%%%%%%%%%%%%%%%%%%%%%%%%%%%%%%%%%%%%%%%%%%%%%%%%%%
The mixed K\"{a}ll\'{e}n function is
\begin{equation}
	\Omega_{ij}(x) = \left[ \left(x - m_{i}^{2} + w_{j}^{2} \right)^{2} + 4 m_{i}^{2} w_{j}^{2} \right],
\end{equation}
where $x$ is a Mandelstam invariant. Note that this function is always positive for $x$ real.
%%%%%%%%%%%%%%%%%%%%%%%%%%%%%%%%%%%%%%%%%%%%%%%%%%%%%%%%%%%%%%%%%%%%%%%%%%%%%%%%%%%%%%%%%%%%%%%%%%%%%%%%%%%%%%%%%%%%
\subsection{3-Simplex Invariants}
%%%%%%%%%%%%%%%%%%%%%%%%%%%%%%%%%%%%%%%%%%%%%%%%%%%%%%%%%%%%%%%%%%%%%%%%%%%%%%%%%%%%%%%%%%%%%%%%%%%%%%%%%%%%%%%%%%%%
Given four dual spacetime positions, the 4-point Cayley-Menger determinant is given by:
\begin{equation}
	C_{IJKL} = -\frac{1}{2} \det{
	\begin{pmatrix}
	0 & \abs{d_{IJ}}^{2} & \abs{d_{IK}}^{2} & \abs{d_{IL}}^{2} & 1 \\
	\abs{d_{IJ}}^{2} & 0 & \abs{d_{JK}}^{2} & \abs{d_{JL}}^{2} & 1 \\
	\abs{d_{IK}}^{2} & \abs{d_{JK}}^{2} & 0 & \abs{d_{KL}}^{2} & 1 \\
	\abs{d_{IL}}^{2} & \abs{d_{JL}}^{2} & \abs{d_{KL}}^{2} & 0 & 1 \\
	1 & 1 & 1 & 1 & 0
	\end{pmatrix}
	}, \qquad d_{IJ} \equiv d_{I} - d_{J}.
\end{equation}
A tetrahedron has only one possible 3-simplex invariant.
%%%%%%%%%%%%%%%%%%%%%%%%%%%%%%%%%%%%%%%%%%%%%%%%%%%%%%%%%%%%%%%%%%%%%%%%%%%%%%%%%%%%%%%%%%%%%%%%%%%%%%%%%%%%%%%%%%%%
\section{Dual Conformal Invariants}
%%%%%%%%%%%%%%%%%%%%%%%%%%%%%%%%%%%%%%%%%%%%%%%%%%%%%%%%%%%%%%%%%%%%%%%%%%%%%%%%%%%%%%%%%%%%%%%%%%%%%%%%%%%%%%%%%%%%
The expression $\abs{d_{I} - d_{J}}^{2}$ is not only Lorentz invariant, but dual Poincar\'{e} invariant. You can study dual conformal invariants by constructing a dual conformal ratio with a quartet of dual spacetime coordinates:
\begin{equation}
	\crat{I}{J}{K}{L} \equiv \frac{\abs{d_{IK}}^{2} \abs{d_{JL}}^{2}}{\abs{d_{IL}}^{2} \abs{d_{JK}}^{2}}, \qquad d_{IJ} \equiv d_{I} - d_{J}.
\end{equation}
In four-point scattering there is a unique quartet of dual spacetime coordinates. However, there are six inequivalent permutations of these coordinates, and thus, six possible values for the dual conformal ratio. Note that
\begin{equation}
	\crat{I}{J}{K}{L} \crat{I}{K}{L}{J} \crat{I}{L}{J}{K} = 1,
\end{equation}
which is analogous to the constraint satisfied by the three Mandelstam invariants.
%%%%%%%%%%%%%%%%%%%%%%%%%%%%%%%%%%%%%%%%%%%%%%%%%%%%%%%%%%%%%%%%%%%%%%%%%%%%%%%%%%%%%%%%%%%%%%%%%%%%%%%%%%%%%%%%%%%%
\section{Speed and Rapidity}
%%%%%%%%%%%%%%%%%%%%%%%%%%%%%%%%%%%%%%%%%%%%%%%%%%%%%%%%%%%%%%%%%%%%%%%%%%%%%%%%%%%%%%%%%%%%%%%%%%%%%%%%%%%%%%%%%%%%
The (slow) mass $m$, spatial momentum $\mathbf{p}$, and energy $E$ of a slow quantum satisfy the on-shell constraint
\begin{equation}
	{-m^{2}} = {-E^{2}} + \abs{\mathbf{p}}^{2}.
\end{equation}
This constraint can be solved by writing
\begin{equation}
	E = m \operatorname{cosh}{\rho}, \qquad \abs{\mathbf{p}} = m \operatorname{sinh}{\rho}.
\end{equation}
Thus,
\begin{equation}
	\operatorname{tanh}{\rho} = \frac{\sqrt{E^{2} - m^{2}}}{E} = \frac{\abs{\mathbf{p}}}{\sqrt{\abs{\mathbf{p}}^{2} + m^{2}}} = \frac{\abs{\mathbf{p}}}{E} \equiv \abs{\mathbf{v}}.
\end{equation}
Here $\rho$ is the (slow) rapidity, and $\abs{\mathbf{v}}$ is the speed of the slow quantum. Note that
\begin{equation}
	m^{2} \leq E^{2} < \infty \text{ or } 0 \leq \abs{\mathbf{p}}^{2} < \infty \quad \Longrightarrow \quad 0 \leq \abs{\mathbf{v}}^{2} < 1.
\end{equation}
That is, the speed of a slow quantum is bounded from above.

Similarly, the (fast) mass $w$, spatial momentum $\mathbf{p}$, and energy $E$ of a fast quantum satisfy the on-shell constraint
\begin{equation}
	{w^{2}} = {-E^{2}} + \abs{\mathbf{p}}^{2}.
\end{equation}
This constraint can be solved by writing
\begin{equation}
	E = w \operatorname{sinh}{\xi}, \qquad \abs{\mathbf{p}} = w \operatorname{cosh}{\xi}.
\end{equation}
Thus,
\begin{equation}
	\operatorname{coth}{\xi} = \frac{\sqrt{E^{2} + w^{2}}}{E} = \frac{\abs{\mathbf{p}}}{\sqrt{\abs{\mathbf{p}}^{2} - w^{2}}} = \frac{\abs{\mathbf{p}}}{E} \equiv \abs{\mathbf{v}}.
\end{equation}
Here $\xi$ is the (fast) rapidity, and $\abs{\mathbf{v}}$ is the speed of the fast quantum. Note that
\begin{equation}
	0 \leq E^{2} < \infty \text{ or } w^{2} \leq \abs{\mathbf{p}}^{2} < \infty \quad \Longrightarrow \quad 1 < \abs{\mathbf{v}}^{2} < \infty.
\end{equation}
That is, the speed of a fast quantum is bounded from below.

Recall that
\begin{equation}
	\operatorname{argtanh}{x} = \frac{1}{2} \log{\left( \frac{1 + x}{1 - x} \right)}, \qquad 0 \leq x < 1,
\end{equation}
and
\begin{equation}
	\operatorname{argcoth}{x} = \frac{1}{2} \log{\left( \frac{x + 1}{x - 1} \right)}, \qquad 1 < x < \infty.
\end{equation}
%%%%%%%%%%%%%%%%%%%%%%%%%%%%%%%%%%%%%%%%%%%%%%%%%%%%%%%%%%%%%%%%%%%%%%%%%%%%%%%%%%%%%%%%%%%%%%%%%%%%%%%%%%%%%%%%%%%%
\section{Two-Body Sudakov Decompositions}
%%%%%%%%%%%%%%%%%%%%%%%%%%%%%%%%%%%%%%%%%%%%%%%%%%%%%%%%%%%%%%%%%%%%%%%%%%%%%%%%%%%%%%%%%%%%%%%%%%%%%%%%%%%%%%%%%%%%
The external energy-momentum vectors can be divided into an incoming set ($p_{1}$ and $p_{2}$), and an outgoing set ($p_{3}$ and $p_{4}$). In a two-body Sudakov decomposition one postulates the existence of two Sudakov momenta $k_{3}$ and $k_{4}$ and writes the outgoing momenta as linear combinations of the Sudakov momenta:
\begin{equation}
	p_{3} = k_{3} + c_{34} k_{4}, \qquad p_{4} = c_{43} k_{3} + k_{4}.
\end{equation}
In other words, the outgoing and Sudakov momenta are related by a linear transformation:
\begin{equation}
	\begin{pmatrix} p_{3} \\
	p_{4} \end{pmatrix} = \begin{pmatrix} 1 & c_{34} \\
	c_{43} & 1 \end{pmatrix} \begin{pmatrix} k_{3} \\
	k_{4} \end{pmatrix}.
\end{equation}
In order for this transformation to be invertible, you need the determinant of the $2 \times 2$ matrix to be nonzero. The inverse transformation is
\begin{equation}
	\begin{pmatrix} k_{3} \\
	k_{4} \end{pmatrix} = \frac{1}{1 - c_{34} c_{43}} \begin{pmatrix} 1 & -c_{34} \\
	-c_{43} & 1 \end{pmatrix} \begin{pmatrix} p_{3} \\
	p_{4} \end{pmatrix}.
\end{equation}
That is, in terms of the outgoing momenta, the Sudakov momenta are
\begin{equation}
	k_{3} = \frac{p_{3} - c_{34} p_{4}}{1 - c_{34} c_{43}}, \qquad k_{4} = \frac{p_{4} - c_{43} p_{3}}{1 - c_{34} c_{43}}.
\end{equation}
% Thus,
% \begin{align}
% 	\abs{k_{3}}^{2} &= \frac{\abs{p_{3}}^{2} - 2 c_{34} \left(p_{3} \cdot p_{4}\right) + c_{34}^{2} \abs{p_{4}}^{2}}{\left( 1 - c_{34} c_{43} \right)^{2}}, \\
% 	\abs{k_{4}}^{2} &= \frac{c_{43}^{2}\abs{p_{3}}^{2} - 2 c_{43} \left(p_{3} \cdot p_{4}\right) + \abs{p_{4}}^{2}}{\left( 1 - c_{34} c_{43} \right)^{2}}, \\
% 	\left(k_{3} \cdot k_{4}\right) &= \frac{ \left( 1 + c_{34}c_{43} \right) \left(p_{3} \cdot p_{4}\right) - c_{43} \abs{p_{3}}^{2} - c_{34} \abs{p_{4}}^{2}}{\left( 1 - c_{34} c_{43} \right)^{2}}.
% \end{align}
You have
\begin{align}
	\abs{p_{3}}^{2} &= \abs{k_{3}}^{2} + c_{34}^{2} \abs{k_{4}}^{2} + 2 c_{34} \left( k_{3} \cdot k_{4} \right), \\
	\abs{p_{4}}^{2} &= c_{43}^{2} \abs{k_{3}}^{2} + \abs{k_{4}}^{2} + 2 c_{43} \left( k_{3} \cdot k_{4} \right), \\
	\left(p_{3} \cdot p_{4} \right) &= c_{43} \abs{k_{3}}^{2} + c_{34} \abs{k_{4}}^{2} + \left( 1 + c_{34} c_{43} \right) \left( k_{3} \cdot k_{4} \right).
\end{align}
and thus
\begin{equation}
	\abs{p_{3}}^{2} \abs{p_{4}}^{2} - \left( p_{3} \cdot p_{4} \right)^{2} = \left(1 - c_{34} c_{43} \right)^{2} \left( \abs{k_{3}}^{2} \abs{k_{4}}^{2} - \left( k_{3} \cdot k_{4} \right)^{2} \right)
\end{equation}
You know the value of $\abs{p_{3}}^{2}$, $\abs{p_{4}}^{2}$, and $p_{3} \cdot p_{4}$, so you can eliminate at most three of the five unknowns. Thus, in order for this decomposition to be useful, you need to specify two additional pieces of information.

Each of the incoming momenta is decomposed into a part that is orthogonal to the subspace spanned by the two Sudakov momenta, and a part that is parallel to this subspace:
\begin{align}
	p_{1} &= P_{1} + c_{13} k_{3} + c_{14} k_{4}, \qquad P_{1} \cdot k_{3} = 0, \qquad P_{1} \cdot k_{4} = 0, \\
	p_{2} &= P_{2} + c_{23} k_{3} + c_{24} k_{4}, \qquad P_{2} \cdot k_{3} = 0, \qquad P_{2} \cdot k_{4} = 0.
\end{align}
Here $P_{1}$ and $P_{2}$ are said to be transversal. It follows that
\begin{align}
	c_{13} = \frac{ \abs{k_{4}}^{2} \left(p_{1} \cdot k_{3} \right) - \left(k_{3} \cdot k_{4} \right) \left(p_{1} \cdot k_{4} \right) }{\abs{k_{3}}^{2} \abs{k_{4}}^{2} - \left( k_{3} \cdot k_{4} \right)^{2}}, &\qquad
	c_{14} = \frac{ \abs{k_{3}}^{2} \left(p_{1} \cdot k_{4} \right) - \left(k_{3} \cdot k_{4} \right) \left(p_{1} \cdot k_{3} \right) }{\abs{k_{3}}^{2} \abs{k_{4}}^{2} - \left( k_{3} \cdot k_{4} \right)^{2}}, \\
	c_{23} = \frac{\abs{k_{4}}^{2} \left(p_{2} \cdot k_{3} \right) - \left(k_{3} \cdot k_{4} \right) \left(p_{2} \cdot k_{4} \right)}{\abs{k_{3}}^{2} \abs{k_{4}}^{2} - \left( k_{3} \cdot k_{4} \right)^{2}}, &\qquad
	c_{24} = \frac{ \abs{k_{3}}^{2} \left(p_{2} \cdot k_{4} \right) - \left(k_{3} \cdot k_{4} \right) \left(p_{2} \cdot k_{3} \right) }{\abs{k_{3}}^{2} \abs{k_{4}}^{2} - \left( k_{3} \cdot k_{4} \right)^{2}}.
\end{align}
Note that
\begin{equation}
	\abs{p_{1}}^{2} = \abs{P_{1}}^{2} + \abs{c_{13} k_{3} + c_{14} k_{4}}^{2}, \qquad \abs{p_{2}}^{2} = \abs{P_{2}}^{2} + \abs{c_{23} k_{3} + c_{24} k_{4}}^{2}.
\end{equation}
Thus,
\begin{align}
	\abs{P_{1}}^{2} &= \abs{p_{1}}^{2} - c_{13}^{2} \abs{k_{3}}^{2} - c_{14}^{2} \abs{k_{4}}^{2} - 2 c_{13} c_{14} \left( k_{3} \cdot k_{4} \right), \\
	\abs{P_{2}}^{2} &= \abs{p_{2}}^{2} - c_{23}^{2} \abs{k_{3}}^{2} - c_{24}^{2} \abs{k_{4}}^{2} - 2 c_{23} c_{24} \left( k_{3} \cdot k_{4} \right).
\end{align}
It is also useful to decompose the linear combinations of external momenta that yield Mandelstam invariants:
\begin{align}
	p_{1} + p_{2} &= P_{1} + P_{2} + \left(c_{13} + c_{23}\right) k_{3} + \left(c_{14} + c_{24}\right) k_{4}, \\
	p_{3} + p_{4} &= \left(1 + c_{43}\right) k_{3} + \left(c_{34} + 1\right) k_{4}, \\
	p_{1} - p_{3} &= P_{1} + \left(c_{13} - 1\right) k_{3} + \left(c_{14} - c_{34}\right) k_{4}, \\
	p_{4} - p_{2} &= -P_{2} + \left(c_{43} - c_{23}\right) k_{3} + \left(1 - c_{24}\right) k_{4}, \\
	p_{1} - p_{4} &= P_{1} + \left(c_{13} - c_{43}\right) k_{3} + \left(c_{14} - 1\right) k_{4}, \\
	p_{3} - p_{2} &= -P_{2} + \left(1 - c_{23}\right) k_{3} + \left(c_{34} - c_{24}\right) k_{4}.
\end{align}
From the conservation constraint, you find
\begin{equation}
	p_{1} + p_{2} = p_{3} + p_{4} \quad \Longrightarrow \quad c_{13} + c_{23} = 1 + c_{43}, \qquad c_{14} + c_{24} = c_{34} + 1.
\end{equation}
It also follows that $P_{1} + P_{2} = 0$.