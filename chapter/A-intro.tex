\chapter{Introduction}
%%%%%%%%%%%%%%%%%%%%%%%%%%%%%%%%%%%%%%%%%%%%%%%%%%%%%%%%%%%%%%%%%%%%%%%%%%%%%%%%%%%%%%%%%%%%%%%%%%%%%%%%%%%%%%%%%%%%
...
%%%%%%%%%%%%%%%%%%%%%%%%%%%%%%%%%%%%%%%%%%%%%%%%%%%%%%%%%%%%%%%%%%%%%%%%%%%%%%%%%%%%%%%%%%%%%%%%%%%%%%%%%%%%%%%%%%%%
\section{Slow Quanta}
%%%%%%%%%%%%%%%%%%%%%%%%%%%%%%%%%%%%%%%%%%%%%%%%%%%%%%%%%%%%%%%%%%%%%%%%%%%%%%%%%%%%%%%%%%%%%%%%%%%%%%%%%%%%%%%%%%%%
Consider four slow momenta:
\begin{equation}
	m_{1}^{2} = -\abs{p_{1}}^{2}, \qquad m_{2}^{2} = -\abs{p_{2}}^{2}, \qquad m_{3}^{2} = -\abs{p_{3}}^{2}, \qquad m_{4}^{2} = -\abs{p_{4}}^{2}.
\end{equation}
%%%%%%%%%%%%%%%%%%%%%%%%%%%%%%%%%%%%%%%%%%%%%%%%%%%%%%%%%%%%%%%%%%%%%%%%%%%%%%%%%%%%%%%%%%%%%%%%%%%%%%%%%%%%%%%%%%%%
\subsection{Mandelstam Invariants}
%%%%%%%%%%%%%%%%%%%%%%%%%%%%%%%%%%%%%%%%%%%%%%%%%%%%%%%%%%%%%%%%%%%%%%%%%%%%%%%%%%%%%%%%%%%%%%%%%%%%%%%%%%%%%%%%%%%%
For convenience, one defines the three (slow) Mandelstam invariants:
\begin{equation}
	s = -\abs{p_{1} + p_{2}}^{2}, \qquad t = -\abs{p_{1} - p_{3}}^{2}, \qquad u = -\abs{p_{1} - p_{4}}^{2}. 
\end{equation}
Energy-momentum conservation leads to a constraint,
\begin{equation}
	p_{1} + p_{2} = p_{3} + p_{4}.
\end{equation}
This constraint means that only three out of the four external energy-momentum vectors are linearly independent. Due to the conservation constraint, the three Mandelstam invariants satisfy a linear relation:
\begin{equation}
	s + t + u = m_{1}^{2} + m_{2}^{2} + m_{3}^{2} + m_{4}^{2}.
	\label{eq:stu_slow}
\end{equation}
Thus, only two of the Mandelstam invariants are linearly independent.
%%%%%%%%%%%%%%%%%%%%%%%%%%%%%%%%%%%%%%%%%%%%%%%%%%%%%%%%%%%%%%%%%%%%%%%%%%%%%%%%%%%%%%%%%%%%%%%%%%%%%%%%%%%%%%%%%%%%
\subsection{Regge-Mandelstam Invariants}
%%%%%%%%%%%%%%%%%%%%%%%%%%%%%%%%%%%%%%%%%%%%%%%%%%%%%%%%%%%%%%%%%%%%%%%%%%%%%%%%%%%%%%%%%%%%%%%%%%%%%%%%%%%%%%%%%%%%
The Regge-Mandelstam invariants are useful dimensionless combinations. There are six:
\begin{align}
	\sigma_{12} = \frac{s - m_{1}^{2} - m_{2}^{2}}{2 m_{1} m_{2}}, \qquad \tau_{13} = \frac{t - m_{1}^{2} - m_{3}^{2}}{2 m_{1} m_{3}}, \qquad \upsilon_{14} = \frac{u - m_{1}^{2} - m_{4}^{2}}{2 m_{1} m_{4}}, \\
	\upsilon_{23} = \frac{u - m_{2}^{2} - m_{3}^{2}}{2 m_{2} m_{3}}, \qquad \tau_{24} = \frac{t - m_{2}^{2} - m_{4}^{2}}{2 m_{2} m_{4}}, \qquad \sigma_{34} = \frac{s - m_{3}^{2} - m_{4}^{2}}{2 m_{3} m_{4}}.
\end{align}
Note that
\begin{align}
	s < \left( m_{1} - m_{2} \right)^{2} \quad &\Longrightarrow \quad \sigma_{12} < -1, \\
	s = \left( m_{1} - m_{2} \right)^{2} \quad &\Longrightarrow \quad \sigma_{12} = -1, \\
	\left( m_{1} - m_{2} \right)^{2} < s < m_{1}^{2} + m_{2}^{2} \quad &\Longrightarrow \quad -1 < \sigma_{12} < 0, \\
	s = m_{1}^{2} + m_{2}^{2} \quad &\Longrightarrow \quad \sigma_{12} = 0, \\
	m_{1}^{2} + m_{2}^{2} < s < \left( m_{1} + m_{2} \right)^{2} \quad &\Longrightarrow \quad 0 < \sigma_{12} < 1, \\
	s = \left( m_{1} + m_{2} \right)^{2} \quad &\Longrightarrow \quad \sigma_{12} = 1, \\
	s > \left( m_{1} + m_{2} \right)^{2} \quad &\Longrightarrow \quad \sigma_{12} > 1.
\end{align}
Similar remarks hold for the other Regge-Mandelstam invariants.
%%%%%%%%%%%%%%%%%%%%%%%%%%%%%%%%%%%%%%%%%%%%%%%%%%%%%%%%%%%%%%%%%%%%%%%%%%%%%%%%%%%%%%%%%%%%%%%%%%%%%%%%%%%%%%%%%%%%
\subsection{K\"{a}ll\'{e}n Function}
%%%%%%%%%%%%%%%%%%%%%%%%%%%%%%%%%%%%%%%%%%%%%%%%%%%%%%%%%%%%%%%%%%%%%%%%%%%%%%%%%%%%%%%%%%%%%%%%%%%%%%%%%%%%%%%%%%%%
The slow K\"{a}ll\'{e}n function is
\begin{equation}
	\Lambda_{ij}(x) = \left[ \left(x - m_{i}^{2} - m_{j}^{2} \right)^{2} - 4 m_{i}^{2} m_{j}^{2} \right] = \kallen{x}{m_{i}}{m_{j}},
\end{equation}
where $x$ is a (slow) Mandelstam invariant. Note that $\Lambda_{ij}(x)$ can be either negative, zero, or positive for real values of $x$.

If the slow K\"{a}ll\'{e}n function is negative, then
\begin{equation}
	\Lambda_{ij}(x) < 0 \quad \Longrightarrow \quad \abs{x - m_{i}^{2} - m_{j}^{2}} < 2 m_{i} m_{j},
\end{equation}
which leads to
\begin{equation}
	\Lambda_{ij}(x) < 0 \quad \Longrightarrow \quad x > \left(m_{i} - m_{j}\right)^{2} \text{ and } x < \left(m_{i} + m_{j}\right)^{2}.
\end{equation}
Similarly, if the slow K\"{a}ll\'{e}n function is zero, then
\begin{equation}
	\Lambda_{ij}(x) = 0 \quad \Longrightarrow \quad \abs{x - m_{i}^{2} - m_{j}^{2}} = 2 m_{i} m_{j},
\end{equation}
which leads to
\begin{equation}
	\Lambda_{ij}(x) = 0 \quad \Longrightarrow \quad x = \left(m_{i} - m_{j}\right)^{2} \text{ or } x = \left(m_{i} + m_{j}\right)^{2}.
\end{equation}
Finally, if the slow K\"{a}ll\'{e}n function is positive, then
\begin{equation}
	\Lambda_{ij}(x) > 0 \quad \Longrightarrow \quad \abs{x - m_{i}^{2} - m_{j}^{2}} > 2 m_{i} m_{j},
\end{equation}
which leads to
\begin{equation}
	\Lambda_{ij}(x) > 0 \quad \Longrightarrow \quad x < \left(m_{i} - m_{j}\right)^{2} \text{ or } x > \left(m_{i} + m_{j}\right)^{2}.
\end{equation}
The region where $\Lambda_{ij}(x)$ is negative is called the (slow-slow) Coulomb region. If one of the (slow) masses becomes zero, you get the null-slow K\"{a}ll\'{e}n function:
\begin{equation}
	\Lambda_{i}(x) = \left( x - m_{i}^{2} \right)^{2},
\end{equation}
which is positive for any real value of $x$. That is, there is no (null-slow) Coulomb region.
%%%%%%%%%%%%%%%%%%%%%%%%%%%%%%%%%%%%%%%%%%%%%%%%%%%%%%%%%%%%%%%%%%%%%%%%%%%%%%%%%%%%%%%%%%%%%%%%%%%%%%%%%%%%%%%%%%%%
\subsection{Speed and Rapidity}
%%%%%%%%%%%%%%%%%%%%%%%%%%%%%%%%%%%%%%%%%%%%%%%%%%%%%%%%%%%%%%%%%%%%%%%%%%%%%%%%%%%%%%%%%%%%%%%%%%%%%%%%%%%%%%%%%%%%
The (slow) mass $m$, spatial momentum $\mathbf{p}$, and energy $E$ of a slow quantum satisfy the on-shell constraint
\begin{equation}
	{-m^{2}} = {-E^{2}} + \abs{\mathbf{p}}^{2}.
\end{equation}
This constraint can be solved by writing
\begin{equation}
	E = m \operatorname{cosh}{\rho}, \qquad \abs{\mathbf{p}} = m \operatorname{sinh}{\rho}.
\end{equation}
Thus,
\begin{equation}
	\operatorname{tanh}{\rho} = \frac{\sqrt{E^{2} - m^{2}}}{E} = \frac{\abs{\mathbf{p}}}{\sqrt{\abs{\mathbf{p}}^{2} + m^{2}}} = \frac{\abs{\mathbf{p}}}{E} \equiv \abs{\mathbf{v}}.
\end{equation}
Here $\rho$ is the (slow) rapidity, and $\abs{\mathbf{v}}$ is the speed of the slow quantum. Note that
\begin{equation}
	m^{2} \leq E^{2} < \infty \text{ or } 0 \leq \abs{\mathbf{p}}^{2} < \infty \quad \Longrightarrow \quad 0 \leq \abs{\mathbf{v}}^{2} < 1.
\end{equation}
That is, the speed of a slow quantum is bounded from above.

Recall that
\begin{equation}
	\operatorname{artanh}{\abs{\mathbf{v}}} = \frac{1}{2} \log{\left( \frac{1 + \abs{\mathbf{v}}}{1 - \abs{\mathbf{v}}} \right)}, \qquad 0 \leq \abs{\mathbf{v}} < 1,
\end{equation}
%%%%%%%%%%%%%%%%%%%%%%%%%%%%%%%%%%%%%%%%%%%%%%%%%%%%%%%%%%%%%%%%%%%%%%%%%%%%%%%%%%%%%%%%%%%%%%%%%%%%%%%%%%%%%%%%%%%%
\subsection{Sudakov Null Decomposition}
%%%%%%%%%%%%%%%%%%%%%%%%%%%%%%%%%%%%%%%%%%%%%%%%%%%%%%%%%%%%%%%%%%%%%%%%%%%%%%%%%%%%%%%%%%%%%%%%%%%%%%%%%%%%%%%%%%%%
The linear combination of two null vectors is, in general, not a null vector. Consider writing the outgoing momenta ($p_{3}$ and $p_{4}$) as
\begin{equation}
	p_{3} = k_{3} - \frac{m_{3}^{2}}{S_{34}} k_{4}, \qquad p_{4} = k_{4} - \frac{m_{4}^{2}}{S_{34}} k_{3}; \qquad S_{34} = -\abs{k_{3} + k_{4}}^{2}.
\end{equation}
Here $k_{3}$ and $k_{4}$ are (null) Sudakov momenta. From $s = -\abs{p_{3} + p_{4}}^{2}$ you find
\begin{equation}
	S_{34} = \frac{1}{2} \left[ s - m_{3}^{2} - m_{4}^{2} + \sqrt{\Lambda_{34}(s)} \right].
\end{equation}
The other momenta in the problem are $p_{1}$ and $p_{2}$. The null decomposition consists of splitting these momenta:
\begin{equation}
	p_{1} = P_{1} + \kappa_{13} k_{3} + \kappa_{14} k_{4}, \qquad p_{2} = P_{2} + \kappa_{23} k_{3} + \kappa_{24} k_{4}.
\end{equation}
Here $P_{1}$ and $P_{2}$ are orthogonal to both $k_{3}$ and $k_{4}$:
\begin{equation}
	P_{1} \cdot k_{3} = P_{1} \cdot k_{4} = 0, \qquad P_{2} \cdot k_{3} = P_{2} \cdot k_{4} = 0.
\end{equation}
From the conservation constraint, you find
\begin{equation}
	P_{1} + P_{2} + \left( \kappa_{13} + \kappa_{23} \right) k_{3} + \left( \kappa_{14} + \kappa_{24} \right) k_{4} = \left(1 - \frac{m_{4}^{2}}{S_{34}} \right) k_{3} + \left(1 - \frac{m_{3}^{2}}{S_{34}} \right) k_{4}.
\end{equation}
Thus,
\begin{equation}
	P_{1} + P_{2} = 0, \qquad \kappa_{13} + \kappa_{23} = 1 - \frac{m_{4}^{2}}{S_{34}}, \qquad \kappa_{14} + \kappa_{24} = 1 - \frac{m_{3}^{2}}{S_{34}}.
\end{equation}
%%%%%%%%%%%%%%%%%%%%%%%%%%%%%%%%%%%%%%%%%%%%%%%%%%%%%%%%%%%%%%%%%%%%%%%%%%%%%%%%%%%%%%%%%%%%%%%%%%%%%%%%%%%%%%%%%%%%
\subsection{Sudakov Null Basis}
%%%%%%%%%%%%%%%%%%%%%%%%%%%%%%%%%%%%%%%%%%%%%%%%%%%%%%%%%%%%%%%%%%%%%%%%%%%%%%%%%%%%%%%%%%%%%%%%%%%%%%%%%%%%%%%%%%%%
Instead of decomposing the incoming momenta in terms of the outgoing Sudakov momenta, you can also introduce incoming (null) Sudakov momenta:
\begin{equation}
	p_{1} = k_{1} - \frac{m_{1}^{2}}{S_{12}} k_{2}, \qquad p_{2} = k_{2} - \frac{m_{2}^{2}}{S_{12}} k_{1}; \qquad S_{12} = -\abs{k_{1} + k_{2}}^{2}.
\end{equation}
Just like before, from $s = -\abs{p_{1} + p_{2}}^{2}$ you find
\begin{equation}
	S_{12} = \frac{1}{2} \left[ s - m_{1}^{2} - m_{2}^{2} + \sqrt{\Lambda_{12}(s)} \right].
\end{equation}
The invariants $S_{12}$ and $S_{34}$ are examples of Sudakov invariants. These are like Mandelstam invariants, but with the null Sudakov momenta instead. There are four other Sudakov invariants:
\begin{equation}
\begin{split}
	T_{13} = - \abs{k_{1} - k_{3}}^{2}, &\qquad U_{14} = - \abs{k_{1} - k_{4}}^{2}, \\
	U_{23} = - \abs{k_{2} - k_{3}}^{2}, &\qquad T_{24} = - \abs{k_{2} - k_{4}}^{2}.
\end{split}
\end{equation}
%%%%%%%%%%%%%%%%%%%%%%%%%%%%%%%%%%%%%%%%%%%%%%%%%%%%%%%%%%%%%%%%%%%%%%%%%%%%%%%%%%%%%%%%%%%%%%%%%%%%%%%%%%%%%%%%%%%%
\section{Fast Quanta}
%%%%%%%%%%%%%%%%%%%%%%%%%%%%%%%%%%%%%%%%%%%%%%%%%%%%%%%%%%%%%%%%%%%%%%%%%%%%%%%%%%%%%%%%%%%%%%%%%%%%%%%%%%%%%%%%%%%%
Now consider four fast momenta:
\begin{equation}
	w_{1}^{2} = \abs{p_{1}}^{2}, \qquad w_{2}^{2} = \abs{p_{2}}^{2}, \qquad w_{3}^{2} = \abs{p_{3}}^{2}, \qquad w_{4}^{2} = \abs{p_{4}}^{2}.
\end{equation}
%%%%%%%%%%%%%%%%%%%%%%%%%%%%%%%%%%%%%%%%%%%%%%%%%%%%%%%%%%%%%%%%%%%%%%%%%%%%%%%%%%%%%%%%%%%%%%%%%%%%%%%%%%%%%%%%%%%%
\subsection{Mandelstam Invariants}
%%%%%%%%%%%%%%%%%%%%%%%%%%%%%%%%%%%%%%%%%%%%%%%%%%%%%%%%%%%%%%%%%%%%%%%%%%%%%%%%%%%%%%%%%%%%%%%%%%%%%%%%%%%%%%%%%%%%
For convenience, one defines the three (fast) Mandelstam invariants:
\begin{equation}
	x = \abs{p_{1} + p_{2}}^{2}, \qquad y = \abs{p_{1} - p_{3}}^{2}, \qquad z = \abs{p_{1} - p_{4}}^{2}. 
\end{equation}
Energy-momentum conservation leads to a constraint,
\begin{equation}
	p_{1} + p_{2} = p_{3} + p_{4}.
\end{equation}
This constraint means that only three out of the four external energy-momentum vectors are linearly independent. Due to the conservation constraint, the three Mandelstam invariants satisfy a linear relation:
\begin{equation}
	x + y + z = w_{1}^{2} + w_{2}^{2} + w_{3}^{2} + w_{4}^{2}.
	\label{eq:stu_fast}
\end{equation}
Thus, only two of the Mandelstam invariants are linearly independent.
%%%%%%%%%%%%%%%%%%%%%%%%%%%%%%%%%%%%%%%%%%%%%%%%%%%%%%%%%%%%%%%%%%%%%%%%%%%%%%%%%%%%%%%%%%%%%%%%%%%%%%%%%%%%%%%%%%%%
\subsection{Regge-Mandelstam Invariants}
%%%%%%%%%%%%%%%%%%%%%%%%%%%%%%%%%%%%%%%%%%%%%%%%%%%%%%%%%%%%%%%%%%%%%%%%%%%%%%%%%%%%%%%%%%%%%%%%%%%%%%%%%%%%%%%%%%%%
...
%%%%%%%%%%%%%%%%%%%%%%%%%%%%%%%%%%%%%%%%%%%%%%%%%%%%%%%%%%%%%%%%%%%%%%%%%%%%%%%%%%%%%%%%%%%%%%%%%%%%%%%%%%%%%%%%%%%%
\subsection{K\"{a}ll\'{e}n Function}
%%%%%%%%%%%%%%%%%%%%%%%%%%%%%%%%%%%%%%%%%%%%%%%%%%%%%%%%%%%%%%%%%%%%%%%%%%%%%%%%%%%%%%%%%%%%%%%%%%%%%%%%%%%%%%%%%%%%
The fast K\"{a}ll\'{e}n function is
\begin{equation}
	\Upsilon_{ij}(x) = \left[ \left(x + w_{i}^{2} + w_{j}^{2} \right)^{2} - 4 w_{i}^{2} w_{j}^{2} \right] = \kallenf{x}{w_{i}}{w_{j}},
\end{equation}
where $x$ is a (slow) Mandelstam invariant. Just like $\Lambda_{ij}(x)$, the function $\Upsilon_{ij}$ can be either negative, zero, or positive for real values of $x$.

If the fast K\"{a}ll\'{e}n function is negative, then
\begin{equation}
	\Upsilon_{ij}(x) < 0 \quad \Longrightarrow \quad \abs{x + w_{i}^{2} + w_{j}^{2}} < 2 w_{i} w_{j},
\end{equation}
which leads to
\begin{equation}
	\Upsilon_{ij}(x) < 0 \quad \Longrightarrow \quad x > -\left(w_{i} + w_{j}\right)^{2} \text{ and } x < -\left(w_{i} - w_{j}\right)^{2}.
\end{equation}
Similarly, if the fast K\"{a}ll\'{e}n function is zero, then
\begin{equation}
	\Upsilon_{ij}(x) = 0 \quad \Longrightarrow \quad \abs{x + w_{i}^{2} + w_{j}^{2}} = 2 w_{i} w_{j},
\end{equation}
which leads to
\begin{equation}
	\Upsilon_{ij}(x) = 0 \quad \Longrightarrow \quad x = -\left(w_{i} + w_{j}\right)^{2} \text{ or } x = -\left(w_{i} - w_{j}\right)^{2}.
\end{equation}
Finally, if the fast K\"{a}ll\'{e}n function is positive, then
\begin{equation}
	\Upsilon_{ij}(x) > 0 \quad \Longrightarrow \quad \abs{x + w_{i}^{2} + w_{j}^{2}} > 2 w_{i} w_{j},
\end{equation}
which leads to
\begin{equation}
	\Upsilon_{ij}(x) > 0 \quad \Longrightarrow \quad x < -\left(w_{i} + w_{j}\right)^{2} \text{ or } x > -\left(w_{i} - w_{j}\right)^{2}.
\end{equation}
The region where $\Upsilon_{ij}(x)$ is negative is called the (fast-fast) Coulomb region. If one of the (fast) masses becomes zero, you get the null-fast K\"{a}ll\'{e}n function:
\begin{equation}
	\Upsilon_{i}(x) = \left( x + w_{i}^{2} \right)^{2},
\end{equation}
%%%%%%%%%%%%%%%%%%%%%%%%%%%%%%%%%%%%%%%%%%%%%%%%%%%%%%%%%%%%%%%%%%%%%%%%%%%%%%%%%%%%%%%%%%%%%%%%%%%%%%%%%%%%%%%%%%%%
\subsection{Speed and Rapidity}
%%%%%%%%%%%%%%%%%%%%%%%%%%%%%%%%%%%%%%%%%%%%%%%%%%%%%%%%%%%%%%%%%%%%%%%%%%%%%%%%%%%%%%%%%%%%%%%%%%%%%%%%%%%%%%%%%%%%
The (fast) mass $w$, spatial momentum $\mathbf{p}$, and energy $E$ of a fast quantum satisfy the on-shell constraint
\begin{equation}
	{w^{2}} = {-E^{2}} + \abs{\mathbf{p}}^{2}.
\end{equation}
This constraint can be solved by writing
\begin{equation}
	E = w \operatorname{sinh}{\xi}, \qquad \abs{\mathbf{p}} = w \operatorname{cosh}{\xi}.
\end{equation}
Thus,
\begin{equation}
	\operatorname{coth}{\xi} = \frac{\sqrt{E^{2} + w^{2}}}{E} = \frac{\abs{\mathbf{p}}}{\sqrt{\abs{\mathbf{p}}^{2} - w^{2}}} = \frac{\abs{\mathbf{p}}}{E} \equiv \abs{\mathbf{v}}.
\end{equation}
Here $\xi$ is the (fast) rapidity, and $\abs{\mathbf{v}}$ is the speed of the fast quantum. Note that
\begin{equation}
	0 \leq E^{2} < \infty \text{ or } w^{2} \leq \abs{\mathbf{p}}^{2} < \infty \quad \Longrightarrow \quad 1 < \abs{\mathbf{v}}^{2} < \infty.
\end{equation}
That is, the speed of a fast quantum is bounded from below.

Recall that
\begin{equation}
	\operatorname{arcoth}{x} = \frac{1}{2} \log{\left( \frac{x + 1}{x - 1} \right)}, \qquad 1 < x < \infty.
\end{equation}
%%%%%%%%%%%%%%%%%%%%%%%%%%%%%%%%%%%%%%%%%%%%%%%%%%%%%%%%%%%%%%%%%%%%%%%%%%%%%%%%%%%%%%%%%%%%%%%%%%%%%%%%%%%%%%%%%%%%
\subsection{Sudakov Null Decomposition}
%%%%%%%%%%%%%%%%%%%%%%%%%%%%%%%%%%%%%%%%%%%%%%%%%%%%%%%%%%%%%%%%%%%%%%%%%%%%%%%%%%%%%%%%%%%%%%%%%%%%%%%%%%%%%%%%%%%%
...
%%%%%%%%%%%%%%%%%%%%%%%%%%%%%%%%%%%%%%%%%%%%%%%%%%%%%%%%%%%%%%%%%%%%%%%%%%%%%%%%%%%%%%%%%%%%%%%%%%%%%%%%%%%%%%%%%%%%
\subsection{Sudakov Null Basis}
%%%%%%%%%%%%%%%%%%%%%%%%%%%%%%%%%%%%%%%%%%%%%%%%%%%%%%%%%%%%%%%%%%%%%%%%%%%%%%%%%%%%%%%%%%%%%%%%%%%%%%%%%%%%%%%%%%%%
...
%%%%%%%%%%%%%%%%%%%%%%%%%%%%%%%%%%%%%%%%%%%%%%%%%%%%%%%%%%%%%%%%%%%%%%%%%%%%%%%%%%%%%%%%%%%%%%%%%%%%%%%%%%%%%%%%%%%%
\section{Dual Spacetime}
%%%%%%%%%%%%%%%%%%%%%%%%%%%%%%%%%%%%%%%%%%%%%%%%%%%%%%%%%%%%%%%%%%%%%%%%%%%%%%%%%%%%%%%%%%%%%%%%%%%%%%%%%%%%%%%%%%%%
You can solve the conservation constraint by introducing dual spacetime coordinates:
\begin{equation}
	p_{1} = d_{\red{1}} - d_{\red{2}}, \quad p_{2} = d_{\red{2}} - d_{\red{3}}, \quad p_{4} = d_{\red{4}} - d_{\red{3}}, \quad p_{3} = d_{\red{1}} - d_{\red{4}}.
	\label{eq:d_red}
\end{equation}
Thus, each energy-momentum vector becomes a distance interval in a dual spacetime. However, the solution (\ref{eq:d_red}) is not the only one allowed. Indeed, two other solutions are allowed:
\begin{equation}
	{-p_{3}} = d_{\blue{1}} - d_{\blue{2}}, \quad p_{1} = d_{\blue{2}} - d_{\blue{3}}, \quad p_{4} = d_{\blue{4}} - d_{\blue{3}}, \quad -p_{2} = d_{\blue{1}} - d_{\blue{4}},
	\label{eq:d_blue}
\end{equation}
and
\begin{equation}
	p_{2} = d_{\green{1}} - d_{\green{2}}, \quad -p_{3} = d_{\green{2}} - d_{\green{3}}, \quad p_{4} = d_{\green{4}} - d_{\green{3}}, \quad -p_{1} = d_{\green{1}} - d_{\green{4}}.
	\label{eq:d_green}
\end{equation}
I will refer to (\ref{eq:d_red}) as the \red{red} planar class, (\ref{eq:d_blue}) as the \blue{blue} planar class, and (\ref{eq:d_green}) as the \green{green} planar class. In each planar class, the dual spacetime coordinates describe the positions of four points in the dual spacetime. These four points can be taken as the vertices of a (Minkowski) tetrahedron. Many kinematic quantities can be understood in terms of the geometry of these (Minkowski) tetrahedra.
%%%%%%%%%%%%%%%%%%%%%%%%%%%%%%%%%%%%%%%%%%%%%%%%%%%%%%%%%%%%%%%%%%%%%%%%%%%%%%%%%%%%%%%%%%%%%%%%%%%%%%%%%%%%%%%%%%%%
\section{Simplicial Invariants}
%%%%%%%%%%%%%%%%%%%%%%%%%%%%%%%%%%%%%%%%%%%%%%%%%%%%%%%%%%%%%%%%%%%%%%%%%%%%%%%%%%%%%%%%%%%%%%%%%%%%%%%%%%%%%%%%%%%%
A tetrahedron is a 3-simplex. As such, it contains four vertices (0-simplex), six edges (1-simplex), four triangular faces (2-simplex), and one tetrahedron (3-simplex). The $n$-volume of an $n$-simplex is found by evaluating an $(n+1)$-point Cayley-Menger determinant. These are useful kinematic invariants.
%%%%%%%%%%%%%%%%%%%%%%%%%%%%%%%%%%%%%%%%%%%%%%%%%%%%%%%%%%%%%%%%%%%%%%%%%%%%%%%%%%%%%%%%%%%%%%%%%%%%%%%%%%%%%%%%%%%%
\subsection{1-Simplex Invariants}
%%%%%%%%%%%%%%%%%%%%%%%%%%%%%%%%%%%%%%%%%%%%%%%%%%%%%%%%%%%%%%%%%%%%%%%%%%%%%%%%%%%%%%%%%%%%%%%%%%%%%%%%%%%%%%%%%%%%
Given two dual spacetime positions, the 2-point Cayley-Menger determinant is given by:
\begin{equation}
	C_{IJ} = -\frac{1}{2} \det{
	\begin{pmatrix}
	0 & \abs{d_{IJ}}^{2} & 1 \\
	\abs{d_{IJ}}^{2} & 0 & 1 \\
	1 & 1 & 0
	\end{pmatrix}
	}, \qquad d_{IJ} \equiv d_{I} - d_{J}.
\end{equation}
Since a tetrahedron has six edges, there are six possible 1-simplex invariants for each tetrahedron. These invariants will be either squared masses or Mandelstam invariants.
%%%%%%%%%%%%%%%%%%%%%%%%%%%%%%%%%%%%%%%%%%%%%%%%%%%%%%%%%%%%%%%%%%%%%%%%%%%%%%%%%%%%%%%%%%%%%%%%%%%%%%%%%%%%%%%%%%%%
\subsection{2-Simplex Invariants}
%%%%%%%%%%%%%%%%%%%%%%%%%%%%%%%%%%%%%%%%%%%%%%%%%%%%%%%%%%%%%%%%%%%%%%%%%%%%%%%%%%%%%%%%%%%%%%%%%%%%%%%%%%%%%%%%%%%%
Given three dual spacetime positions, the 3-point Cayley-Menger determinant is given by:
\begin{equation}
	C_{IJK} = \det{
	\begin{pmatrix}
	0 & \abs{d_{IJ}}^{2} & \abs{d_{IK}}^{2} & 1 \\
	\abs{d_{IJ}}^{2} & 0 & \abs{d_{JK}}^{2} & 1 \\
	\abs{d_{IK}}^{2} & \abs{d_{JK}}^{2} & 0 & 1 \\
	1 & 1 & 1 & 0
	\end{pmatrix}
	}, \qquad d_{IJ} \equiv d_{I} - d_{J}.
\end{equation}
Since a tetrahedron has four triangular faces, there are four possible 2-simplex invariants for each tetrahedron. These invariants take the form of K\"{a}ll\'{e}n functions.
%%%%%%%%%%%%%%%%%%%%%%%%%%%%%%%%%%%%%%%%%%%%%%%%%%%%%%%%%%%%%%%%%%%%%%%%%%%%%%%%%%%%%%%%%%%%%%%%%%%%%%%%%%%%%%%%%%%%
\subsubsection{Mixed K\"{a}ll\'{e}n Function}
%%%%%%%%%%%%%%%%%%%%%%%%%%%%%%%%%%%%%%%%%%%%%%%%%%%%%%%%%%%%%%%%%%%%%%%%%%%%%%%%%%%%%%%%%%%%%%%%%%%%%%%%%%%%%%%%%%%%
The mixed K\"{a}ll\'{e}n function is
\begin{equation}
	\Omega_{ij}(x) = \left[ \left(x - m_{i}^{2} + w_{j}^{2} \right)^{2} + 4 m_{i}^{2} w_{j}^{2} \right],
\end{equation}
where $x$ is a Mandelstam invariant. Note that this function is always positive for $x$ real.
%%%%%%%%%%%%%%%%%%%%%%%%%%%%%%%%%%%%%%%%%%%%%%%%%%%%%%%%%%%%%%%%%%%%%%%%%%%%%%%%%%%%%%%%%%%%%%%%%%%%%%%%%%%%%%%%%%%%
\subsection{3-Simplex Invariants}
%%%%%%%%%%%%%%%%%%%%%%%%%%%%%%%%%%%%%%%%%%%%%%%%%%%%%%%%%%%%%%%%%%%%%%%%%%%%%%%%%%%%%%%%%%%%%%%%%%%%%%%%%%%%%%%%%%%%
Given four dual spacetime positions, the 4-point Cayley-Menger determinant is given by:
\begin{equation}
	C_{IJKL} = -\frac{1}{2} \det{
	\begin{pmatrix}
	0 & \abs{d_{IJ}}^{2} & \abs{d_{IK}}^{2} & \abs{d_{IL}}^{2} & 1 \\
	\abs{d_{IJ}}^{2} & 0 & \abs{d_{JK}}^{2} & \abs{d_{JL}}^{2} & 1 \\
	\abs{d_{IK}}^{2} & \abs{d_{JK}}^{2} & 0 & \abs{d_{KL}}^{2} & 1 \\
	\abs{d_{IL}}^{2} & \abs{d_{JL}}^{2} & \abs{d_{KL}}^{2} & 0 & 1 \\
	1 & 1 & 1 & 1 & 0
	\end{pmatrix}
	}, \qquad d_{IJ} \equiv d_{I} - d_{J}.
\end{equation}
A tetrahedron has only one possible 3-simplex invariant.
%%%%%%%%%%%%%%%%%%%%%%%%%%%%%%%%%%%%%%%%%%%%%%%%%%%%%%%%%%%%%%%%%%%%%%%%%%%%%%%%%%%%%%%%%%%%%%%%%%%%%%%%%%%%%%%%%%%%
\section{Dual Conformal Invariants}
%%%%%%%%%%%%%%%%%%%%%%%%%%%%%%%%%%%%%%%%%%%%%%%%%%%%%%%%%%%%%%%%%%%%%%%%%%%%%%%%%%%%%%%%%%%%%%%%%%%%%%%%%%%%%%%%%%%%
The expression $\abs{d_{I} - d_{J}}^{2}$ is not only Lorentz invariant, but dual Poincar\'{e} invariant. You can study dual conformal invariants by constructing a dual conformal ratio with a quartet of dual spacetime coordinates:
\begin{equation}
	\crat{I}{J}{K}{L} \equiv \frac{\abs{d_{IK}}^{2} \abs{d_{JL}}^{2}}{\abs{d_{IL}}^{2} \abs{d_{JK}}^{2}}, \qquad d_{IJ} \equiv d_{I} - d_{J}.
\end{equation}
In four-point scattering there is a unique quartet of dual spacetime coordinates. However, there are six inequivalent permutations of these coordinates, and thus, six possible values for the dual conformal ratio. Note that
\begin{equation}
	\crat{I}{J}{K}{L} \crat{I}{K}{L}{J} \crat{I}{L}{J}{K} = 1,
\end{equation}
which is analogous to the constraint satisfied by the three Mandelstam invariants.
%%%%%%%%%%%%%%%%%%%%%%%%%%%%%%%%%%%%%%%%%%%%%%%%%%%%%%%%%%%%%%%%%%%%%%%%%%%%%%%%%%%%%%%%%%%%%%%%%%%%%%%%%%%%%%%%%%%%
\section{Two-Body Sudakov Null Decomposition}
%%%%%%%%%%%%%%%%%%%%%%%%%%%%%%%%%%%%%%%%%%%%%%%%%%%%%%%%%%%%%%%%%%%%%%%%%%%%%%%%%%%%%%%%%%%%%%%%%%%%%%%%%%%%%%%%%%%%
The external energy-momentum vectors can be divided into an incoming set ($p_{1}$ and $p_{2}$), and an outgoing set ($p_{3}$ and $p_{4}$). In a two-body Sudakov decomposition one postulates the existence of two (null) Sudakov momenta $k_{3}$ and $k_{4}$ and writes the outgoing momenta as linear combinations of the Sudakov momenta:
\begin{equation}
	p_{3} = k_{3} + \frac{\abs{p_{3}}^{2}}{S_{34}} k_{4}, \qquad p_{4} = k_{4} + \frac{\abs{p_{4}}^{2}}{S_{34}} k_{3}; \qquad S_{34} \equiv -\abs{k_{3} + k_{4}}^{2}.
\end{equation}
Using $s = - \abs{p_{3} + p_{4}}^{2}$, you find that
\begin{equation}
	S_{34} = \frac{1}{2} \left[ \right].
\end{equation}

Each of the incoming momenta is decomposed into a part that is orthogonal to the subspace spanned by the two Sudakov momenta, and a part that is parallel to this subspace:
\begin{align}
	p_{1} &= P_{1} + c_{13} k_{3} + c_{14} k_{4}, \qquad P_{1} \cdot k_{3} = 0, \qquad P_{1} \cdot k_{4} = 0, \\
	p_{2} &= P_{2} + c_{23} k_{3} + c_{24} k_{4}, \qquad P_{2} \cdot k_{3} = 0, \qquad P_{2} \cdot k_{4} = 0.
\end{align}
Here $P_{1}$ and $P_{2}$ are said to be transversal. It follows that
\begin{align}
	c_{13} = \frac{ \abs{k_{4}}^{2} \left(p_{1} \cdot k_{3} \right) - \left(k_{3} \cdot k_{4} \right) \left(p_{1} \cdot k_{4} \right) }{\abs{k_{3}}^{2} \abs{k_{4}}^{2} - \left( k_{3} \cdot k_{4} \right)^{2}}, &\qquad
	c_{14} = \frac{ \abs{k_{3}}^{2} \left(p_{1} \cdot k_{4} \right) - \left(k_{3} \cdot k_{4} \right) \left(p_{1} \cdot k_{3} \right) }{\abs{k_{3}}^{2} \abs{k_{4}}^{2} - \left( k_{3} \cdot k_{4} \right)^{2}}, \\
	c_{23} = \frac{\abs{k_{4}}^{2} \left(p_{2} \cdot k_{3} \right) - \left(k_{3} \cdot k_{4} \right) \left(p_{2} \cdot k_{4} \right)}{\abs{k_{3}}^{2} \abs{k_{4}}^{2} - \left( k_{3} \cdot k_{4} \right)^{2}}, &\qquad
	c_{24} = \frac{ \abs{k_{3}}^{2} \left(p_{2} \cdot k_{4} \right) - \left(k_{3} \cdot k_{4} \right) \left(p_{2} \cdot k_{3} \right) }{\abs{k_{3}}^{2} \abs{k_{4}}^{2} - \left( k_{3} \cdot k_{4} \right)^{2}}.
\end{align}
Note that
\begin{equation}
	\abs{p_{1}}^{2} = \abs{P_{1}}^{2} + \abs{c_{13} k_{3} + c_{14} k_{4}}^{2}, \qquad \abs{p_{2}}^{2} = \abs{P_{2}}^{2} + \abs{c_{23} k_{3} + c_{24} k_{4}}^{2}.
\end{equation}
Thus,
\begin{align}
	\abs{P_{1}}^{2} &= \abs{p_{1}}^{2} - c_{13}^{2} \abs{k_{3}}^{2} - c_{14}^{2} \abs{k_{4}}^{2} - 2 c_{13} c_{14} \left( k_{3} \cdot k_{4} \right), \\
	\abs{P_{2}}^{2} &= \abs{p_{2}}^{2} - c_{23}^{2} \abs{k_{3}}^{2} - c_{24}^{2} \abs{k_{4}}^{2} - 2 c_{23} c_{24} \left( k_{3} \cdot k_{4} \right).
\end{align}
It is also useful to decompose the linear combinations of external momenta that yield Mandelstam invariants:
\begin{align}
	p_{1} + p_{2} &= P_{1} + P_{2} + \left(c_{13} + c_{23}\right) k_{3} + \left(c_{14} + c_{24}\right) k_{4}, \\
	p_{3} + p_{4} &= \left(1 + c_{43}\right) k_{3} + \left(c_{34} + 1\right) k_{4}, \\
	p_{1} - p_{3} &= P_{1} + \left(c_{13} - 1\right) k_{3} + \left(c_{14} - c_{34}\right) k_{4}, \\
	p_{4} - p_{2} &= -P_{2} + \left(c_{43} - c_{23}\right) k_{3} + \left(1 - c_{24}\right) k_{4}, \\
	p_{1} - p_{4} &= P_{1} + \left(c_{13} - c_{43}\right) k_{3} + \left(c_{14} - 1\right) k_{4}, \\
	p_{3} - p_{2} &= -P_{2} + \left(1 - c_{23}\right) k_{3} + \left(c_{34} - c_{24}\right) k_{4}.
\end{align}
From the conservation constraint, you find
\begin{equation}
	p_{1} + p_{2} = p_{3} + p_{4} \quad \Longrightarrow \quad c_{13} + c_{23} = 1 + c_{43}, \qquad c_{14} + c_{24} = c_{34} + 1.
\end{equation}
It also follows that $P_{1} + P_{2} = 0$.
%%%%%%%%%%%%%%%%%%%%%%%%%%%%%%%%%%%%%%%%%%%%%%%%%%%%%%%%%%%%%%%%%%%%%%%%%%%%%%%%%%%%%%%%%%%%%%%%%%%%%%%%%%%%%%%%%%%%
\section{Null Sudakov Basis}
%%%%%%%%%%%%%%%%%%%%%%%%%%%%%%%%%%%%%%%%%%%%%%%%%%%%%%%%%%%%%%%%%%%%%%%%%%%%%%%%%%%%%%%%%%%%%%%%%%%%%%%%%%%%%%%%%%%%
Instead of using Sudakov momenta to study transversality of the incoming momenta, you can make a full change of basis to null Sudakov momenta:
\begin{equation}
	p_{1} = k_{1} + c_{12} k_{2}, \qquad p_{2} = k_{2} + c_{21} k_{1}, \qquad p_{3} = k_{3} + c_{34} k_{4}, \qquad p_{4} = k_{4} + c_{43} k_{3};
\end{equation}
with
\begin{equation}
	\abs{k_{1}}^{2} = \abs{k_{2}}^{2} = \abs{k_{3}}^{2} = \abs{k_{4}}^{2} = 0.
\end{equation}
There are ten unknowns, and ten relations:
\begin{align}
	\abs{p_{1}}^{2} &= 2 c_{12} \left( k_{1} \cdot k_{2} \right), \\
	\left( p_{1} \cdot p_{2} \right) &= \left(1 + c_{12}c_{21} \right) \left( k_{1} \cdot k_{2} \right), \\
	\left( p_{1} \cdot p_{3} \right) &= \left( k_{1} \cdot k_{3} \right) + c_{34} \left( k_{1} \cdot k_{4} \right) + c_{12} \left( k_{2} \cdot k_{3} \right) + c_{12} c_{34} \left( k_{2} \cdot k_{4} \right), \\
	\left( p_{1} \cdot p_{4} \right) &= \left( k_{1} \cdot k_{4} \right) + c_{43} \left( k_{1} \cdot k_{3} \right) + c_{12} \left( k_{2} \cdot k_{4} \right) + c_{12} c_{43} \left( k_{2} \cdot k_{3} \right), \\
	\abs{p_{2}}^{2} &= 2 c_{21} \left( k_{1} \cdot k_{2} \right), \\
	\left( p_{2} \cdot p_{3} \right) &= \left( k_{2} \cdot k_{3} \right) + c_{34} \left( k_{2} \cdot k_{4} \right) + c_{21} \left( k_{1} \cdot k_{3} \right) + c_{21} c_{34} \left( k_{1} \cdot k_{4} \right), \\
	\left( p_{2} \cdot p_{4} \right) &= \left( k_{2} \cdot k_{4} \right) + c_{43} \left( k_{2} \cdot k_{3} \right) + c_{21} \left( k_{1} \cdot k_{4} \right) + c_{21} c_{43} \left( k_{1} \cdot k_{3} \right), \\
	\abs{p_{3}}^{2} &= 2 c_{34} \left( k_{3} \cdot k_{4} \right), \\
	\left( p_{3} \cdot p_{4} \right) &= \left(1 + c_{34}c_{43} \right) \left( k_{3} \cdot k_{4} \right), \\
	\abs{p_{4}}^{2} &= 2 c_{43} \left( k_{3} \cdot k_{4} \right).
\end{align}
You find two solutions for $c_{12}$ and $c_{21}$:
\begin{equation}
\begin{split}
	c_{12} &= -\frac{s + \abs{p_{1}}^{2} + \abs{p_{2}}^{2} \pm \sqrt{\left( s + \abs{p_{1}}^{2} + \abs{p_{2}}^{2} \right)^{2} - 4 \abs{p_{1}}^{2} \abs{p_{2}}^{2}}}{2\abs{p_{2}}^{2}}, \\
	c_{21} &= -\frac{s + \abs{p_{1}}^{2} + \abs{p_{2}}^{2} \pm \sqrt{\left( s + \abs{p_{1}}^{2} + \abs{p_{2}}^{2} \right)^{2} - 4 \abs{p_{1}}^{2} \abs{p_{2}}^{2}}}{2\abs{p_{1}}^{2}};
\end{split}
\end{equation}
two solutions for $c_{34}$ and $c_{43}$:
\begin{equation}
\begin{split}
	c_{34} &= -\frac{s + \abs{p_{3}}^{2} + \abs{p_{4}}^{2} \pm \sqrt{\left( s + \abs{p_{3}}^{2} + \abs{p_{4}}^{2} \right)^{2} - 4 \abs{p_{3}}^{2} \abs{p_{4}}^{2}}}{2\abs{p_{4}}^{2}}, \\
	c_{43} &= -\frac{s + \abs{p_{3}}^{2} + \abs{p_{4}}^{2} \pm \sqrt{\left( s + \abs{p_{3}}^{2} + \abs{p_{4}}^{2} \right)^{2} - 4 \abs{p_{3}}^{2} \abs{p_{4}}^{2}}}{2\abs{p_{3}}^{2}};
\end{split}
\end{equation}
and the rest,
\begin{align}
	\left( k_{1} \cdot k_{2} \right) &= \frac{\left( p_{1} \cdot p_{2} \right)}{1 + c_{12} c_{21}}, \\
	\left( k_{1} \cdot k_{3} \right) &= \frac{ \left( p_{1} \cdot p_{3} \right) - c_{12} \left( p_{2} \cdot p_{3} \right) - c_{34} \left( p_{1} \cdot p_{4} \right) + c_{12} c_{34} \left( p_{2} \cdot p_{4} \right) }{\left( 1 - c_{12} c_{21} \right) \left( 1 - c_{34} c_{43} \right)}, \\
	\left( k_{1} \cdot k_{4} \right) &= \frac{ \left( p_{1} \cdot p_{4} \right) - c_{12} \left( p_{2} \cdot p_{4} \right) - c_{43} \left( p_{1} \cdot p_{3} \right) + c_{12} c_{43} \left( p_{2} \cdot p_{3} \right) }{\left( 1 - c_{12} c_{21} \right) \left( 1 - c_{34} c_{43} \right)}, \\
	\left( k_{2} \cdot k_{3} \right) &= \frac{ \left( p_{2} \cdot p_{3} \right) - c_{21} \left( p_{1} \cdot p_{3} \right) - c_{34} \left( p_{2} \cdot p_{4} \right) + c_{21} c_{34} \left( p_{1} \cdot p_{4} \right) }{\left( 1 - c_{12} c_{21} \right) \left( 1 - c_{34} c_{43} \right)}, \\
	\left( k_{2} \cdot k_{4} \right) &= \frac{ \left( p_{2} \cdot p_{4} \right) - c_{21} \left( p_{1} \cdot p_{4} \right) - c_{43} \left( p_{2} \cdot p_{3} \right) + c_{21} c_{43} \left( p_{1} \cdot p_{3} \right) }{\left( 1 - c_{12} c_{21} \right) \left( 1 - c_{34} c_{43} \right)}, \\
	\left( k_{3} \cdot k_{4} \right) &= \frac{\left( p_{3} \cdot p_{4} \right)}{1 + c_{34} c_{43}}.
\end{align}