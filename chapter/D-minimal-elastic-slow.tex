\chapter{Minimal Elastic Slow Diversity}
%%%%%%%%%%%%%%%%%%%%%%%%%%%%%%%%%%%%%%%%%%%%%%%%%%%%%%%%%%%%%%%%%%%%%%%%%%%%%%%%%%%%%%%%%%%%%%%%%%%%%%%%%%%%%%%%%%%%
In this chapter I will consider the kinematics of a 2-to-2 scattering process that has minimal elastic slow diversity. This means that all external quanta are slow, but you have only one distinct mass:
\begin{equation}
	m^{2} = -\abs{p_{1}}^{2} = -\abs{p_{2}}^{2} = -\abs{p_{3}}^{2} = -\abs{p_{4}}^{2}.
\end{equation}
Thus, this is an elastic process. For example,
\begin{equation}
	A(p_{1}) + A(p_{2}) \longrightarrow A(p_{3}) + A(p_{4}).
\end{equation}
For this process, (\ref{eq:stu_slow}) becomes
\begin{equation}
	s + t + u = 4m^{2}.
\end{equation}
%%%%%%%%%%%%%%%%%%%%%%%%%%%%%%%%%%%%%%%%%%%%%%%%%%%%%%%%%%%%%%%%%%%%%%%%%%%%%%%%%%%%%%%%%%%%%%%%%%%%%%%%%%%%%%%%%%%%
\section{Simplicial Invariants}
%%%%%%%%%%%%%%%%%%%%%%%%%%%%%%%%%%%%%%%%%%%%%%%%%%%%%%%%%%%%%%%%%%%%%%%%%%%%%%%%%%%%%%%%%%%%%%%%%%%%%%%%%%%%%%%%%%%%
With elastic minimal slow diversity you have even less distinct values.
%%%%%%%%%%%%%%%%%%%%%%%%%%%%%%%%%%%%%%%%%%%%%%%%%%%%%%%%%%%%%%%%%%%%%%%%%%%%%%%%%%%%%%%%%%%%%%%%%%%%%%%%%%%%%%%%%%%%
\subsection{1-Simplex Invariants}
%%%%%%%%%%%%%%%%%%%%%%%%%%%%%%%%%%%%%%%%%%%%%%%%%%%%%%%%%%%%%%%%%%%%%%%%%%%%%%%%%%%%%%%%%%%%%%%%%%%%%%%%%%%%%%%%%%%%
For the \red{red} tetrahedron you have
\begin{equation}
	C_{\red{12}} = C_{\red{14}} = C_{\red{23}} = C_{\red{34}} = m^{2}, \qquad C_{\red{13}} = s, \qquad C_{\red{24}} = t.
\end{equation}
Similarly, for the \blue{blue} tetrahedron you have
\begin{equation}
	C_{\blue{12}} = C_{\blue{14}} = C_{\blue{23}} = C_{\blue{34}} = m^{2}, \qquad C_{\blue{13}} = t, \qquad C_{\blue{24}} = u.
\end{equation}
Finally, for the \green{green} tetrahedron you have
\begin{equation}
	C_{\green{12}} = C_{\green{14}} = C_{\green{23}} = C_{\green{34}} = m^{2}, \qquad C_{\green{13}} = u, \qquad C_{\green{24}} = s.
\end{equation}
%%%%%%%%%%%%%%%%%%%%%%%%%%%%%%%%%%%%%%%%%%%%%%%%%%%%%%%%%%%%%%%%%%%%%%%%%%%%%%%%%%%%%%%%%%%%%%%%%%%%%%%%%%%%%%%%%%%%
\subsection{2-Simplex Invariants}
%%%%%%%%%%%%%%%%%%%%%%%%%%%%%%%%%%%%%%%%%%%%%%%%%%%%%%%%%%%%%%%%%%%%%%%%%%%%%%%%%%%%%%%%%%%%%%%%%%%%%%%%%%%%%%%%%%%%
For the \red{red} tetrahedron you have
\begin{equation}
	C_{\red{123}} = C_{\red{134}} = s \left[s - 4 m^{2} \right] = \Lambda_{12}(s), \qquad C_{\red{124}} = C_{\red{234}} = t \left[t - 4 m^{2} \right] = \Lambda_{13}(t).
\end{equation}
Similarly, for the \blue{blue} tetrahedron you have
\begin{equation}
	C_{\blue{123}} = C_{\blue{134}} = t \left[t - 4 m^{2} \right] = \Lambda_{13}(t), \qquad C_{\blue{124}} = C_{\blue{234}} = u \left[u - 4 m^{2} \right] = \Lambda_{14}(u).
\end{equation}
Finally, for the \green{green} tetrahedron you have
\begin{equation}
	C_{\green{123}} = C_{\green{134}} = u \left[u - 4 m^{2} \right] = \Lambda_{14}(u), \qquad C_{\green{124}} = C_{\green{234}} = s \left[s - 4 m^{2} \right] = \Lambda_{12}(s).
\end{equation}
%%%%%%%%%%%%%%%%%%%%%%%%%%%%%%%%%%%%%%%%%%%%%%%%%%%%%%%%%%%%%%%%%%%%%%%%%%%%%%%%%%%%%%%%%%%%%%%%%%%%%%%%%%%%%%%%%%%%
\subsection{3-Simplex Invariants}
%%%%%%%%%%%%%%%%%%%%%%%%%%%%%%%%%%%%%%%%%%%%%%%%%%%%%%%%%%%%%%%%%%%%%%%%%%%%%%%%%%%%%%%%%%%%%%%%%%%%%%%%%%%%%%%%%%%%
A tetrahedron has only one possible 3-simplex invariant. Indeed,
\begin{equation}
	C_{\red{1234}} = C_{\blue{1234}} = C_{\green{1234}} = stu.
\end{equation}
%%%%%%%%%%%%%%%%%%%%%%%%%%%%%%%%%%%%%%%%%%%%%%%%%%%%%%%%%%%%%%%%%%%%%%%%%%%%%%%%%%%%%%%%%%%%%%%%%%%%%%%%%%%%%%%%%%%%
\section{Dual Conformal Invariants}
%%%%%%%%%%%%%%%%%%%%%%%%%%%%%%%%%%%%%%%%%%%%%%%%%%%%%%%%%%%%%%%%%%%%%%%%%%%%%%%%%%%%%%%%%%%%%%%%%%%%%%%%%%%%%%%%%%%%
For the \red{red} tetrahedron, you have
\begin{equation}
\begin{split}
	\crat{\red{1}}{\red{2}}{\red{3}}{\red{4}} = \frac{s t}{m^{4}}, \qquad
	\crat{\red{1}}{\red{3}}{\red{4}}{\red{2}} &= 1, \qquad
	\crat{\red{1}}{\red{4}}{\red{2}}{\red{3}} = \frac{m^{4}}{s t}, \\
	\crat{\red{1}}{\red{2}}{\red{4}}{\red{3}} = \frac{m^{4}}{s t}, \qquad
	\crat{\red{1}}{\red{3}}{\red{2}}{\red{4}} &= 1, \qquad
	\crat{\red{1}}{\red{4}}{\red{3}}{\red{2}} = \frac{s t}{m^{4}}.
\end{split}
\end{equation}
Similarly, for the \blue{blue} tetrahedron, you have
\begin{equation}
\begin{split}
	\crat{\blue{1}}{\blue{2}}{\blue{3}}{\blue{4}} = \frac{t u}{m^{4}}, \qquad
	\crat{\blue{1}}{\blue{3}}{\blue{4}}{\blue{2}} &= 1, \qquad
	\crat{\blue{1}}{\blue{4}}{\blue{2}}{\blue{3}} = \frac{m^{4}}{t u}, \\
	\crat{\blue{1}}{\blue{2}}{\blue{4}}{\blue{3}} = \frac{m^{4}}{t u}, \qquad
	\crat{\blue{1}}{\blue{3}}{\blue{2}}{\blue{4}} &= 1, \qquad
	\crat{\blue{1}}{\blue{4}}{\blue{3}}{\blue{2}} = \frac{t u}{m^{4}}.
\end{split}
\end{equation}
Finally, for the \green{green} tetrahedron, you have
\begin{equation}
\begin{split}
	\crat{\green{1}}{\green{2}}{\green{3}}{\green{4}} = \frac{s u}{m^{4}}, \qquad
	\crat{\green{1}}{\green{3}}{\green{4}}{\green{2}} &= 1, \qquad
	\crat{\green{1}}{\green{4}}{\green{2}}{\green{3}} = \frac{m^{4}}{s u}, \\
	\crat{\green{1}}{\green{2}}{\green{4}}{\green{3}} = \frac{m^{4}}{s u}, \qquad
	\crat{\green{1}}{\green{3}}{\green{2}}{\green{4}} &= 1, \qquad
	\crat{\green{1}}{\green{4}}{\green{3}}{\green{2}} = \frac{s u}{m^{4}}.
\end{split}
\end{equation}
%%%%%%%%%%%%%%%%%%%%%%%%%%%%%%%%%%%%%%%%%%%%%%%%%%%%%%%%%%%%%%%%%%%%%%%%%%%%%%%%%%%%%%%%%%%%%%%%%%%%%%%%%%%%%%%%%%%%
\section{Center-of-Momentum Frame}
%%%%%%%%%%%%%%%%%%%%%%%%%%%%%%%%%%%%%%%%%%%%%%%%%%%%%%%%%%%%%%%%%%%%%%%%%%%%%%%%%%%%%%%%%%%%%%%%%%%%%%%%%%%%%%%%%%%%
In the center-of-momentum frame you write the energy-momentum vectors as
\begin{equation}
	p_{1} = \begin{pmatrix} E_{1} & \mathbf{p}_{1} \end{pmatrix}, \qquad p_{2} = \begin{pmatrix} E_{2} & -\mathbf{p}_{1} \end{pmatrix}, \qquad p_{3} = \begin{pmatrix} E_{3} & \mathbf{p}_{3} \end{pmatrix}, \qquad p_{4} = \begin{pmatrix} E_{4} & -\mathbf{p}_{3} \end{pmatrix}.
\end{equation}
One of the first things to notice is that
\begin{equation}
	s = (E_{1} + E_{2})^{2} = (E_{3} + E_{4})^{2}.
\end{equation}
Thus, $s$ can be interpreted as the total energy in the center-of-momentum frame. It also follows that $s$ must be positive.

From the on-shell constraints, it follows that
\begin{equation}
\begin{split}
	E_{1} = \sqrt{m^{2} + \abs{\mathbf{p}_{1}}^{2}}, &\qquad E_{2} = \sqrt{m^{2} + \abs{\mathbf{p}_{1}}^{2}}, \\
	E_{3} = \sqrt{m^{2} + \abs{\mathbf{p}_{3}}^{2}}, &\qquad E_{4} = \sqrt{m^{2} + \abs{\mathbf{p}_{3}}^{2}}.
\end{split}
\end{equation}
Using the relations
\begin{equation}
	E_{2} = \sqrt{s} - E_{1}, \qquad E_{4} = \sqrt{s} - E_{3},
\end{equation}
you find that
\begin{equation}
	\abs{\mathbf{p}_{1}} = \abs{\mathbf{p}_{3}} = \frac{\sqrt{s - 4m^{2}}}{2}.
\end{equation}
Thus,
\begin{equation}
	E_{1} = E_{2} = E_{3} = E_{4} = \frac{\sqrt{s}}{2}.
\end{equation}
%%%%%%%%%%%%%%%%%%%%%%%%%%%%%%%%%%%%%%%%%%%%%%%%%%%%%%%%%%%%%%%%%%%%%%%%%%%%%%%%%%%%%%%%%%%%%%%%%%%%%%%%%%%%%%%%%%%%
\subsection{Speed and Rapidity}
%%%%%%%%%%%%%%%%%%%%%%%%%%%%%%%%%%%%%%%%%%%%%%%%%%%%%%%%%%%%%%%%%%%%%%%%%%%%%%%%%%%%%%%%%%%%%%%%%%%%%%%%%%%%%%%%%%%%
The speed of each external quantum is
\begin{equation}
	\abs{\mathbf{v}_{1}} = \abs{\mathbf{v}_{2}} = \abs{\mathbf{v}_{3}} = \abs{\mathbf{v}_{4}} = \sqrt{1 - \frac{4 m^{2}}{s}}.
\end{equation}
The (slow) rapidity of each external quantum is
\begin{equation}
	\eta_{1} = \eta_{2} = \eta_{3} = \eta_{4} = \frac{1}{2} \log{\left[\frac{\sqrt{s} + \sqrt{s - 4 m^{2}}}{\sqrt{s} - \sqrt{s - 4 m^{2}}}\right]}.
\end{equation}
The sum of the incoming rapidites is equal to the sum of the outgoing rapidities:
\begin{equation}
	\eta_{1} + \eta_{2} = \eta_{3} + \eta_{4} = \log{\left[\frac{\sqrt{s} + \sqrt{s - 4 m^{2}}}{\sqrt{s} - \sqrt{s - 4 m^{2}}}\right]}.
\end{equation}
%%%%%%%%%%%%%%%%%%%%%%%%%%%%%%%%%%%%%%%%%%%%%%%%%%%%%%%%%%%%%%%%%%%%%%%%%%%%%%%%%%%%%%%%%%%%%%%%%%%%%%%%%%%%%%%%%%%%
\subsection{Physical Scattering Region}
%%%%%%%%%%%%%%%%%%%%%%%%%%%%%%%%%%%%%%%%%%%%%%%%%%%%%%%%%%%%%%%%%%%%%%%%%%%%%%%%%%%%%%%%%%%%%%%%%%%%%%%%%%%%%%%%%%%%
For this particular process, the physical scattering region is
\begin{equation}
	s > 4m^2, \qquad t < 0, \qquad u < 0.
\end{equation}
%%%%%%%%%%%%%%%%%%%%%%%%%%%%%%%%%%%%%%%%%%%%%%%%%%%%%%%%%%%%%%%%%%%%%%%%%%%%%%%%%%%%%%%%%%%%%%%%%%%%%%%%%%%%%%%%%%%%
\subsection{Sudakov Slow Decomposition}
%%%%%%%%%%%%%%%%%%%%%%%%%%%%%%%%%%%%%%%%%%%%%%%%%%%%%%%%%%%%%%%%%%%%%%%%%%%%%%%%%%%%%%%%%%%%%%%%%%%%%%%%%%%%%%%%%%%%
...
%%%%%%%%%%%%%%%%%%%%%%%%%%%%%%%%%%%%%%%%%%%%%%%%%%%%%%%%%%%%%%%%%%%%%%%%%%%%%%%%%%%%%%%%%%%%%%%%%%%%%%%%%%%%%%%%%%%%
\subsection{Sudakov Null Decomposition}
%%%%%%%%%%%%%%%%%%%%%%%%%%%%%%%%%%%%%%%%%%%%%%%%%%%%%%%%%%%%%%%%%%%%%%%%%%%%%%%%%%%%%%%%%%%%%%%%%%%%%%%%%%%%%%%%%%%%
Let $k_{3}$ and $k_{4}$ be null momenta related to the outgoing momenta via the transformation:
\begin{equation}
	p_{3} = k_{3} + b_{3} k_{4}, \qquad p_{4} = k_{4} + a_{4} k_{3}.
	\label{eq:sudakov_3_4}
\end{equation}
Then,
\begin{align}
	m^{2} = - \abs{p_{3}}^{2} \quad &\Longrightarrow \quad m^{2} = - 2 b_{3} \left( k_{3} \cdot k_{4} \right), \\
	m^{2} = - \abs{p_{4}}^{2} \quad &\Longrightarrow \quad m^{2} = - 2 a_{4} \left( k_{3} \cdot k_{4} \right), \\
	s = - \abs{p_{3} + p_{4}}^{2} \quad &\Longrightarrow \quad s = - 2 \left(1 + b_{3} \right) \left(1 + a_{4} \right) \left( k_{3} \cdot k_{4} \right).
\end{align}
Solving these equations yield
\begin{equation}
	b_{3} = a_{4} = \frac{2 m^{2}}{s - 2m^{2} + \sqrt{\Lambda_{12}(s)}};
\end{equation}
and
\begin{equation}
	k_{3} \cdot k_{4} = -\frac{1}{4} \left( s - 2m^{2} + \sqrt{\Lambda_{12}(s)} \right).
\end{equation}
You have
\begin{equation}
	b_{3} a_{4} = \frac{s - 2m^{2} - \sqrt{\Lambda_{12}(s)}}{s - 2m^{2} + \sqrt{\Lambda_{12}(s)}},
\end{equation}
\begin{equation}
	1 + b_{3} a_{4} = \frac{ 2 \left( s - 2m^{2} \right)}{s - 2m^{2} + \sqrt{\Lambda_{12}(s)}}, \qquad
	1 - b_{3} a_{4} = \frac{ 2 \sqrt{\Lambda_{12}(s)}}{s - 2m^{2} + \sqrt{\Lambda_{12}(s)}};
\end{equation}
and
\begin{equation}
	b_{3} + a_{4} = \frac{ 4m^{2} }{s - 2m^{2} + \sqrt{\Lambda_{12}(s)}}, \qquad
	b_{3} - a_{4} = 0.
\end{equation}
Also
\begin{align}
	1 + b_{3} &= \frac{s + \sqrt{\Lambda_{12}(s)}}{s - 2m^{2} + \sqrt{\Lambda_{12}(s)}} = \frac{s - \sqrt{\Lambda_{12}(s)}}{2m^{2}}, \\
	1 + a_{4} &= \frac{s + \sqrt{\Lambda_{12}(s)}}{s - 2m^{2} + \sqrt{\Lambda_{12}(s)}} = \frac{s - \sqrt{\Lambda_{12}(s)}}{2m^{2}};
\end{align}
and
\begin{equation}
	1 - b_{3} = \frac{s - 4m^{2} + \sqrt{\Lambda_{12}(s)}}{s - 2m^{2} + \sqrt{\Lambda_{12}(s)}}, \qquad
	1 - a_{4} = \frac{s - 4m^{2} + \sqrt{\Lambda_{12}(s)}}{s - 2m^{2} + \sqrt{\Lambda_{12}(s)}}.
\end{equation}
Thus
\begin{align}
	\left( 1 + b_{3} \right) \left( 1 + a_{4} \right) &= \frac{2s}{s - 2m^{2} + \sqrt{\Lambda_{12}(s)}}, \\
	\left( 1 - b_{3} \right) \left( 1 + a_{4} \right) &= \frac{ 2 \sqrt{\Lambda_{12}(s)} }{s - 2m^{2} + \sqrt{\Lambda_{12}(s)}}, \\
	\left( 1 + b_{3} \right) \left( 1 - a_{4} \right) &= \frac{ 2 \sqrt{\Lambda_{12}(s)} }{s - 2m^{2} + \sqrt{\Lambda_{12}(s)}}, \\
	\left( 1 - b_{3} \right) \left( 1 - a_{4} \right) &= \frac{ 2 \left( s - 4m^{2} \right)}{s - 2m^{2} + \sqrt{\Lambda_{12}(s)}}.
\end{align}

In terms of $a_{4}$ and $b_{3}$, the outgoing speeds in the center-of-momentum frame are
\begin{equation}
	\abs{\mathbf{v}_{3}} = \abs{\mathbf{v}_{4}} = \frac{1 - b_{3} a_{4}}{1 + 2 a_{4} + b_{3} a_{4}} = \frac{\sqrt{\Lambda_{12}(s)}}{s}.
\end{equation}
Other relevant results are
\begin{equation}
	\frac{1}{b_{3}} = \frac{1}{a_{4}} = \frac{s + \sqrt{\Lambda_{12}(s)}}{s - \sqrt{\Lambda_{12}(s)}};
\end{equation}
and
\begin{equation}
	\frac{1}{b_{3}a_{4}} = \frac{s - 2m^{2} + \sqrt{\Lambda_{12}(s)}}{s - 2m^{2} - \sqrt{\Lambda_{12}(s)}}.
\end{equation}
The logarithm of these expressions are related to rapidities in the center-of-momentum frame.

The inverse of (\ref{eq:sudakov_3_4}) is
\begin{equation}
	k_{3} = \frac{p_{3} - b_{3} p_{4}}{1 - b_{3} a_{4}}, \qquad k_{4} = \frac{p_{4} - a_{4} p_{3}}{1 - b_{3} a_{4}}.
\end{equation}
Thus,
\begin{align}
	p_{1} \cdot k_{3} &= \frac{p_{1} \cdot p_{3} - b_{3} \left( p_{1} \cdot p_{4} \right)}{1 - b_{3} a_{4}} = \frac{ t - 2m^{2} - b_{3} \left( u - 2m^{2} \right) }{2 \left(1 - b_{3} a_{4} \right)}, \\
	p_{2} \cdot k_{3} &= \frac{p_{2} \cdot p_{3} - b_{3} \left( p_{2} \cdot p_{4} \right)}{1 - b_{3} a_{4}} = \frac{ u - 2m^{2} - b_{3} \left( t - 2m^{2} \right) }{2 \left(1 - b_{3} a_{4} \right)}, \\
	p_{3} \cdot k_{3} &= \frac{\abs{p_{3}}^{2} - b_{3} \left( p_{3} \cdot p_{4} \right)}{1 - b_{3} a_{4}} = \frac{ b_{3} \left( s - 2m^{2} \right) - 2m^{2} }{2 \left(1 - b_{3} a_{4} \right)}, \\
	p_{4} \cdot k_{3} &= \frac{p_{3} \cdot p_{4} - b_{3} \abs{p_{4}}^{2}}{1 - b_{3} a_{4}} = \frac{ 2m^{2} - s + 2m^{2} b_{3}}{2 \left(1 - b_{3} a_{4} \right)};
\end{align}
and
\begin{align}
	p_{1} \cdot k_{4} &= \frac{p_{1} \cdot p_{4} - a_{4} \left( p_{1} \cdot p_{3} \right)}{1 - b_{3} a_{4}} = \frac{u - 2m^{2} - a_{4} \left(t - 2m^{2} \right)}{2 \left( 1 - b_{3} a_{4} \right)}, \\
	p_{2} \cdot k_{4} &= \frac{p_{2} \cdot p_{4} - a_{4} \left( p_{2} \cdot p_{3} \right)}{1 - b_{3} a_{4}} = \frac{t - 2m^{2} - a_{4} \left( u - 2m^{2} \right)}{2 \left( 1 - b_{3} a_{4} \right)}, \\
	p_{3} \cdot k_{4} &= \frac{p_{3} \cdot p_{4} - a_{4} \abs{p_{3}}^{2}}{1 - b_{3} a_{4}} = \frac{2m^{2} - s + 2m^{2} a_{4}}{2 \left( 1 - b_{3} a_{4} \right)}, \\
	p_{4} \cdot k_{4} &= \frac{ \abs{p_{4}}^{2} - a_{4} \left( p_{3} \cdot p_{4} \right)}{1 - b_{3} a_{4}} = \frac{a_{4} \left( s - 2m^{2} \right) - 2m^{2}}{2 \left( 1 - b_{3} a_{4} \right)}.
\end{align}
Given a spacetime vector $v$, the Sudakov null decomposition is given by
\begin{equation}
	v = V + a_{v} k_{3} + b_{v} k_{4}, \qquad V \cdot k_{3} = V \cdot k_{4} = 0.
\end{equation}
Here $V$ lives in the orthogonal complement to the Sudakov subspace, and $a_{v}$ and $b_{v}$ are the Sudakov moduli. Since the Sudakov momenta are null, you have
\begin{equation}
	a_{v} = \frac{v \cdot k_{4}}{k_{3} \cdot k_{4}}, \qquad b_{v} = \frac{v \cdot k_{3}}{k_{3} \cdot k_{4}}.
\end{equation}
Thus,
\begin{equation}
	\abs{V}^{2} = \abs{v}^{2} - 2 a_{v} b_{v} \left( k_{3} \cdot k_{4} \right).
\end{equation}
The external momenta split into the incoming set ($p_{1}$ and $p_{2}$) and the outgoing set ($p_{3}$ and $p_{4}$). You have already null-decomposed the outgoing set by introducing the Sudakov momenta in (\ref{eq:sudakov_3_4}). Hence, by definition, the outgoing set live in the Sudakov subspace. The Sudakov null decomposition of the incoming set is
\begin{equation}
	p_{1} = P_{1} + a_{1} k_{3} + b_{1} k_{4}, \qquad p_{2} = P_{2} + a_{2} k_{3} + b_{2} k_{4}.
\end{equation}
Using the conservation constraint, you find
\begin{equation}
	P_{1} + P_{2} + \left( a_{1} + a_{2} \right) k_{3} + \left( b_{1} + b_{2} \right) k_{4} = \left( 1 + a_{4} \right) k_{3} + \left( 1 + b_{3} \right) k_{4}.
\end{equation}
This leads to three relations:
\begin{equation}
	P_{1} + P_{2} = 0, \qquad a_{1} + a_{2} = 1 + a_{4}, \qquad b_{1} + b_{2} = 1 + b_{3}.
	\label{eq:conv_rel}
\end{equation}
You have
\begin{equation}
	a_{1} = \frac{a_{4} \left( t - 2m^{2} \right) - \left( u - 2m^{2} \right) }{\sqrt{\Lambda_{12}(s)}}, \qquad
	b_{1} = \frac{b_{3} \left( u - 2m^{2} \right) - \left( t - 2m^{2} \right) }{\sqrt{\Lambda_{12}(s)}};
\end{equation}
and
\begin{equation}
	a_{2} = \frac{a_{4} \left( u - 2m^{2} \right) - \left( t - 2m^{2} \right) }{\sqrt{\Lambda_{12}(s)}}, \qquad
	b_{2} = \frac{b_{3} \left( t - 2m^{2} \right) - \left( u - 2m^{2} \right) }{\sqrt{\Lambda_{12}(s)}}.
\end{equation}
With the null decomposition of the external momenta, you can null-decompose any other combination of external momenta. Let
\begin{equation}
	p \equiv p_{1} + p_{2} = p_{3} + p_{4}, \qquad q \equiv p_{1} - p_{3} = p_{4} - p_{2}, \qquad r \equiv p_{1} - p_{4} = p_{3} - p_{2}.
\end{equation}
These are the spacetime vectors that yield the Mandelstam invariants:
\begin{equation}
	s = -\abs{p}^{2}, \qquad t = -\abs{q}^{2}, \qquad u = -\abs{r}^{2}.
\end{equation}
The Sudakov null decomposition of these linear combinations of momenta is
\begin{equation}
	p = P + a_{p} k_{3} + b_{p} k_{4}, \qquad q = Q + a_{q} k_{3} + b_{q} k_{4}, \qquad r = R + a_{r} k_{3} + b_{r} k_{4}.
\end{equation}
Using the null decomposition of the external momenta yields:
\begin{align}
	P = P_{1} + P_{2} = 0, \qquad a_{p} &= a_{1} + a_{2} = 1 + a_{4}, \qquad b_{p} = b_{1} + b_{2} = 1 + b_{3}; \\
	Q = P_{1} = -P_{2}, \qquad a_{q} &= a_{1} - 1 = a_{4} - a_{2}, \qquad b_{q} = b_{1} - b_{3} = 1 - b_{2}; \\
	R = P_{1} = -P_{2}, \qquad a_{r} &= a_{1} - a_{4} = 1 - a_{2}, \qquad b_{r} = b_{1} - 1 = b_{3} - b_{2}.
\end{align}
Explicitly,
\begin{align}
	a_{1} + a_{2} &= \frac{ s \left(1 - a_{4} \right) }{\sqrt{\Lambda_{12}(s)}}, \\
	b_{1} + b_{2} &= \frac{ s \left(1 - b_{3} \right) }{\sqrt{\Lambda_{12}(s)}}.
\end{align}
Using
\begin{equation}
	\frac{s}{\sqrt{\Lambda_{12}(s)}} = \frac{\left( 1 + b_{3} \right) + b_{3} \left( 1 + a_{4} \right)}{1 - b_{3} a_{4}}, \qquad \frac{s}{\sqrt{\Lambda_{12}(s)}} = \frac{\left( 1 + a_{4} \right) + a_{4} \left( 1 + b_{3} \right)}{1 - b_{3} a_{4}};
\end{equation}
you find that $a_{p} = 1 + a_{4}$ and $b_{p} = 1 + b_{3}$. This checks the relations from (\ref{eq:conv_rel}). You also have
\begin{equation}
	a_{q} = \frac{\left( 1 + a_{4} \right) t}{\sqrt{\Lambda_{12}(s)}}, \qquad
	b_{q} = -\frac{\left( 1 + b_{3} \right) t}{\sqrt{\Lambda_{12}(s)}};
\end{equation}
and
\begin{equation}
	a_{r} = \frac{\left( 1 + a_{4} \right) t}{\sqrt{\Lambda_{12}(s)}} + 1 - a_{4}, \qquad 
	b_{r} = - \frac{\left( 1 + b_{3} \right) t}{\sqrt{\Lambda_{12}(s)}} - 1 + b_{3}.
\end{equation}
From $t = -\abs{q}^{2}$ it follows that
\begin{equation}
	\abs{Q}^{2} = -t \left[ 1 + \frac{s t}{\Lambda_{12}(s)} \right].
\end{equation}
Thus,
\begin{equation}
	\abs{R}^{2} = \abs{P_{1}}^{2} = \abs{P_{2}}^{2} = \abs{Q}^{2} = -t \left[ 1 + \frac{s t}{\Lambda_{12}(s)} \right].
\end{equation}
%%%%%%%%%%%%%%%%%%%%%%%%%%%%%%%%%%%%%%%%%%%%%%%%%%%%%%%%%%%%%%%%%%%%%%%%%%%%%%%%%%%%%%%%%%%%%%%%%%%%%%%%%%%%%%%%%%%%
\subsection{Scattering Regimes}
%%%%%%%%%%%%%%%%%%%%%%%%%%%%%%%%%%%%%%%%%%%%%%%%%%%%%%%%%%%%%%%%%%%%%%%%%%%%%%%%%%%%%%%%%%%%%%%%%%%%%%%%%%%%%%%%%%%%
...
%%%%%%%%%%%%%%%%%%%%%%%%%%%%%%%%%%%%%%%%%%%%%%%%%%%%%%%%%%%%%%%%%%%%%%%%%%%%%%%%%%%%%%%%%%%%%%%%%%%%%%%%%%%%%%%%%%%%
\subsubsection{Slow Small-Speed Scattering}
%%%%%%%%%%%%%%%%%%%%%%%%%%%%%%%%%%%%%%%%%%%%%%%%%%%%%%%%%%%%%%%%%%%%%%%%%%%%%%%%%%%%%%%%%%%%%%%%%%%%%%%%%%%%%%%%%%%%
Small (slow) speeds occur when $s$ is close to the threshold value $4 m^{2}$.

In this regime you have
\begin{equation}
	b_{3} \rightarrow 1, \qquad a_{4} \rightarrow 1.
\end{equation}
%%%%%%%%%%%%%%%%%%%%%%%%%%%%%%%%%%%%%%%%%%%%%%%%%%%%%%%%%%%%%%%%%%%%%%%%%%%%%%%%%%%%%%%%%%%%%%%%%%%%%%%%%%%%%%%%%%%%
\subsubsection{Slow Fixed-Speed Scattering}
%%%%%%%%%%%%%%%%%%%%%%%%%%%%%%%%%%%%%%%%%%%%%%%%%%%%%%%%%%%%%%%%%%%%%%%%%%%%%%%%%%%%%%%%%%%%%%%%%%%%%%%%%%%%%%%%%%%%
The (slow) speed of each external quantum is a function of one dimensionless ratio,
\begin{equation}
	\frac{s}{m^{2}}.
\end{equation}
Fixed-speed scattering corresponds to the kinematic regime where we keep this ratio fixed:
\begin{equation}
	\frac{s}{m^{2}} \text{ fixed}.
\end{equation}
This regime is appropriate for either large $s$ and large mass, or small $s$ and small mass. Note that fixed-speed is equivalent to fixed-rapidity. By itself, this regime is not very helpful, but when combined with other limits it leads to important approximations.

In this regime both $b_{3}$ and $a_{4}$ are kept fixed.
%%%%%%%%%%%%%%%%%%%%%%%%%%%%%%%%%%%%%%%%%%%%%%%%%%%%%%%%%%%%%%%%%%%%%%%%%%%%%%%%%%%%%%%%%%%%%%%%%%%%%%%%%%%%%%%%%%%%
\subsubsection{Slow Large-Speed Scattering}
%%%%%%%%%%%%%%%%%%%%%%%%%%%%%%%%%%%%%%%%%%%%%%%%%%%%%%%%%%%%%%%%%%%%%%%%%%%%%%%%%%%%%%%%%%%%%%%%%%%%%%%%%%%%%%%%%%%%
Large-speed scattering involves the regime
\begin{equation}
	\frac{s}{m^{2}} \rightarrow \infty.
\end{equation}
That is, $\sqrt{s}$ is very large compared to the mass.

In this regime you have
\begin{equation}
	b_{3} \rightarrow 0, \qquad a_{4} \rightarrow 0.
\end{equation}
That is, both $p_{3}$ and $p_{4}$ become null momenta.
%%%%%%%%%%%%%%%%%%%%%%%%%%%%%%%%%%%%%%%%%%%%%%%%%%%%%%%%%%%%%%%%%%%%%%%%%%%%%%%%%%%%%%%%%%%%%%%%%%%%%%%%%%%%%%%%%%%%
\subsubsection{Regge Scattering}
%%%%%%%%%%%%%%%%%%%%%%%%%%%%%%%%%%%%%%%%%%%%%%%%%%%%%%%%%%%%%%%%%%%%%%%%%%%%%%%%%%%%%%%%%%%%%%%%%%%%%%%%%%%%%%%%%%%%
Regge scattering is the regime of fixed-speed and large (unphysical) $z_{13}$. This corresponds to
\begin{equation}
	\frac{t}{s} \rightarrow \infty, \qquad \frac{s}{m^{2}} \text{ fixed}.
\end{equation}
As a corollary, you have
\begin{equation}
	u = 4m^{2} - s - t \quad \Longrightarrow \quad \frac{u}{s} \rightarrow -\infty.
\end{equation}
In this regime all dual conformal invariants become trivial:
\begin{equation}
\begin{split}
	\crat{\red{1}}{\red{2}}{\red{3}}{\red{4}} \rightarrow \infty, \qquad
	\crat{\red{1}}{\red{3}}{\red{4}}{\red{2}} &= 1, \qquad
	\crat{\red{1}}{\red{4}}{\red{2}}{\red{3}} \rightarrow 0, \\
	\crat{\red{1}}{\red{2}}{\red{4}}{\red{3}} \rightarrow 0, \qquad
	\crat{\red{1}}{\red{3}}{\red{2}}{\red{4}} &= 1, \qquad
	\crat{\red{1}}{\red{4}}{\red{3}}{\red{2}} \rightarrow \infty.
\end{split}
\end{equation}
\begin{equation}
\begin{split}
	\crat{\blue{1}}{\blue{2}}{\blue{3}}{\blue{4}} \rightarrow -\infty, \qquad
	\crat{\blue{1}}{\blue{3}}{\blue{4}}{\blue{2}} &= 1, \qquad
	\crat{\blue{1}}{\blue{4}}{\blue{2}}{\blue{3}} \rightarrow 0, \\
	\crat{\blue{1}}{\blue{2}}{\blue{4}}{\blue{3}} \rightarrow 0, \qquad
	\crat{\blue{1}}{\blue{3}}{\blue{2}}{\blue{4}} &= 1, \qquad
	\crat{\blue{1}}{\blue{4}}{\blue{3}}{\blue{2}} \rightarrow -\infty.
\end{split}
\end{equation}
\begin{equation}
\begin{split}
	\crat{\green{1}}{\green{2}}{\green{3}}{\green{4}} \rightarrow -\infty, \qquad
	\crat{\green{1}}{\green{3}}{\green{4}}{\green{2}} &= 1, \qquad
	\crat{\green{1}}{\green{4}}{\green{2}}{\green{3}} \rightarrow 0, \\
	\crat{\green{1}}{\green{2}}{\green{4}}{\green{3}} \rightarrow 0, \qquad
	\crat{\green{1}}{\green{3}}{\green{2}}{\green{4}} &= 1, \qquad
	\crat{\green{1}}{\green{4}}{\green{3}}{\green{2}} \rightarrow -\infty.
\end{split}
\end{equation}
%%%%%%%%%%%%%%%%%%%%%%%%%%%%%%%%%%%%%%%%%%%%%%%%%%%%%%%%%%%%%%%%%%%%%%%%%%%%%%%%%%%%%%%%%%%%%%%%%%%%%%%%%%%%%%%%%%%%
\subsubsection{Forward Scattering}
%%%%%%%%%%%%%%%%%%%%%%%%%%%%%%%%%%%%%%%%%%%%%%%%%%%%%%%%%%%%%%%%%%%%%%%%%%%%%%%%%%%%%%%%%%%%%%%%%%%%%%%%%%%%%%%%%%%%
Forward scattering is the regime of fixed-speed and small (physical) scattering angles (i.e. $z_{13} \rightarrow 1$). This can be stated as
\begin{equation}
	\frac{t}{s} \rightarrow 0, \qquad \frac{s}{m^{2}} \text{ fixed}.
\end{equation}
As a corollary, you have
\begin{equation}
	u = 4m^{2} - s - t \quad \Longrightarrow \quad \frac{u}{s} \text{ fixed}.
\end{equation}
Unlike the case of Regge scattering, in the forward regime only some of the dual conformal invariants are trivial:
\begin{equation}
\begin{split}
	\crat{\red{1}}{\red{2}}{\red{3}}{\red{4}} \rightarrow 0, \qquad
	\crat{\red{1}}{\red{3}}{\red{4}}{\red{2}} &= 1, \qquad
	\crat{\red{1}}{\red{4}}{\red{2}}{\red{3}} \rightarrow \infty, \\
	\crat{\red{1}}{\red{2}}{\red{4}}{\red{3}} \rightarrow \infty, \qquad
	\crat{\red{1}}{\red{3}}{\red{2}}{\red{4}} &= 1, \qquad
	\crat{\red{1}}{\red{4}}{\red{3}}{\red{2}} \rightarrow 0.
\end{split}
\end{equation}
\begin{equation}
\begin{split}
	\crat{\blue{1}}{\blue{2}}{\blue{3}}{\blue{4}} \rightarrow 0, \qquad
	\crat{\blue{1}}{\blue{3}}{\blue{4}}{\blue{2}} &= 1, \qquad
	\crat{\blue{1}}{\blue{4}}{\blue{2}}{\blue{3}} \rightarrow \infty, \\
	\crat{\blue{1}}{\blue{2}}{\blue{4}}{\blue{3}} \rightarrow \infty, \qquad
	\crat{\blue{1}}{\blue{3}}{\blue{2}}{\blue{4}} &= 1, \qquad
	\crat{\blue{1}}{\blue{4}}{\blue{3}}{\blue{2}} \rightarrow 0.
\end{split}
\end{equation}
The rest of the dual conformal invariants are fixed and nontrivial:
\begin{equation}
\begin{split}
	\crat{\green{1}}{\green{2}}{\green{3}}{\green{4}} = \frac{s u}{m^{4}}, \qquad
	\crat{\green{1}}{\green{3}}{\green{4}}{\green{2}} &= 1, \qquad
	\crat{\green{1}}{\green{4}}{\green{2}}{\green{3}} = \frac{m^{4}}{s u}, \\
	\crat{\green{1}}{\green{2}}{\green{4}}{\green{3}} = \frac{m^{4}}{s u}, \qquad
	\crat{\green{1}}{\green{3}}{\green{2}}{\green{4}} &= 1, \qquad
	\crat{\green{1}}{\green{4}}{\green{3}}{\green{2}} = \frac{s u}{m^{4}}.
\end{split}
\end{equation}
%%%%%%%%%%%%%%%%%%%%%%%%%%%%%%%%%%%%%%%%%%%%%%%%%%%%%%%%%%%%%%%%%%%%%%%%%%%%%%%%%%%%%%%%%%%%%%%%%%%%%%%%%%%%%%%%%%%%
\subsubsection{Backward Scattering}
%%%%%%%%%%%%%%%%%%%%%%%%%%%%%%%%%%%%%%%%%%%%%%%%%%%%%%%%%%%%%%%%%%%%%%%%%%%%%%%%%%%%%%%%%%%%%%%%%%%%%%%%%%%%%%%%%%%%
Backward scattering is the regime of fixed-speed and large (physical) scattering angles (i.e. $z_{13} \rightarrow -1$). This can be stated as
\begin{equation}
	\frac{u}{s} \rightarrow 0, \qquad \frac{s}{m^{2}} \text{ fixed}.
\end{equation}
As a corollary, you have
\begin{equation}
	t = 4m^{2} - s - u \quad \Longrightarrow \quad \frac{t}{s} \text{ fixed}.
\end{equation}
Similarly to the forward scattering regime, some of the dual conformal invariants are trivial:
\begin{equation}
\begin{split}
	\crat{\blue{1}}{\blue{2}}{\blue{3}}{\blue{4}} \rightarrow 0, \qquad
	\crat{\blue{1}}{\blue{3}}{\blue{4}}{\blue{2}} &= 1, \qquad
	\crat{\blue{1}}{\blue{4}}{\blue{2}}{\blue{3}} \rightarrow \infty, \\
	\crat{\blue{1}}{\blue{2}}{\blue{4}}{\blue{3}} \rightarrow \infty, \qquad
	\crat{\blue{1}}{\blue{3}}{\blue{2}}{\blue{4}} &= 1, \qquad
	\crat{\blue{1}}{\blue{4}}{\blue{3}}{\blue{2}} \rightarrow 0.
\end{split}
\end{equation}
\begin{equation}
\begin{split}
	\crat{\green{1}}{\green{2}}{\green{3}}{\green{4}} \rightarrow 0, \qquad
	\crat{\green{1}}{\green{3}}{\green{4}}{\green{2}} &= 1, \qquad
	\crat{\green{1}}{\green{4}}{\green{2}}{\green{3}} \rightarrow \infty, \\
	\crat{\green{1}}{\green{2}}{\green{4}}{\green{3}} \rightarrow \infty, \qquad
	\crat{\green{1}}{\green{3}}{\green{2}}{\green{4}} &= 1, \qquad
	\crat{\green{1}}{\green{4}}{\green{3}}{\green{2}} \rightarrow 0.
\end{split}
\end{equation}
The rest of the dual conformal invariants are fixed and nontrivial:
\begin{equation}
\begin{split}
	\crat{\red{1}}{\red{2}}{\red{3}}{\red{4}} = \frac{s t}{m^{4}}, \qquad
	\crat{\red{1}}{\red{3}}{\red{4}}{\red{2}} &= 1, \qquad
	\crat{\red{1}}{\red{4}}{\red{2}}{\red{3}} = \frac{m^{4}}{s t}, \\
	\crat{\red{1}}{\red{2}}{\red{4}}{\red{3}} = \frac{m^{4}}{s t}, \qquad
	\crat{\red{1}}{\red{3}}{\red{2}}{\red{4}} &= 1, \qquad
	\crat{\red{1}}{\red{4}}{\red{3}}{\red{2}} = \frac{s t}{m^{4}}.
\end{split}
\end{equation}
%%%%%%%%%%%%%%%%%%%%%%%%%%%%%%%%%%%%%%%%%%%%%%%%%%%%%%%%%%%%%%%%%%%%%%%%%%%%%%%%%%%%%%%%%%%%%%%%%%%%%%%%%%%%%%%%%%%%
\subsubsection{Fixed-Angle Scattering}
%%%%%%%%%%%%%%%%%%%%%%%%%%%%%%%%%%%%%%%%%%%%%%%%%%%%%%%%%%%%%%%%%%%%%%%%%%%%%%%%%%%%%%%%%%%%%%%%%%%%%%%%%%%%%%%%%%%%
Fixed-angle scattering is the regime of large-speed and (physical) fixed-angle. This can be stated as
\begin{equation}
	\frac{t}{s} \text{ fixed}, \qquad \frac{s}{m^{2}} \rightarrow \infty.
\end{equation}
As a corollary, you have
\begin{equation}
	u = 4m^{2} - s - t \quad \Longrightarrow \quad \frac{u}{s} \text{ fixed}.
\end{equation}
In this regime all dual conformal invariants become trivial:
\begin{equation}
\begin{split}
	\crat{\red{1}}{\red{2}}{\red{3}}{\red{4}} \rightarrow -\infty, \qquad
	\crat{\red{1}}{\red{3}}{\red{4}}{\red{2}} &= 1, \qquad
	\crat{\red{1}}{\red{4}}{\red{2}}{\red{3}} \rightarrow 0, \\
	\crat{\red{1}}{\red{2}}{\red{4}}{\red{3}} \rightarrow 0, \qquad
	\crat{\red{1}}{\red{3}}{\red{2}}{\red{4}} &= 1, \qquad
	\crat{\red{1}}{\red{4}}{\red{3}}{\red{2}} \rightarrow -\infty.
\end{split}
\end{equation}
\begin{equation}
\begin{split}
	\crat{\blue{1}}{\blue{2}}{\blue{3}}{\blue{4}} \rightarrow \infty, \qquad
	\crat{\blue{1}}{\blue{3}}{\blue{4}}{\blue{2}} &= 1, \qquad
	\crat{\blue{1}}{\blue{4}}{\blue{2}}{\blue{3}} \rightarrow 0, \\
	\crat{\blue{1}}{\blue{2}}{\blue{4}}{\blue{3}} \rightarrow 0, \qquad
	\crat{\blue{1}}{\blue{3}}{\blue{2}}{\blue{4}} &= 1, \qquad
	\crat{\blue{1}}{\blue{4}}{\blue{3}}{\blue{2}} \rightarrow \infty.
\end{split}
\end{equation}
\begin{equation}
\begin{split}
	\crat{\green{1}}{\green{2}}{\green{3}}{\green{4}} \rightarrow -\infty, \qquad
	\crat{\green{1}}{\green{3}}{\green{4}}{\green{2}} &= 1, \qquad
	\crat{\green{1}}{\green{4}}{\green{2}}{\green{3}} \rightarrow 0, \\
	\crat{\green{1}}{\green{2}}{\green{4}}{\green{3}} \rightarrow 0, \qquad
	\crat{\green{1}}{\green{3}}{\green{2}}{\green{4}} &= 1, \qquad
	\crat{\green{1}}{\green{4}}{\green{3}}{\green{2}} \rightarrow -\infty.
\end{split}
\end{equation}
From the point of view of dual conformal invariants, Regge scattering and fixed-angle scattering are very similar.