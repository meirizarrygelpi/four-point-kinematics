\chapter{Maximal Inelastic Slow Diversity}
%%%%%%%%%%%%%%%%%%%%%%%%%%%%%%%%%%%%%%%%%%%%%%%%%%%%%%%%%%%%%%%%%%%%%%%%%%%%%%%%%%%%%%%%%%%%%%%%%%%%%%%%%%%%%%%%%%%%
In this chapter I will consider the kinematics of a 2-to-2 scattering process that has maximal inelastic slow diversity. This means that all external quanta are slow, and all of the corresponding (slow) masses are distinct:
\begin{equation}
	m_{1}^{2} = -\abs{p_{1}}^{2}, \qquad m_{2}^{2} = -\abs{p_{2}}^{2}, \qquad m_{3}^{2} = -\abs{p_{3}}^{2}, \qquad m_{4}^{2} = -\abs{p_{4}}^{2}.
	\label{eq:shell_inelastic_slow}
\end{equation}
Thus, this is an inelastic process. For example,
\begin{equation}
	A(p_{1}) + B(p_{2}) \longrightarrow Y(p_{3}) + Z(p_{4}).
\end{equation}
For this process, (\ref{eq:stu_slow}) becomes
\begin{equation}
	s + t + u = m_{1}^{2} + m_{2}^{2} + m_{3}^{2} + m_{4}^{2}.
	\label{stu_1234}
\end{equation}
%%%%%%%%%%%%%%%%%%%%%%%%%%%%%%%%%%%%%%%%%%%%%%%%%%%%%%%%%%%%%%%%%%%%%%%%%%%%%%%%%%%%%%%%%%%%%%%%%%%%%%%%%%%%%%%%%%%%
\section{Simplicial Invariants}
%%%%%%%%%%%%%%%%%%%%%%%%%%%%%%%%%%%%%%%%%%%%%%%%%%%%%%%%%%%%%%%%%%%%%%%%%%%%%%%%%%%%%%%%%%%%%%%%%%%%%%%%%%%%%%%%%%%%
Since this scattering process is inelastic, the simplicial invariants take many distinct values. 
%%%%%%%%%%%%%%%%%%%%%%%%%%%%%%%%%%%%%%%%%%%%%%%%%%%%%%%%%%%%%%%%%%%%%%%%%%%%%%%%%%%%%%%%%%%%%%%%%%%%%%%%%%%%%%%%%%%%
\subsection{1-Simplex Invariants}
%%%%%%%%%%%%%%%%%%%%%%%%%%%%%%%%%%%%%%%%%%%%%%%%%%%%%%%%%%%%%%%%%%%%%%%%%%%%%%%%%%%%%%%%%%%%%%%%%%%%%%%%%%%%%%%%%%%%
For the \red{red} tetrahedron you have
\begin{equation}
	C_{\red{12}} = m_{1}^{2}, \quad C_{\red{13}} = s, \quad C_{\red{14}} = m_{3}^{2}, \quad
	C_{\red{23}} = m_{2}^{2}, \quad C_{\red{24}} = t, \quad C_{\red{34}} = m_{4}^{2}.
\end{equation}
Similarly, for the \blue{blue} tetrahedron you have
\begin{equation}
	C_{\blue{12}} = m_{3}^{2}, \quad C_{\blue{13}} = t, \quad C_{\blue{14}} = m_{2}^{2}, \quad
	C_{\blue{23}} = m_{1}^{2}, \quad C_{\blue{24}} = u, \quad C_{\blue{34}} = m_{4}^{2}.
\end{equation}
Finally, for the \green{green} tetrahedron you have
\begin{equation}
	C_{\green{12}} = m_{2}^{2}, \quad C_{\green{13}} = u, \quad C_{\green{14}} = m_{1}^{2}, \quad
	C_{\green{23}} = m_{3}^{2}, \quad C_{\green{24}} = s, \quad C_{\green{34}} = m_{4}^{2}.
\end{equation}
%%%%%%%%%%%%%%%%%%%%%%%%%%%%%%%%%%%%%%%%%%%%%%%%%%%%%%%%%%%%%%%%%%%%%%%%%%%%%%%%%%%%%%%%%%%%%%%%%%%%%%%%%%%%%%%%%%%%
\subsection{2-Simplex Invariants}
%%%%%%%%%%%%%%%%%%%%%%%%%%%%%%%%%%%%%%%%%%%%%%%%%%%%%%%%%%%%%%%%%%%%%%%%%%%%%%%%%%%%%%%%%%%%%%%%%%%%%%%%%%%%%%%%%%%%
For the \red{red} tetrahedron you have
\begin{equation}
\begin{split}
	C_{\red{123}} &= \kallen{s}{m_{1}}{m_{2}} = \Lambda_{12}(s), \\
	C_{\red{124}} &= \kallen{t}{m_{1}}{m_{3}} = \Lambda_{13}(t), \\
	C_{\red{134}} &= \kallen{s}{m_{3}}{m_{4}} = \Lambda_{34}(s), \\
	C_{\red{234}} &= \kallen{t}{m_{2}}{m_{4}} = \Lambda_{24}(t).
\end{split}
\end{equation}
Similarly, for the \blue{blue} tetrahedron you have
\begin{equation}
\begin{split}
	C_{\blue{123}} &= \kallen{t}{m_{1}}{m_{3}} = \Lambda_{13}(t), \\
	C_{\blue{124}} &= \kallen{u}{m_{2}}{m_{3}} = \Lambda_{23}(u), \\
	C_{\blue{134}} &= \kallen{t}{m_{2}}{m_{4}} = \Lambda_{24}(t), \\
	C_{\blue{234}} &= \kallen{u}{m_{1}}{m_{4}} = \Lambda_{14}(u).
\end{split}
\end{equation}
Finally, for the \green{green} tetrahedron you have
\begin{equation}
\begin{split}
	C_{\green{123}} &= \kallen{u}{m_{2}}{m_{3}} = \Lambda_{23}(u), \\
	C_{\green{124}} &= \kallen{s}{m_{1}}{m_{2}} = \Lambda_{12}(s), \\
	C_{\green{134}} &= \kallen{u}{m_{1}}{m_{4}} = \Lambda_{14}(u), \\
	C_{\green{234}} &= \kallen{s}{m_{3}}{m_{4}} = \Lambda_{34}(s).
\end{split}
\end{equation}
%%%%%%%%%%%%%%%%%%%%%%%%%%%%%%%%%%%%%%%%%%%%%%%%%%%%%%%%%%%%%%%%%%%%%%%%%%%%%%%%%%%%%%%%%%%%%%%%%%%%%%%%%%%%%%%%%%%%
\subsection{3-Simplex Invariants}
%%%%%%%%%%%%%%%%%%%%%%%%%%%%%%%%%%%%%%%%%%%%%%%%%%%%%%%%%%%%%%%%%%%%%%%%%%%%%%%%%%%%%%%%%%%%%%%%%%%%%%%%%%%%%%%%%%%%
Each tetrahedron has only one possible 3-simplex invariant. Indeed,
\begin{equation}
	C_{\red{1234}} = C_{\blue{1234}} = C_{\green{1234}}.
\end{equation}
%%%%%%%%%%%%%%%%%%%%%%%%%%%%%%%%%%%%%%%%%%%%%%%%%%%%%%%%%%%%%%%%%%%%%%%%%%%%%%%%%%%%%%%%%%%%%%%%%%%%%%%%%%%%%%%%%%%%
\section{Dual Conformal Invariants}
%%%%%%%%%%%%%%%%%%%%%%%%%%%%%%%%%%%%%%%%%%%%%%%%%%%%%%%%%%%%%%%%%%%%%%%%%%%%%%%%%%%%%%%%%%%%%%%%%%%%%%%%%%%%%%%%%%%%
For the \red{red} tetrahedron, you have
\begin{equation}
\begin{split}
	\crat{\red{1}}{\red{2}}{\red{3}}{\red{4}} = \frac{s t}{m_{2}^{2} m_{3}^{2}}, \qquad
	\crat{\red{1}}{\red{3}}{\red{4}}{\red{2}} &= \frac{m_{2}^{2} m_{3}^{2}}{m_{1}^{2} m_{4}^{2}}, \qquad
	\crat{\red{1}}{\red{4}}{\red{2}}{\red{3}} = \frac{m_{1}^{2} m_{4}^{2}}{s t}, \\
	\crat{\red{1}}{\red{2}}{\red{4}}{\red{3}} = \frac{m_{2}^{2} m_{3}^{2}}{s t}, \qquad
	\crat{\red{1}}{\red{3}}{\red{2}}{\red{4}} &= \frac{m_{1}^{2} m_{4}^{2}}{m_{2}^{2} m_{3}^{2}}, \qquad
	\crat{\red{1}}{\red{4}}{\red{3}}{\red{2}} = \frac{s t}{m_{1}^{2} m_{4}^{2}}.
\end{split}
\end{equation}
Similarly, for the \blue{blue} tetrahedron, you have
\begin{equation}
\begin{split}
	\crat{\blue{1}}{\blue{2}}{\blue{3}}{\blue{4}} = \frac{t u}{m_{1}^{2} m_{2}^{2}}, \qquad
	\crat{\blue{1}}{\blue{3}}{\blue{4}}{\blue{2}} &= \frac{m_{1}^{2} m_{2}^{2}}{m_{3}^{2} m_{4}^{2}}, \qquad
	\crat{\blue{1}}{\blue{4}}{\blue{2}}{\blue{3}} = \frac{m_{3}^{2} m_{4}^{2}}{t u}, \\
	\crat{\blue{1}}{\blue{2}}{\blue{4}}{\blue{3}} = \frac{m_{1}^{2} m_{2}^{2}}{t u}, \qquad
	\crat{\blue{1}}{\blue{3}}{\blue{2}}{\blue{4}} &= \frac{m_{3}^{2} m_{4}^{2}}{m_{1}^{2} m_{2}^{2}}, \qquad
	\crat{\blue{1}}{\blue{4}}{\blue{3}}{\blue{2}} = \frac{t u}{m_{3}^{2} m_{4}^{2}}.
\end{split}
\end{equation}
Finally, for the \green{green} tetrahedron, you have
\begin{equation}
\begin{split}
	\crat{\green{1}}{\green{2}}{\green{3}}{\green{4}} = \frac{s u}{m_{1}^{2} m_{3}^{2}}, \qquad
	\crat{\green{1}}{\green{3}}{\green{4}}{\green{2}} &= \frac{m_{1}^{2} m_{3}^{2}}{m_{2}^{2} m_{4}^{2}}, \qquad
	\crat{\green{1}}{\green{4}}{\green{2}}{\green{3}} = \frac{m_{2}^{2} m_{4}^{2}}{s u}, \\
	\crat{\green{1}}{\green{2}}{\green{4}}{\green{3}} = \frac{m_{1}^{2} m_{3}^{2}}{s u}, \qquad
	\crat{\green{1}}{\green{3}}{\green{2}}{\green{4}} &= \frac{m_{2}^{2} m_{4}^{2}}{m_{1}^{2} m_{3}^{2}}, \qquad
	\crat{\green{1}}{\green{4}}{\green{3}}{\green{2}} = \frac{s u}{m_{2}^{2} m_{4}^{2}}.
\end{split}
\end{equation}
%%%%%%%%%%%%%%%%%%%%%%%%%%%%%%%%%%%%%%%%%%%%%%%%%%%%%%%%%%%%%%%%%%%%%%%%%%%%%%%%%%%%%%%%%%%%%%%%%%%%%%%%%%%%%%%%%%%%
\section{Sudakov Decompositions}
%%%%%%%%%%%%%%%%%%%%%%%%%%%%%%%%%%%%%%%%%%%%%%%%%%%%%%%%%%%%%%%%%%%%%%%%%%%%%%%%%%%%%%%%%%%%%%%%%%%%%%%%%%%%%%%%%%%%
Since all the external quanta are slow, you can have either slow or null Sudakov momenta.
%%%%%%%%%%%%%%%%%%%%%%%%%%%%%%%%%%%%%%%%%%%%%%%%%%%%%%%%%%%%%%%%%%%%%%%%%%%%%%%%%%%%%%%%%%%%%%%%%%%%%%%%%%%%%%%%%%%%
\subsection{Slow Sudakov Momenta}
%%%%%%%%%%%%%%%%%%%%%%%%%%%%%%%%%%%%%%%%%%%%%%%%%%%%%%%%%%%%%%%%%%%%%%%%%%%%%%%%%%%%%%%%%%%%%%%%%%%%%%%%%%%%%%%%%%%%
The simplest choice for slow Sudakov momenta is
\begin{equation}
	k_{3} = p_{3}, \qquad k_{4} = p_{4}.
\end{equation}
That is,
\begin{equation}
	c_{34} = c_{43} = 0.
\end{equation}
You have
\begin{equation}
	\abs{k_{3}}^{2} = -m_{3}^{2}, \qquad \abs{k_{4}}^{2} = -m_{4}^{2} \quad \Longrightarrow \quad 2\left( k_{3} \cdot k_{4} \right) = m_{3}^{2} + m_{4}^{2} - s.
\end{equation}
Note that
\begin{equation}
	\abs{k_{3}}^{2} \abs{k_{4}}^{2} - \left( k_{3} \cdot k_{4} \right)^{2} = \frac{1}{4} \left[ s - \left( m_{3} - m_{4} \right)^{2} \right] \left[ \left( m_{3} + m_{4} \right)^{2} - s \right] = - \frac{\Lambda_{34}(s)}{4}.
\end{equation}
For this choice of Sudakov momenta, you find
\begin{align}
	c_{13} &= \frac{2 m_{4}^{2} \left(t - m_{1}^{2} - m_{3}^{2} \right) - \left(s - m_{3}^{2} - m_{4}^{2} \right) \left(u - m_{1}^{2} - m_{4}^{2} \right) }{\Lambda_{34}(s)}, \\
	c_{14} &= \frac{2 m_{3}^{2} \left(u - m_{1}^{2} - m_{4}^{2} \right) - \left(s - m_{3}^{2} - m_{4}^{2} \right) \left(t - m_{1}^{2} - m_{3}^{2} \right) }{\Lambda_{34}(s)}, \\
	c_{23} &= \frac{2 m_{4}^{2} \left(u - m_{2}^{2} - m_{3}^{2} \right) - \left(s - m_{3}^{2} - m_{4}^{2} \right) \left(t - m_{2}^{2} - m_{4}^{2} \right) }{\Lambda_{34}(s)}, \\
	c_{24} &= \frac{2 m_{3}^{2} \left(t - m_{2}^{2} - m_{4}^{2} \right) - \left(s - m_{3}^{2} - m_{4}^{2} \right) \left(u - m_{2}^{2} - m_{3}^{2} \right) }{\Lambda_{34}(s)}.
\end{align}
Thus,
\begin{equation}
	\abs{P_{1}}^{2} = ..., \qquad \abs{P_{2}}^{2} = ....
\end{equation}
%%%%%%%%%%%%%%%%%%%%%%%%%%%%%%%%%%%%%%%%%%%%%%%%%%%%%%%%%%%%%%%%%%%%%%%%%%%%%%%%%%%%%%%%%%%%%%%%%%%%%%%%%%%%%%%%%%%%
\subsection{Null Sudakov Momenta}
%%%%%%%%%%%%%%%%%%%%%%%%%%%%%%%%%%%%%%%%%%%%%%%%%%%%%%%%%%%%%%%%%%%%%%%%%%%%%%%%%%%%%%%%%%%%%%%%%%%%%%%%%%%%%%%%%%%%
If the Sudakov momenta are null, then
\begin{equation}
	\abs{k_{3}}^{2} = \abs{k_{4}}^{2} = 0.
\end{equation}
This leads to
\begin{equation}
\begin{split}
	m_{3}^{2} &= -2 c_{34} \left(k_{3} \cdot k_{4} \right), \\
	m_{4}^{2} &= -2 c_{43} \left(k_{3} \cdot k_{4} \right), \\
	s - m_{3}^{2} - m_{4}^{2} &= -2\left(1 + c_{34}c_{43} \right) \left(k_{3} \cdot k_{4} \right).
\end{split}
\end{equation}
From here it follows that
\begin{equation}
	\left( \frac{m_{3}}{m_{4}} \right)^{2} = \frac{c_{34}}{c_{43}}, \qquad \frac{s - m_{3}^{2} - m_{4}^{2}}{m_{3}^{2}} = \frac{1}{c_{34}} + c_{43}.
\end{equation}
Solving these equations yields
\begin{equation}
	c_{43} = \frac{s - m_{3}^{2} - m_{4}^{2} \pm \sqrt{\Lambda_{34}(s)}}{2 m_{3}^{2}}, \qquad c_{34} = \left( \frac{m_{3}}{m_{4}} \right)^{2} c_{43} = \frac{s - m_{3}^{2} - m_{4}^{2} \pm \sqrt{\Lambda_{34}(s)}}{2 m_{4}^{2}}.
\end{equation}
Thus,
\begin{align}
	1 + c_{34} &= \frac{s - m_{3}^{2} + m_{4}^{2} \pm \sqrt{\Lambda_{34}(s)}}{2 m_{4}^{2}}, \\
	1 + c_{43} &= \frac{s + m_{3}^{2} - m_{4}^{2} \pm \sqrt{\Lambda_{34}(s)}}{2 m_{3}^{2}}, \\
	c_{34} c_{43} &= \frac{s - m_{3}^{2} - m_{4}^{2} \pm \sqrt{\Lambda_{34}(s)}}{s - m_{3}^{2} - m_{4}^{2} \mp \sqrt{\Lambda_{34}(s)}},
\end{align}
which leads to
\begin{equation}
	1 - c_{34} c_{43} = \frac{\mp 2 \sqrt{\Lambda_{34}(s)}}{s - m_{3}^{2} - m_{4}^{2} \mp \sqrt{\Lambda_{34}(s)}} = \mp \frac{\sqrt{\Lambda_{34}(s)}}{m_{3} m_{4}} \left[ \frac{s - m_{3}^{2} - m_{4}^{2} \pm \sqrt{\Lambda_{34}(s)}}{2 m_{3} m_{4}} \right] .
\end{equation}
Furthermore,
\begin{equation}
	\left( k_{3} \cdot k_{4} \right) = - \frac{m_{3} m_{4}}{2} \left[ \frac{2m_{3} m_{4}}{s - m_{3}^{2} - m_{4}^{2} \pm \sqrt{\Lambda_{34}(s)}} \right],
\end{equation}
which can also be written as
\begin{equation}
	\left( k_{3} \cdot k_{4} \right) = - \frac{m_{3} m_{4}}{2} \left[ \frac{s - m_{3}^{2} - m_{4}^{2} \mp \sqrt{\Lambda_{34}(s)}}{2m_{3} m_{4}} \right].
\end{equation}
In the end, it is more useful to write $k_{3} \cdot k_{4}$ as an implicit function:
\begin{equation}
	\left( k_{3} \cdot k_{4} \right) = - \frac{m_{3}^{2}}{2 c_{34}} = - \frac{m_{4}^{2}}{2 c_{43}} = - \frac{s}{2 \left(1 + c_{34} \right) \left(1 + c_{43} \right) } = -\frac{s - m_{3}^{2} - m_{4}^{2}}{2 \left(1 + c_{34} c_{43} \right)}.
\end{equation}
Note that
\begin{align}
	p_{1} \cdot k_{3} &= \frac{ \left( p_{1} \cdot p_{3} \right) - c_{34} \left( p_{1} \cdot p_{4} \right) }{1 - c_{34} c_{43}} = \frac{ \left( t - m_{1}^{2} - m_{3}^{2} \right) - c_{34} \left( u - m_{1}^{2} - m_{4}^{2} \right) }{2 \left( 1 - c_{34} c_{43} \right)}, \\
	p_{1} \cdot k_{4} &= \frac{ \left( p_{1} \cdot p_{4} \right) - c_{43} \left( p_{1} \cdot p_{3} \right) }{1 - c_{34} c_{43}} = \frac{ \left( u - m_{1}^{2} - m_{4}^{2} \right) - c_{43} \left( t - m_{1}^{2} - m_{3}^{2} \right) }{2 \left( 1 - c_{34} c_{43} \right)}, \\
	p_{2} \cdot k_{3} &= \frac{ \left( p_{2} \cdot p_{3} \right) - c_{34} \left( p_{2} \cdot p_{4} \right) }{1 - c_{34} c_{43}} = \frac{ \left( u - m_{2}^{2} - m_{3}^{2} \right) - c_{34} \left( t - m_{2}^{2} - m_{4}^{2} \right) }{2 \left( 1 - c_{34} c_{43} \right)}, \\
	p_{2} \cdot k_{4} &= \frac{ \left( p_{2} \cdot p_{4} \right) - c_{43} \left( p_{2} \cdot p_{3} \right) }{1 - c_{34} c_{43}} = \frac{ \left( t - m_{2}^{2} - m_{4}^{2} \right) - c_{43} \left( u - m_{2}^{2} - m_{3}^{2} \right) }{2 \left( 1 - c_{34} c_{43} \right)}.
\end{align}
Using these, you find
\begin{align}
	c_{13} &= \frac{p_{1} \cdot k_{4}}{k_{3} \cdot k_{4}} = \frac{ \left( u - m_{1}^{2} - m_{4}^{2} \right) - c_{43} \left( t - m_{1}^{2} - m_{3}^{2} \right) }{s} \left[ \frac{\left(1 + c_{34} \right) \left(1 + c_{43} \right)}{\left(1 - c_{34} c_{43} \right)} \right], \\
	c_{14} &= \frac{p_{1} \cdot k_{3}}{k_{3} \cdot k_{4}} = \frac{ \left( t - m_{1}^{2} - m_{3}^{2} \right) - c_{34} \left( u - m_{1}^{2} - m_{4}^{2} \right) }{s} \left[ \frac{\left(1 + c_{34} \right) \left(1 + c_{43} \right)}{\left(1 - c_{34} c_{43} \right)} \right], \\
	c_{23} &= \frac{p_{2} \cdot k_{4}}{k_{3} \cdot k_{4}} = \frac{ \left( t - m_{2}^{2} - m_{4}^{2} \right) - c_{43} \left( u - m_{2}^{2} - m_{3}^{2} \right) }{s} \left[ \frac{\left(1 + c_{34} \right) \left(1 + c_{43} \right)}{\left(1 - c_{34} c_{43} \right)} \right], \\
	c_{24} &= \frac{p_{2} \cdot k_{3}}{k_{3} \cdot k_{4}} = \frac{ \left( u - m_{2}^{2} - m_{3}^{2} \right) - c_{34} \left( t - m_{2}^{2} - m_{4}^{2} \right) }{s} \left[ \frac{\left(1 + c_{34} \right) \left(1 + c_{43} \right)}{\left(1 - c_{34} c_{43} \right)} \right].
\end{align}
More importantly,
\begin{align}
	c_{13} + c_{23} &= \left[ \frac{s - m_{3}^{2} + m_{4}^{2} - c_{43} \left( s + m_{3}^{2} - m_{4}^{2} \right)}{s} \right] \left[ \frac{\left(1 + c_{34} \right) \left(1 + c_{43} \right)}{\left(1 - c_{34} c_{43} \right)} \right], \\
	c_{14} + c_{24} &= \left[ \frac{s + m_{3}^{2} - m_{4}^{2} - c_{34} \left( s - m_{3}^{2} + m_{4}^{2} \right)}{s} \right] \left[ \frac{\left(1 + c_{34} \right) \left(1 + c_{43} \right)}{\left(1 - c_{34} c_{43} \right)} \right], \\
	c_{13} - 1 &= ..., \\
	c_{14} - c_{34} &= ..., \\
	c_{43} - c_{23} &= ..., \\
	1 - c_{24} &= ..., \\
	c_{13} - c_{43} &= ..., \\
	c_{14} - 1 &= ..., \\
	1 - c_{23} &= ..., \\
	c_{34} - c_{24} &= ....
\end{align}
%%%%%%%%%%%%%%%%%%%%%%%%%%%%%%%%%%%%%%%%%%%%%%%%%%%%%%%%%%%%%%%%%%%%%%%%%%%%%%%%%%%%%%%%%%%%%%%%%%%%%%%%%%%%%%%%%%%%
\section{Center-of-Momentum Frame}
%%%%%%%%%%%%%%%%%%%%%%%%%%%%%%%%%%%%%%%%%%%%%%%%%%%%%%%%%%%%%%%%%%%%%%%%%%%%%%%%%%%%%%%%%%%%%%%%%%%%%%%%%%%%%%%%%%%%
In the center-of-momentum frame you can write the energy-momentum vectors as
\begin{equation}
	p_{1} = \begin{pmatrix} E_{1} & \mathbf{p}_{1} \end{pmatrix}, \qquad p_{2} = \begin{pmatrix} E_{2} & -\mathbf{p}_{1} \end{pmatrix}, \qquad p_{3} = \begin{pmatrix} E_{3} & \mathbf{p}_{3} \end{pmatrix}, \qquad p_{4} = \begin{pmatrix} E_{4} & -\mathbf{p}_{3} \end{pmatrix}.
\end{equation}
From the on-shell relations (\ref{eq:shell_inelastic_slow}) it follows that
\begin{equation}
	\abs{\mathbf{p}_{1}}^{2} = E_{1}^{2} - m_{1}^{2} = E_{2}^{2} - m_{2}^{2}, \qquad \abs{\mathbf{p}_{3}}^{2} = E_{3}^{2} - m_{3}^{2} = E_{4}^{2} - m_{4}^{2}.
\end{equation}
Using $s = -\abs{p_{1} + p_{2}}^{2}$, it follows that
\begin{equation}
	s = \left( E_{1} + E_{2} \right)^{2} \quad \Longrightarrow \quad E_{2} = \sqrt{s} - E_{1}.
\end{equation}
Similarly, using $s = -\abs{p_{3} + p_{4}}^{2}$, it follows that
\begin{equation}
	s = \left( E_{3} + E_{4} \right)^{2} \quad \Longrightarrow \quad E_{4} = \sqrt{s} - E_{3}.
\end{equation}
Note that $s$ must be positive.

Using the on-shell relations, you can find the energies:
\begin{equation}
	E_{1} = \frac{s + m_{1}^{2} - m_{2}^{2}}{2 \sqrt{s}} \quad \Longrightarrow \quad E_{2} = \frac{s - m_{1}^{2} + m_{2}^{2}}{2 \sqrt{s}},
\end{equation}
and
\begin{equation}
	E_{3} = \frac{s + m_{3}^{2} - m_{4}^{2}}{2 \sqrt{s}} \quad \Longrightarrow \quad E_{4} = \frac{s - m_{3}^{2} + m_{4}^{2}}{2 \sqrt{s}}.
\end{equation}
These lead to the magnitude of the spatial momenta:
\begin{equation}
	\abs{\mathbf{p}_{1}} = \frac{\sqrt{\Lambda_{12}(s)}}{2 \sqrt{s}}, \qquad \abs{\mathbf{p}_{3}} = \frac{\sqrt{\Lambda_{34}(s)}}{2 \sqrt{s}}.
\end{equation}
Using $t = -\abs{p_{1} - p_{3}}^{2}$, it follows that
\begin{equation}
	t = \left( E_{1} - E_{3} \right)^{2} - \abs{\mathbf{p}_{1} - \mathbf{p}_{3}}^{2}.
\end{equation}
Similarly, using $u = -\abs{p_{1} - p_{4}}^{2}$, it follows that
\begin{equation}
	u = \left( E_{1} - E_{4} \right)^{2} - \abs{\mathbf{p}_{1} + \mathbf{p}_{3}}^{2}.
\end{equation}
Thus,
\begin{equation}
	u - t = \left( E_{1} - E_{4} \right)^{2} - \left( E_{1} - E_{3} \right)^{2} - 4 \left( \mathbf{p}_{1} \cdot \mathbf{p}_{3} \right),
\end{equation}
and
\begin{equation}
	u + t = \left( E_{1} - E_{4} \right)^{2} + \left( E_{1} - E_{3} \right)^{2} - 2 \left( \abs{\mathbf{p}_{1}}^{2} + \abs{\mathbf{p}_{3}}^{2} \right).
\end{equation}
It follows that
\begin{equation}
	\left( \mathbf{p}_{1} \cdot \mathbf{p}_{3} \right) = \frac{ \left( m_{1}^{2} - m_{2}^{2} \right) \left( m_{3}^{2} - m_{4}^{2} \right) - s \left(u - t\right) }{4s}.
\end{equation}
Alternatively, you also find that
\begin{equation}
	\abs{\mathbf{p}_{1}}^{2} + \abs{\mathbf{p}_{3}}^{2} = \frac{\left( m_{1}^{2} - m_{2}^{2} \right)^{2} +  \left( m_{3}^{2} - m_{4}^{2} \right)^{2} - 2s \left(u + t\right)}{4s}.
\end{equation}
Using
\begin{equation}
	\frac{1}{\abs{\mathbf{p}_{1}} \abs{\mathbf{p}_{3}}} = \frac{1}{\abs{\mathbf{p}_{1}}^{2} + \abs{\mathbf{p}_{3}}^{2}} \left( \frac{\abs{\mathbf{p}_{1}}}{\abs{\mathbf{p}_{3}}} + \frac{\abs{\mathbf{p}_{3}}}{\abs{\mathbf{p}_{1}}} \right).
\end{equation}
From here you can find that
\begin{equation}
	z_{13} \equiv \frac{\mathbf{p}_{1} \cdot \mathbf{p}_{3}}{ \abs{\mathbf{p}_{1}} \abs{\mathbf{p}_{3}} } = \left[ \frac{\Lambda_{12}(s) + \Lambda_{34}(s)}{ \sqrt{\Lambda_{12}(s)} \sqrt{\Lambda_{34}(s)} } \right] \left[ 
	\frac{ \left( m_{1}^{2} - m_{2}^{2} \right) \left( m_{3}^{2} - m_{4}^{2} \right) - s \left(u - t\right) }{\left( m_{1}^{2} - m_{2}^{2} \right)^{2} +  \left( m_{3}^{2} - m_{4}^{2} \right)^{2} - 2s \left(u + t\right)} \right].
\end{equation}
Note that
\begin{align}
	\Lambda_{12}(s) + \Lambda_{34}(s) &= \left( m_{1}^{2} - m_{2}^{2} \right)^{2} +  \left( m_{3}^{2} - m_{4}^{2} \right)^{2} - 2s \left(t + u\right), \\
	\Lambda_{13}(t) + \Lambda_{24}(t) &= \left( m_{1}^{2} - m_{3}^{2} \right)^{2} +  \left( m_{2}^{2} - m_{4}^{2} \right)^{2} - 2t \left(s + u\right), \\
	\Lambda_{14}(u) + \Lambda_{23}(u) &= \left( m_{1}^{2} - m_{4}^{2} \right)^{2} +  \left( m_{2}^{2} - m_{3}^{2} \right)^{2} - 2u \left(s + t\right).
\end{align}
The right-hand side has dependence on one Mandelstam invariant after using (\ref{stu_1234}).
%%%%%%%%%%%%%%%%%%%%%%%%%%%%%%%%%%%%%%%%%%%%%%%%%%%%%%%%%%%%%%%%%%%%%%%%%%%%%%%%%%%%%%%%%%%%%%%%%%%%%%%%%%%%%%%%%%%%
\subsection{Speed and Rapidity}
%%%%%%%%%%%%%%%%%%%%%%%%%%%%%%%%%%%%%%%%%%%%%%%%%%%%%%%%%%%%%%%%%%%%%%%%%%%%%%%%%%%%%%%%%%%%%%%%%%%%%%%%%%%%%%%%%%%%
The speed of each external quantum is
\begin{align}
	\abs{\mathbf{v}_{1}} &= \frac{\sqrt{\Lambda_{12}(s)}}{s + m_{1}^{2} - m_{2}^{2}}, \\
	\abs{\mathbf{v}_{2}} &= \frac{\sqrt{\Lambda_{12}(s)}}{s - m_{1}^{2} + m_{2}^{2}}, \\
	\abs{\mathbf{v}_{3}} &= \frac{\sqrt{\Lambda_{34}(s)}}{s + m_{3}^{2} - m_{4}^{2}}, \\
	\abs{\mathbf{v}_{4}} &= \frac{\sqrt{\Lambda_{34}(s)}}{s - m_{3}^{2} + m_{4}^{2}}.
\end{align}
The rapidity of each external quantum is
\begin{align}
	\rho_{1} &= \frac{1}{2} \log{\left[ \frac{s + m_{1}^{2} - m_{2}^{2} + \sqrt{\Lambda_{12}(s)}}{s + m_{1}^{2} - m_{2}^{2} - \sqrt{\Lambda_{12}(s)}} \right]}, \\
	\rho_{2} &= \frac{1}{2} \log{\left[ \frac{s - m_{1}^{2} + m_{2}^{2} + \sqrt{\Lambda_{12}(s)}}{s - m_{1}^{2} + m_{2}^{2} - \sqrt{\Lambda_{12}(s)}} \right]}, \\
	\rho_{3} &= \frac{1}{2} \log{\left[ \frac{s + m_{3}^{2} - m_{4}^{2} + \sqrt{\Lambda_{34}(s)}}{s + m_{3}^{2} - m_{4}^{2} - \sqrt{\Lambda_{34}(s)}} \right]}, \\
	\rho_{4} &= \frac{1}{2} \log{\left[ \frac{s - m_{3}^{2} + m_{4}^{2} + \sqrt{\Lambda_{34}(s)}}{s - m_{3}^{2} + m_{4}^{2} - \sqrt{\Lambda_{34}(s)}} \right]}.
\end{align}
The sum of the incoming rapidities is
\begin{equation}
	\rho_{1} + \rho_{2} = \frac{1}{2} \log{ \left[ \frac{s - m_{1}^{2} - m_{2}^{2} + \sqrt{\Lambda_{12}(s)}}{s - m_{1}^{2} - m_{2}^{2} - \sqrt{\Lambda_{12}(s)}} \right] }.
\end{equation}
Similarly, the sum of the outgoing rapidities is
\begin{equation}
	\rho_{3} + \rho_{4} = \frac{1}{2} \log{ \left[ \frac{s - m_{3}^{2} - m_{4}^{2} + \sqrt{\Lambda_{34}(s)}}{s - m_{3}^{2} - m_{4}^{2} - \sqrt{\Lambda_{34}(s)}} \right] }.
\end{equation}
%%%%%%%%%%%%%%%%%%%%%%%%%%%%%%%%%%%%%%%%%%%%%%%%%%%%%%%%%%%%%%%%%%%%%%%%%%%%%%%%%%%%%%%%%%%%%%%%%%%%%%%%%%%%%%%%%%%%
\subsection{Physical Scattering Region}
%%%%%%%%%%%%%%%%%%%%%%%%%%%%%%%%%%%%%%%%%%%%%%%%%%%%%%%%%%%%%%%%%%%%%%%%%%%%%%%%%%%%%%%%%%%%%%%%%%%%%%%%%%%%%%%%%%%%
Inside the physical scattering region, you must require that $\abs{\mathbf{p}_{1}}$ be real and positive. This leads to the condition
\begin{equation}
	s > (m_{1} + m_{2})^{2}.
\end{equation}
Similarly, requiring that $\abs{\mathbf{p}_{3}}$ be real and positive leads to the condition
\begin{equation}
	s > (m_{3} + m_{4})^{2}.
\end{equation}
Thus, we take the overlap of these two regions:
\begin{equation}
	s > \operatorname{max}{\left\lbrace (m_{1} + m_{2})^{2}, (m_{3} + m_{4})^{2} \right\rbrace}.
\end{equation}
Other requirements include that $\abs{ \mathbf{p}_{1} - \mathbf{p}_{3} }^{2}$ and $\abs{ \mathbf{p}_{1} - \mathbf{p}_{3} }^{2}$ be real and positive. These lead to two conditions:
\begin{equation}
	\left(E_{1} - E_{3} \right)^{2} > t, \qquad \left(E_{1} - E_{4} \right)^{2} > u.
\end{equation}
These conditions translate to the statements
\begin{equation}
	\left(m_{1}^{2} - m_{2}^{2} - m_{3}^{2} + m_{4}^{2} \right)^{2} > 4st, \qquad \left(m_{1}^{2} - m_{2}^{2} + m_{3}^{2} - m_{4}^{2} \right)^{2} > 4su.
\end{equation}
%%%%%%%%%%%%%%%%%%%%%%%%%%%%%%%%%%%%%%%%%%%%%%%%%%%%%%%%%%%%%%%%%%%%%%%%%%%%%%%%%%%%%%%%%%%%%%%%%%%%%%%%%%%%%%%%%%%%
\subsection{Scattering Regimes}
%%%%%%%%%%%%%%%%%%%%%%%%%%%%%%%%%%%%%%%%%%%%%%%%%%%%%%%%%%%%%%%%%%%%%%%%%%%%%%%%%%%%%%%%%%%%%%%%%%%%%%%%%%%%%%%%%%%%
...