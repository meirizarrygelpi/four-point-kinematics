\chapter{Null Kinematics}
%%%%%%%%%%%%%%%%%%%%%%%%%%%%%%%%%%%%%%%%%%%%%%%%%%%%%%%%%%%%%%%%%%%%%%%%%%%%%%%%%%%%%%%%%%%%%%%%%%%%%%%%%%%%%%%%%%%%
In this chapter I will consider the kinematics of a 2-to-2 scattering process that involves only identical massless quanta:
\begin{equation}
	\abs{p_{1}}^{2} = \abs{p_{2}}^{2} = \abs{p_{3}}^{2} = \abs{p_{4}}^{2} = 0.
\end{equation}
This is an elastic process. For example,
\begin{equation}
	\gamma(p_{1}) + \gamma(p_{2}) \longrightarrow \gamma(p_{3}) + \gamma(p_{4}).
\end{equation}
For this process, (\ref{eq:stu_slow}) becomes
\begin{equation}
	s + t + u = 0.
\end{equation}
%%%%%%%%%%%%%%%%%%%%%%%%%%%%%%%%%%%%%%%%%%%%%%%%%%%%%%%%%%%%%%%%%%%%%%%%%%%%%%%%%%%%%%%%%%%%%%%%%%%%%%%%%%%%%%%%%%%%
\section{Mandelstam Invariants}
%%%%%%%%%%%%%%%%%%%%%%%%%%%%%%%%%%%%%%%%%%%%%%%%%%%%%%%%%%%%%%%%%%%%%%%%%%%%%%%%%%%%%%%%%%%%%%%%%%%%%%%%%%%%%%%%%%%%
For convenience, one defines the three Mandelstam invariants:
\begin{equation}
	s = -\abs{p_{1} + p_{2}}^{2}, \qquad t = -\abs{p_{1} - p_{3}}^{2}, \qquad u = -\abs{p_{1} - p_{4}}^{2}. 
\end{equation}
Due to the conservation constraint, these three invariants satisfy a linear relation:
\begin{equation}
	s + t + u = 0.
\end{equation}
%%%%%%%%%%%%%%%%%%%%%%%%%%%%%%%%%%%%%%%%%%%%%%%%%%%%%%%%%%%%%%%%%%%%%%%%%%%%%%%%%%%%%%%%%%%%%%%%%%%%%%%%%%%%%%%%%%%%
\section{Dual Spacetime}
%%%%%%%%%%%%%%%%%%%%%%%%%%%%%%%%%%%%%%%%%%%%%%%%%%%%%%%%%%%%%%%%%%%%%%%%%%%%%%%%%%%%%%%%%%%%%%%%%%%%%%%%%%%%%%%%%%%%
You can solve the conservation constraint by introducing dual spacetime coordinates:
\begin{equation}
	p_{1} = d_{\red{1}} - d_{\red{2}}, \quad p_{2} = d_{\red{2}} - d_{\red{3}}, \quad p_{4} = d_{\red{4}} - d_{\red{3}}, \quad p_{3} = d_{\red{1}} - d_{\red{4}}.
	\label{eq:d_red_massless}
\end{equation}
Thus, each energy-momentum vector becomes a distance interval in a dual spacetime. However, the solution (\ref{eq:d_red_massless}) is not the only one allowed. Indeed, two other solutions are allowed:
\begin{equation}
	{-p_{3}} = d_{\blue{1}} - d_{\blue{2}}, \quad p_{1} = d_{\blue{2}} - d_{\blue{3}}, \quad p_{4} = d_{\blue{4}} - d_{\blue{3}}, \quad -p_{2} = d_{\blue{1}} - d_{\blue{4}},
	\label{eq:d_blue_massless}
\end{equation}
and
\begin{equation}
	p_{2} = d_{\green{1}} - d_{\green{2}}, \quad -p_{3} = d_{\green{2}} - d_{\green{3}}, \quad p_{4} = d_{\green{4}} - d_{\green{3}}, \quad -p_{1} = d_{\green{1}} - d_{\green{4}}.
	\label{eq:d_green_massless}
\end{equation}
I will refer to (\ref{eq:d_red_massless}) as the \red{red} planar class, (\ref{eq:d_blue_massless}) as the \blue{blue} planar class, and (\ref{eq:d_green_massless}) as the \green{green} planar class. In each planar class, the dual spacetime coordinates describe the positions of four points in the dual spacetime. These four points can be taken as the vertices of a tetrahedron. Many kinematic quantities can be understood in terms of the geometry of these tetrahedra.
%%%%%%%%%%%%%%%%%%%%%%%%%%%%%%%%%%%%%%%%%%%%%%%%%%%%%%%%%%%%%%%%%%%%%%%%%%%%%%%%%%%%%%%%%%%%%%%%%%%%%%%%%%%%%%%%%%%%
\section{Simplicial Invariants}
%%%%%%%%%%%%%%%%%%%%%%%%%%%%%%%%%%%%%%%%%%%%%%%%%%%%%%%%%%%%%%%%%%%%%%%%%%%%%%%%%%%%%%%%%%%%%%%%%%%%%%%%%%%%%%%%%%%%
A tetrahedron is a 3-simplex. As such, it contains four vertices (0-simplex), six edges (1-simplex), four triangular faces (2-simplex), and one tetrahedron (3-simplex). The $n$-volume of an $n$-simplex is found by evaluating an $(n+1)$-point Cayley-Menger determinant. 
%%%%%%%%%%%%%%%%%%%%%%%%%%%%%%%%%%%%%%%%%%%%%%%%%%%%%%%%%%%%%%%%%%%%%%%%%%%%%%%%%%%%%%%%%%%%%%%%%%%%%%%%%%%%%%%%%%%%
\subsection{1-Simplex Invariants}
%%%%%%%%%%%%%%%%%%%%%%%%%%%%%%%%%%%%%%%%%%%%%%%%%%%%%%%%%%%%%%%%%%%%%%%%%%%%%%%%%%%%%%%%%%%%%%%%%%%%%%%%%%%%%%%%%%%%
Given two dual spacetime positions, the 2-point Cayley-Menger determinant is given by:
\begin{equation}
	C_{ij} = -\frac{1}{2} \det{
	\begin{pmatrix}
	0 & \abs{d_{ij}}^{2} & 1 \\
	\abs{d_{ij}}^{2} & 0 & 1 \\
	1 & 1 & 0
	\end{pmatrix}
	}.
\end{equation}
Since a tetrahedron has six edges, there are six possible 1-simplex invariants. These invariants will be either squared masses or Mandelstam invariants. For the \red{red} tetrahedron you have
\begin{equation}
\begin{split}
	C_{\red{12}} = 0, \qquad C_{\red{13}} &= s, \qquad C_{\red{14}} = 0, \\
	C_{\red{23}} = 0, \qquad C_{\red{24}} &= t, \qquad C_{\red{34}} = 0. \\
\end{split}
\end{equation}
Similarly, for the \blue{blue} tetrahedron you have
\begin{equation}
\begin{split}
	C_{\blue{12}} = 0, \qquad C_{\blue{13}} &= t, \qquad C_{\blue{14}} = 0, \\
	C_{\blue{23}} = 0, \qquad C_{\blue{24}} &= u, \qquad C_{\blue{34}} = 0. \\
\end{split}
\end{equation}
Finally, for the \green{green} tetrahedron you have
\begin{equation}
\begin{split}
	C_{\green{12}} = 0, \qquad C_{\green{13}} &= u, \qquad C_{\green{14}} = 0, \\
	C_{\green{23}} = 0, \qquad C_{\green{24}} &= s, \qquad C_{\green{34}} = 0. \\
\end{split}
\end{equation}
%%%%%%%%%%%%%%%%%%%%%%%%%%%%%%%%%%%%%%%%%%%%%%%%%%%%%%%%%%%%%%%%%%%%%%%%%%%%%%%%%%%%%%%%%%%%%%%%%%%%%%%%%%%%%%%%%%%%
\subsection{2-Simplex Invariants}
%%%%%%%%%%%%%%%%%%%%%%%%%%%%%%%%%%%%%%%%%%%%%%%%%%%%%%%%%%%%%%%%%%%%%%%%%%%%%%%%%%%%%%%%%%%%%%%%%%%%%%%%%%%%%%%%%%%%
Given three dual spacetime positions, the 3-point Cayley-Menger determinant is given by:
\begin{equation}
	C_{ijk} = \det{
	\begin{pmatrix}
	0 & \abs{d_{ij}}^{2} & \abs{d_{ik}}^{2} & 1 \\
	\abs{d_{ij}}^{2} & 0 & \abs{d_{jk}}^{2} & 1 \\
	\abs{d_{ik}}^{2} & \abs{d_{jk}}^{2} & 0 & 1 \\
	1 & 1 & 1 & 0
	\end{pmatrix}
	}.
\end{equation}
Since a tetrahedron has four triangular faces, there are four possible 2-simplex invariants. Each of these invariants has the form
\begin{equation}
	\Lambda_{IJ}(x) = \kallen{x}{m_{I}}{m_{J}},
\end{equation}
where $x$ is a Mandelstam invariant. This function is also known as the K\"{a}ll\'{e}n function. For the \red{red} tetrahedron you have
\begin{equation}
	C_{\red{123}} = C_{\red{134}} = s^{2}, \qquad C_{\red{124}} = C_{\red{234}} = t^{2}.
\end{equation}
Similarly, for the \blue{blue} tetrahedron you have
\begin{equation}
	C_{\blue{123}} = C_{\blue{134}} = t^{2}, \qquad C_{\blue{124}} = C_{\blue{234}} = u^{2}.
\end{equation}
Finally, for the \green{green} tetrahedron you have
\begin{equation}
	C_{\green{123}} = C_{\green{134}} = u^{2}, \qquad C_{\green{124}} = C_{\green{234}} = s^{2}.
\end{equation}
Note that all 2-simplex invariants are strictly positive by definition.
%%%%%%%%%%%%%%%%%%%%%%%%%%%%%%%%%%%%%%%%%%%%%%%%%%%%%%%%%%%%%%%%%%%%%%%%%%%%%%%%%%%%%%%%%%%%%%%%%%%%%%%%%%%%%%%%%%%%
\subsection{3-Simplex Invariants}
%%%%%%%%%%%%%%%%%%%%%%%%%%%%%%%%%%%%%%%%%%%%%%%%%%%%%%%%%%%%%%%%%%%%%%%%%%%%%%%%%%%%%%%%%%%%%%%%%%%%%%%%%%%%%%%%%%%%
Given four dual spacetime positions, the 4-point Cayley-Menger determinant is given by:
\begin{equation}
	C_{ijkl} = -\frac{1}{2} \det{
	\begin{pmatrix}
	0 & \abs{d_{ij}}^{2} & \abs{d_{ik}}^{2} & \abs{d_{il}}^{2} & 1 \\
	\abs{d_{ij}}^{2} & 0 & \abs{d_{jk}}^{2} & \abs{d_{jl}}^{2} & 1 \\
	\abs{d_{ik}}^{2} & \abs{d_{jk}}^{2} & 0 & \abs{d_{kl}}^{2} & 1 \\
	\abs{d_{il}}^{2} & \abs{d_{jl}}^{2} & \abs{d_{kl}}^{2} & 0 & 1 \\
	1 & 1 & 1 & 1 & 0
	\end{pmatrix}
	}.
\end{equation}
A tetrahedron has only one possible 3-simplex invariant. Indeed,
\begin{equation}
	C_{\red{1234}} = C_{\blue{1234}} = C_{\green{1234}} = stu.
\end{equation}
%%%%%%%%%%%%%%%%%%%%%%%%%%%%%%%%%%%%%%%%%%%%%%%%%%%%%%%%%%%%%%%%%%%%%%%%%%%%%%%%%%%%%%%%%%%%%%%%%%%%%%%%%%%%%%%%%%%%
\section{Dual Conformal Invariants}
%%%%%%%%%%%%%%%%%%%%%%%%%%%%%%%%%%%%%%%%%%%%%%%%%%%%%%%%%%%%%%%%%%%%%%%%%%%%%%%%%%%%%%%%%%%%%%%%%%%%%%%%%%%%%%%%%%%%
Each of the 1-simplex invariants has the form $\abs{d_{i} - d_{j}}^{2}$. This is not only Lorentz invariant, but dual Poincar\'{e} invariant. You can study dual conformal invariants by constructing a dual conformal ratio with a quartet of dual spacetime coordinates:
\begin{equation}
	\crat{i}{j}{k}{l} \equiv \frac{\abs{d_{ik}}^{2} \abs{d_{jl}}^{2}}{\abs{d_{il}}^{2} \abs{d_{jk}}^{2}}.
\end{equation}
In four-point scattering there is a unique quartet of dual spacetime coordinates. However, there are six inequivalent permutations of these coordinates, and thus, six possible values for the dual conformal ratio.

Due to the appearance of only massless quanta, all dual conformal invariants are trivial. For the \red{red} tetrahedron, you have
\begin{equation}
\begin{split}
	\crat{\red{1}}{\red{2}}{\red{3}}{\red{4}} \rightarrow \infty, \qquad
	\crat{\red{1}}{\red{3}}{\red{4}}{\red{2}} &= 1, \qquad
	\crat{\red{1}}{\red{4}}{\red{2}}{\red{3}} \rightarrow 0, \\
	\crat{\red{1}}{\red{2}}{\red{4}}{\red{3}} \rightarrow 0, \qquad
	\crat{\red{1}}{\red{3}}{\red{2}}{\red{4}} &= 1, \qquad
	\crat{\red{1}}{\red{4}}{\red{3}}{\red{2}} \rightarrow \infty.
\end{split}
\end{equation}
Similarly, for the \blue{blue} tetrahedron, you have
\begin{equation}
\begin{split}
	\crat{\blue{1}}{\blue{2}}{\blue{3}}{\blue{4}} \rightarrow \infty, \qquad
	\crat{\blue{1}}{\blue{3}}{\blue{4}}{\blue{2}} &= 1, \qquad
	\crat{\blue{1}}{\blue{4}}{\blue{2}}{\blue{3}} \rightarrow 0, \\
	\crat{\blue{1}}{\blue{2}}{\blue{4}}{\blue{3}} \rightarrow 0, \qquad
	\crat{\blue{1}}{\blue{3}}{\blue{2}}{\blue{4}} &= 1, \qquad
	\crat{\blue{1}}{\blue{4}}{\blue{3}}{\blue{2}} \rightarrow \infty.
\end{split}
\end{equation}
Finally, for the \green{green} tetrahedron, you have
\begin{equation}
\begin{split}
	\crat{\green{1}}{\green{2}}{\green{3}}{\green{4}} \rightarrow \infty, \qquad
	\crat{\green{1}}{\green{3}}{\green{4}}{\green{2}} &= 1, \qquad
	\crat{\green{1}}{\green{4}}{\green{2}}{\green{3}} \rightarrow 0, \\
	\crat{\green{1}}{\green{2}}{\green{4}}{\green{3}} \rightarrow 0, \qquad
	\crat{\green{1}}{\green{3}}{\green{2}}{\green{4}} &= 1, \qquad
	\crat{\green{1}}{\green{4}}{\green{3}}{\green{2}} \rightarrow \infty.
\end{split}
\end{equation}
Note that
\begin{equation}
	\crat{1}{2}{3}{4} \crat{1}{3}{4}{2} \crat{1}{4}{2}{3} = 1,
\end{equation}
which is analogous to the constraint satisfied by the three Mandelstam invariants.
%%%%%%%%%%%%%%%%%%%%%%%%%%%%%%%%%%%%%%%%%%%%%%%%%%%%%%%%%%%%%%%%%%%%%%%%%%%%%%%%%%%%%%%%%%%%%%%%%%%%%%%%%%%%%%%%%%%%
\section{Center-of-Momentum Frame}
%%%%%%%%%%%%%%%%%%%%%%%%%%%%%%%%%%%%%%%%%%%%%%%%%%%%%%%%%%%%%%%%%%%%%%%%%%%%%%%%%%%%%%%%%%%%%%%%%%%%%%%%%%%%%%%%%%%%
In the center-of-momentum frame you write the energy-momentum vectors as
\begin{equation}
	p_{1} = \begin{pmatrix} E_{1} & \mathbf{p}_{1} \end{pmatrix}, \qquad p_{2} = \begin{pmatrix} E_{2} & -\mathbf{p}_{1} \end{pmatrix}, \qquad p_{3} = \begin{pmatrix} E_{3} & \mathbf{p}_{3} \end{pmatrix}, \qquad p_{4} = \begin{pmatrix} E_{4} & -\mathbf{p}_{3} \end{pmatrix}.
\end{equation}
One of the first things to notice is that
\begin{equation}
	s = (E_{1} + E_{2})^{2} = (E_{3} + E_{4})^{2}.
\end{equation}
Thus, $s$ can be interpreted as the total energy in the center-of-momentum frame. It also follows that $s$ must be positive.

From the on-shell constraints, it follows that
\begin{equation}
\begin{split}
	E_{1} = \sqrt{m^{2} + \abs{\mathbf{p}_{1}}^{2}}, &\qquad E_{2} = \sqrt{m^{2} + \abs{\mathbf{p}_{1}}^{2}}, \\
	E_{3} = \sqrt{m^{2} + \abs{\mathbf{p}_{3}}^{2}}, &\qquad E_{4} = \sqrt{m^{2} + \abs{\mathbf{p}_{3}}^{2}}.
\end{split}
\end{equation}
Using the relations
\begin{equation}
	E_{2} = \sqrt{s} - E_{1}, \qquad E_{4} = \sqrt{s} - E_{3},
\end{equation}
you find that
\begin{equation}
	\abs{\mathbf{p}_{1}} = \abs{\mathbf{p}_{3}} = \frac{\sqrt{s - 4m^{2}}}{2}.
\end{equation}
Thus,
\begin{equation}
	E_{1} = E_{2} = E_{3} = E_{4} = \frac{\sqrt{s}}{2}.
\end{equation}
%%%%%%%%%%%%%%%%%%%%%%%%%%%%%%%%%%%%%%%%%%%%%%%%%%%%%%%%%%%%%%%%%%%%%%%%%%%%%%%%%%%%%%%%%%%%%%%%%%%%%%%%%%%%%%%%%%%%
\subsection{Speed and Rapidity}
%%%%%%%%%%%%%%%%%%%%%%%%%%%%%%%%%%%%%%%%%%%%%%%%%%%%%%%%%%%%%%%%%%%%%%%%%%%%%%%%%%%%%%%%%%%%%%%%%%%%%%%%%%%%%%%%%%%%
A quantum with mass $m$, spatial momentum $\mathbf{p}$, and energy $E$ has a speed $\abs{\mathbf{v}}$ given by
\begin{equation}
	\abs{\mathbf{v}} = \frac{\sqrt{E^{2} - m^{2}}}{E^{2}} = \frac{\abs{\mathbf{p}}}{\sqrt{m^{2} + \abs{\mathbf{p}}^{2}}} = \frac{\abs{\mathbf{p}}}{E}.
\end{equation}
If the quantum is physical, then $E \geq m$, or equivalently, $\abs{\mathbf{p}} < E$. Both of these conditions lead to a bound on speed:
\begin{equation}
	0 \leq \abs{\mathbf{v}} < 1.
\end{equation}
In contrast with energy, mass, and spatial momentum, speed and velocity are dimensionless. However, energy and spatial momentum are conserved, but velocity is not. The speed of each external quantum is
\begin{equation}
	\abs{\mathbf{v}_{1}} = \abs{\mathbf{v}_{2}} = \abs{\mathbf{v}_{3}} = \abs{\mathbf{v}_{4}} = \sqrt{1 - \frac{4 m^{2}}{s}}.
\end{equation}
Given the speed of a quantum, you can find its rapidity $\eta$ via
\begin{equation}
	\eta = \operatorname{arctanh}{\abs{\mathbf{v}}} = \frac{1}{2} \log{\left( \frac{1 + \abs{\mathbf{v}}}{1 - \abs{\mathbf{v}}} \right)}.
\end{equation}
Since the speed of a physical quantum is bounded, the rapidity of a physical quantum is bounded from below:
\begin{equation}
	0 \leq \eta < \infty.
\end{equation}
The rapidity of each external quantum is
\begin{equation}
	\eta_{1} = \eta_{2} = \eta_{3} = \eta_{4} = \frac{1}{2} \log{\left[\frac{\sqrt{s} + \sqrt{s - 4 m^{2}}}{\sqrt{s} - \sqrt{s - 4 m^{2}}}\right]}.
\end{equation}
The sum of the incoming rapidites is equal to the sum of the outgoing rapidities:
\begin{equation}
	\eta_{1} + \eta_{2} = \eta_{3} + \eta_{4} = \log{\left[\frac{\sqrt{s} + \sqrt{s - 4 m^{2}}}{\sqrt{s} - \sqrt{s - 4 m^{2}}}\right]}.
\end{equation}
%%%%%%%%%%%%%%%%%%%%%%%%%%%%%%%%%%%%%%%%%%%%%%%%%%%%%%%%%%%%%%%%%%%%%%%%%%%%%%%%%%%%%%%%%%%%%%%%%%%%%%%%%%%%%%%%%%%%
\subsection{Physical Scattering Region}
%%%%%%%%%%%%%%%%%%%%%%%%%%%%%%%%%%%%%%%%%%%%%%%%%%%%%%%%%%%%%%%%%%%%%%%%%%%%%%%%%%%%%%%%%%%%%%%%%%%%%%%%%%%%%%%%%%%%
...
%%%%%%%%%%%%%%%%%%%%%%%%%%%%%%%%%%%%%%%%%%%%%%%%%%%%%%%%%%%%%%%%%%%%%%%%%%%%%%%%%%%%%%%%%%%%%%%%%%%%%%%%%%%%%%%%%%%%
\subsubsection{Regge Scattering}
%%%%%%%%%%%%%%%%%%%%%%%%%%%%%%%%%%%%%%%%%%%%%%%%%%%%%%%%%%%%%%%%%%%%%%%%%%%%%%%%%%%%%%%%%%%%%%%%%%%%%%%%%%%%%%%%%%%%
Regge scattering is the regime of large (unphysical) $z_{13}$. This corresponds to
\begin{equation}
	\frac{t}{s} \rightarrow \infty.
\end{equation}
As a corollary, you have
\begin{equation}
	u = - s - t \quad \Longrightarrow \quad \frac{u}{s} \rightarrow -\infty.
\end{equation}
%%%%%%%%%%%%%%%%%%%%%%%%%%%%%%%%%%%%%%%%%%%%%%%%%%%%%%%%%%%%%%%%%%%%%%%%%%%%%%%%%%%%%%%%%%%%%%%%%%%%%%%%%%%%%%%%%%%%
\subsubsection{Forward Scattering}
%%%%%%%%%%%%%%%%%%%%%%%%%%%%%%%%%%%%%%%%%%%%%%%%%%%%%%%%%%%%%%%%%%%%%%%%%%%%%%%%%%%%%%%%%%%%%%%%%%%%%%%%%%%%%%%%%%%%
Forward scattering is the regime of small (physical) scattering angles (i.e. $z_{13} \rightarrow 1$). This can be stated as
\begin{equation}
	\frac{t}{s} \rightarrow 0.
\end{equation}
As a corollary, you have
\begin{equation}
	u = - s - t \quad \Longrightarrow \quad \frac{u}{s} \text{ fixed}.
\end{equation}
%%%%%%%%%%%%%%%%%%%%%%%%%%%%%%%%%%%%%%%%%%%%%%%%%%%%%%%%%%%%%%%%%%%%%%%%%%%%%%%%%%%%%%%%%%%%%%%%%%%%%%%%%%%%%%%%%%%%
\subsubsection{Backward Scattering}
%%%%%%%%%%%%%%%%%%%%%%%%%%%%%%%%%%%%%%%%%%%%%%%%%%%%%%%%%%%%%%%%%%%%%%%%%%%%%%%%%%%%%%%%%%%%%%%%%%%%%%%%%%%%%%%%%%%%
Backward scattering is the regime of large (physical) scattering angles (i.e. $z_{13} \rightarrow -1$). This can be stated as
\begin{equation}
	\frac{u}{s} \rightarrow 0.
\end{equation}
As a corollary, you have
\begin{equation}
	t = - s - u \quad \Longrightarrow \quad \frac{t}{s} \text{ fixed}.
\end{equation}
%%%%%%%%%%%%%%%%%%%%%%%%%%%%%%%%%%%%%%%%%%%%%%%%%%%%%%%%%%%%%%%%%%%%%%%%%%%%%%%%%%%%%%%%%%%%%%%%%%%%%%%%%%%%%%%%%%%%
\subsubsection{Fixed-Angle Scattering}
%%%%%%%%%%%%%%%%%%%%%%%%%%%%%%%%%%%%%%%%%%%%%%%%%%%%%%%%%%%%%%%%%%%%%%%%%%%%%%%%%%%%%%%%%%%%%%%%%%%%%%%%%%%%%%%%%%%%
Fixed-angle scattering is the regime of (physical) fixed-angle. This can be stated as
\begin{equation}
	\frac{t}{s} \text{ fixed}.
\end{equation}
As a corollary, you have
\begin{equation}
	u = - s - t \quad \Longrightarrow \quad \frac{u}{s} \text{ fixed}.
\end{equation}